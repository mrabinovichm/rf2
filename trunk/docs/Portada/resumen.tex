\begin{abstract}
\begin{itshape}
\dsp

El presente documento describe el prototipo Recarga Fácil por Radio Frecuencia, RF$^{2}$, realizado como proyecto de fin de carrera de Ingeniería Eléctrica en la Universidad de la República entre marzo de 2010 y julio de 2011. El mismo consiste en un sistema embebido para recarga y consulta de tarjetas RFID, como las que se utilizan hoy día en el Sistema de Transporte Metropolitano, y fue diseñado para operar de forma autónoma interactuando directamente con el usuario.

\bigskip
El hardware fue enteramente diseñado por el grupo de trabajo a excepción de la single board computer. Las herramientas de software utilizadas son open source, así como también las bibliotecas usadas para desarrollar la aplicación final. El diseño, la fabricación, y el armado del prototipo, fueron realizados en su totalidad en Uruguay.

\end{itshape}
\end{abstract}