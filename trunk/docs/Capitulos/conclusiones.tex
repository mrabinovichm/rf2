\chapter{Conclusiones}

\section{Conclusión final}
El mundo de la tecnología RFID está poco explorado en nuestro país, éste 
tal vez sea el primer proyecto que incluye el diseño y fabricación de un 
lector/escritor RFID capaz de operar con tarjetas sin contacto en la banda de
frecuencia de 13,56 MHz.

Las posibilidades de aplicación son muy bastas, entre las que se pueden 
nombrar, sistemas de transporte, pasaporte electrónico, identificación 
civil, tarjetas de crédito y/o débito, salud, telefonía, control de acceso,
entre otras.

El aporte realizado en este campo es tan solo una primera aproximación y aún 
queda mucho por hacer al respecto. Sobre este proyecto en particular es
necesario mejorar varios aspectos antes de pasar de la fase de prototipo
a la de producción.

\bigskip
Repasando en particular los criterios de éxito, el proyecto ha resultado satisfactorio
porque se logró construir un dispositivo capaz de consultar y recargar tarjetas RFID,
aunque la solución alcanzada no sea estrictamente igual a la propuesta en el comienzo.

\bigskip
En lo que tiene que ver con el módulo de seguridad, SAM, no fue posible su empleo 
por no lograr integrar el lector de tarjetas de contacto con la biblioteca pcsclite.
La función de derivar las claves, suministradas por la SAM, fue sustituída agregándolas
directamente en el código de la aplicación final. En cuanto a la comunicación
segura con un servidor establecida por la SAM, se descartó su uso en una de 
las replanificaciones de las tareas a cumplir.

\bigskip
El enfoque de este proyecto está dirigido al uso en un sistema de transporte, pero 
puede ser empleado en cualquiera de las aplicaciones ya mencionadas sin necesidad
de hacer grandes modificaciones a nivel de software.

\bigskip
En suma, el equipo está conforme con el trabajo realizado.

\bigskip
Para finalizar, se sugiere que el Instituto de Ingeniería Eléctrica, y el Instituto de Computación, se integren en proyectos que involucren hardware que permite portar sistemas operativos como ser Linux (de los más usados en sistemas embebidos).

\section{Ventajas y desventajas}
La ventaja principal de este prototipo comparada con el sistema que se usa hoy 
en día mediante computadoras de escritorio, es su considerable menor consumo de 
energía, factor a tener en cuenta en un sistema 24/7 como es el transporte de 
pasajeros.
La otra gran ventaja es que no incorpora una impresora de ticket, esto reduce el
tamaño del prototipo así como disminuye su costo de mantenimiento en todo lo que
tenga que ver con cambio de rollos de papel, atasco del mismo y cambio de piezas
de la impresora. 
Los puntos mencionadas antes van de la mano con el tema ecología, tan en boga
en estos días y por lo que se hace muy poco.

\bigskip
Entre las ventajas que se pueden hallar en este dispositivo es que el lector/escritor de tarjetas RFID es un diseño realizado en PCB de dos capas que lo hace más sencillo y económico que el diseño del lector/escritor OpenPCD, y al igual que este último es compatible con la biblioteca open source conocida como librfid.\\
Por su lado, en el lector de tarjetas de contacto se debe destacar su simplicidad, ya que no cuenta con ningún tipo de hardware específico (ASIC) que cumpla con el estandar ISO7816, sólo es necesario tener disponible un puerto serial (UART) y un par de puertos de entrada/salida de propósito general. Esto lo hace portable a cualquier SBC que cuente con los puertos detallados anteriormente.

\bigskip
Algo que no se tiene en cuenta muchas veces es qué tan simple puede resultar
el armado de las partes del dispositivo, en este caso se pensó en un 
método simple y rápido donde no se emplearan cables para los conexionados.
El ensamblado entre sí de las placas de hardware que conforman el prototipo es 
una tarea simple que puede ser hecha por un niño, como si armara su mecano.
Las distintas partes encajan una encima de otra y son aseguradas a través
de un separador con tuerca, no existiendo posibilidad que sean conectadas al 
revés.

\section{Aprendizaje}
Los conceptos aprendidos sobre reglas de diseño de lectores/escritores de 
tarjetas sin contacto, compatibles con la norma ISO14443, son muy importantes
para llevar a cabo mejoras y modificaciones futuras. Las dificultades afrontadas
durante la fase inicial, generadas por errores en las ecuaciones suministradas
por una de las notas de aplicación usada, y la falta de soporte técnico por
parte del fabricante (solamente brindan soporte a empresas), permitió ver la 
necesidad del uso de instrumentos de medición adecuados para validar el 
diseño y comparar con los resultados teóricos. La moraleja en cuanto a los errores cometidos en la etapa de diseño es, el ser cautos a la hora de confiar 100\% en las publicaciones que son usadas como referencia, no debiendo ser tomadas como dogmas.

\bigskip
El manejo de herramientas de diseño CAD, fue de mucha utilidad para luego
llevar adelante la fabricación de las placas de circuito impreso, se aprendió a utilizar
no sólo una, sino dos aplicaciones ya que una de ellas (KiCad) no permite trazar
pistas en forma de curva, cosa necesaria para el diseño del lector/escritor RFID.

\bigskip
A nivel de software se dejó de ser un simple usuario del sistema operativo
Linux, adquiriendo conocimientos para croscompilar su Kernel, e incorporar
modificaciones y parches que permitieran poner en funcionamiento el hardware
que conforma el prototipo. Se debió entender el funcionamiento de buena parte
de las bibliotecas usadas, para modificarlas y que fuera posible su uso en la 
aplicación final.\\
La elección de la herramienta de desarrollo es muy importante para obtener buenos resultados.

\bigskip
En cuanto al cliente, se entendió que maneja tiempos y prioridades distintas. Éstas últimas
podrían cambiar durante el transcurso de un proyecto, por lo que no podía ser 
tenido en cuenta como cliente. La IM pasó a ser un actor que suministró ambos tipos de
tarjetas RFID y SAM, y el dispositivo OpenPCD. \\
Se aprendió que los mecanismos de compra en la IM son muy complejos y no tienen flexibilidad
a la hora de hacer compras en el exterior.

\bigskip
Por último, pero no menos importante, lo que parece salir en dos días puede demorar dos meses por lo que es recomendable al planificar sobredimensionar tiempos previéndolo, además de calcular un período con el cual poder moverse con holgura si existen inconvenientes extra.



\section{Mejoras y trabajos a futuro}
Entre las cosas que se deben investigar, es que el blindaje incluido en la 
primer antena fabricada cumpla con las regulaciones de compatibilidad 
electromagnética, EMC, definidas por la EN de Europa o por la FCC de Estados 
Unidos. A tales efectos sería necesario contar con el equipamiento adecuado
para efectuar las mediciones necesarias.\\
Tal vez la disminución del tamaño del PCB de la antena, o la incorporación
de ésta al resto del hardware, puedan ser vistas como mejoras desde el 
punto de vista de disminución de costos.

\bigskip
En cuanto al lector de tarjetas de contacto, su incorporación a la lista
de lectores soportados por la biblioteca pcsclite quedó inconclusa; si 
bien el hardware está operativo y es posible enviar comandos APDU al 
módulo SAM de seguridad a través de una pequeña aplicación, sería 
importante alcanzar el objetivo planteado al comenzar el proyecto.
Un aspecto importante a destacar es que no se detallan los comandos
APDU empleados, ya que por motivos de seguridad los mismos no pueden 
ser revelados.

\bigskip
Se podría integrar la comunicación del prototipo RF$^{2}$ con un servidor, 
que gestione todo lo relacionado con las transacciones entre los dispositivos 
y las tarjetas.

\bigskip
 En cuanto a diseño industrial faltaría diseñar y fabricar una 
carcasa acorde, que permita el anclaje rápido y seguro de las distintas
partes de hardware, sin que el material del cual esté hecha interfiera
con la propagación de radio frecuencia, esto descarta la posibilidad de
usar metal como elemento a emplear.

\bigskip
Algo no previsto en el prototipo, es el hecho de que ocurra un corte de energía. Puede
solucionarse incluyendo un sistema con una batería, que se active al detectar la falta de 
energía externa, capaz de alimentar el dispositivo hasta terminar una posible transacción en 
curso y luego ingrese en modo fuera de línea.

\bigskip
Todos los elementos que forman parte del software del sistema se encuentran almacenados
en la memoria SD, sin embargo este tipo de memoria es de menor calidad que la NAND Flash
con la que cuenta la SBC utilizada, y por tanto aumenta la posibilidad de fallas, hecho que 
ocurrió con una de las SD al tener varios bloques de almacenamiento defectuosos. El 
inconveniente anterior, y otros asociados con temas mecánicos como vibraciones (si el 
dispositivo se instala en un vehículo) o suciedad en los contactos, etc., podrían evitarse 
si los archivos son almacenados directamente en memoria Flash.

\bigskip
Un detalle a mejorar en la inicialización del lector/escritor RFID es que verifique
que realmente se trata de este dispositivo y no de otro que se encuentre conectado
al puerto SPI. Esto puede hacerse efectuando una lectura del identificador de producto
desde la memoria EEPROM del integrado CL RC632.

\bigskip
Actualmente existe la versión 3.0 de kernel, se podría probar migrar de la versión que se usó (2.6.32) a la más actual.


