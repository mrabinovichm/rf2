\chapter{Descripci\'on del proyecto}

\section{Definici\'on}

A partir de la puesta en marcha del sistema de transporte metropolitano, surge la necesidad de consultar y recargar tarjetas RFID (utilizadas en dicho sistema) en l\'inea con un servidor, de forma r\'apida, segura y autogestionada por parte del usuario, en diversos puntos de Montevideo.

\section{Antecedentes}

\begin{itemize}

\item AFE: Prototipo de sistema embebido capaz de cargar y consultar tarjetas RFID como las utilizadas en el sistema de transporte metropolitano. El mismo se compone de varios m\'odulos: una SBC (single board computer), un lector-escritor de tarjetas RFID, un lector de tarjetas de contacto, un m\'odem 3G/GPRS y una interfaz con el usuario que consta de un display, leds y buzzer.

\item OpenPCD: Dise'no de hardware libre para dispositivos de proximidad de acoplamiento (PCD) basado en comunicaci\'on RF de 13,56MHz. Este dispositivo es capaz de desplegar informaci\'on desde Tarjetas de proximidad de Circuito Integrado (PICC) que se ajusten a las normas de proveedores independientes, tales como ISO 14443, ISO 15693, as\'i como los protocolos propietarios como Mifare Classic.

\end{itemize}