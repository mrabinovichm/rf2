\chapter{Objetivo general del proyecto}

\section{?`Qu\'e y para qu\'e?}

El objetivo primero del proyecto es la interacción con tarjetas inteligentes sin contacto, 
a través de un dispositivo que permita tanto su lectura como escritura.
Esto implica el diseño y la fabricación de un prototipo de sistema embebido
que está compuesto por dos lectores/escritores de tarjetas inteligentes, uno para tarjetas
sin contacto de tipo Mifare y otro para tarjetas con contacto; cuenta también con una
interfaz para el usuario capaz de informar el estado de la transacción mediante mensajes adecuados.

El diseño hardware del lector/escritor de tarjetas sin contacto se basa en el dispositivo
OpenPCD, a nivel de software es compatible con la misma biblioteca que este último y busca
disminuir el elevado costo de emplear un lector/escritor OpenPCD dentro del prototipo.

En relación al dispositivo AFE mencionado en los antecedentes, el objetivo es mejorar
su arquitectura rompiendo dependencias tecnológicas con su actual lector/escritor de
tarjetas sin contacto y hacer uso de una single board computer que permita una mayor
flexibilidad que la usada por este dispositivo hasta el momento.


\section{?`Por qu\'e cambiar la arquitectura actual?}

Como arquitectura precedente existe la del prototipo AFE (Artefacto Feo de Exhibir), realizada por el grupo de electr\'onica de la Intendencia de Montevideo. La misma consiste en una SBC, que se fabrica con otro prop\'osito y es utilizada en esta aplicaci\'on puesto que es la \'unica forma de adquirir este tipo de hardware en plaza. A \'esta se conectan a trav\'es de puertos USB, un lector/escritor de tarjetas Mifare, un lector de tarjetas de contacto, un modem 3G y un dispositivo dise'nado a partir de un microcontrolador PIC, llamado USB4ALL \cite{usb4all}, el cual es open-hardware y open-firmware, en el que se pueden conectar otro tipo de dispositivos cuya interfaz nativa no sea USB, como ser display, buzzer, leds, sensores, etc, los cuales no pueden ser conectados directamente a la SBC porque la misma no cuenta con los puertos de expansi\'on necesarios.

Surge entonces la necesidad de cambiar la configuraci\'on de dicha arquitectura. Se hace necesario romper dependencias tecnol\'ogicas con el lector/escritor de tarjetas Mifare, ya que dej\'o de ser soportado por la biblioteca pcsclite \cite{pcsclite} (a pedido del fabricante); y con la SBC, que es empleada en una aplicaci\'on espec\'ifica y puede dejar de fabricarse o sufrir cambios dr\'asticos que ya no permitan su uso.


\section{Alcance}

\begin{itemize}

\item Hardware: Se fabricar\'a un m\'odulo donde se insertar\'a la tarjeta de contacto (SAM).
Se agregar\'a un display, leds y buzzer como interfaz para el usuario. Se fabricar\'a un m\'odulo de lectura/escritura RFID. Se estudiar\'a la forma de conectar los perif\'ericos a la placa de la SBC.

\item Software: Se har\'a lo necesario para que el lector/escritor RFID funcione como un dispositivo soportado por la biblioteca librfid \cite{librfid}. Se hará lo posible para lograr compatibilidad hacia atrás con el AFE.
    
\end{itemize}

\section{Especificaci\'on funcional}

El prototipo final deber\'a ser capaz de interactuar con tarjetas RFID a trav\'es de la antena del dispositivo lector/escritor RFID, y con una tarjeta de contacto (SAM). Luego de los controles correspondientes y autenticaci\'on de la tarjeta (con datos encriptados), comenzar\'a la interacci\'on con el usuario mediante un display que ser\'a la interfaz de comunicaci\'on con el mismo. El display informar\'a al usuario de las tareas que se est\'en realizando, mensajes cortos y descriptivos. Los tiempos de recarga y consulta deber\'an ser menores a un minuto.

\section{Criterios de \'exito}

\begin{itemize}

\item Lograr recargar y consultar tarjetas RFID mediante el dispositivo embebido.

\item Los tiempos de recarga y consulta deber\'an ser menores a un minuto.

\end{itemize}