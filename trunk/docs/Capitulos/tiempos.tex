\chapter{Tiempos}

Cuando se planteó el proyecto, los objetivos eran similares pero el enfoque de solución era otro. Se tenía pensado utilizar, en vez de la SBC, un dispositivo (OpenPCD) como corazón del sistema. Luego de estudiarlo fue descartado y se estudiaron una serie de posibles soluciones, detalladas en 3.2, hasta dar con la solución desarrollada. Esto provocó un cambio en la planificación inicial. Para el primer hito ya se pensó en la posibilidad de replanificar sacrificando la comunicación con servidor, pero se dejó en suspenso dependiendo un poco de la respuesta que se tuviera del “cliente” (IM), que hasta el momento había sido de tiempos muy lentos para la planificación planteada y los plazos manejados. Pesaba además el grado de avance que se pudiera lograr a partir del nuevo planteo. El heho de realizar un lector/escritor RFID fue un gran desafío que se decidió tomar, a riesgo de que el proyecto se hiciera muy cuesta arriba en tiempos puesto que agregó varias tareas a la planificación pero como se mencionó quedó intacta en cuanto a alcance.

Para el segundo hito, aún el lector/escritor RFID no funcionaba y los tiempos se agotaban, ya se tenía un atraso de unos 2 meses dados los problemas (mencionados en el capítulo ensayos) con la SBC seleccionada en principio (Hawkboard) pero se propuso esperar por una replanificación, viendo que el hecho de solucionar el problema del lector/escritor RFID se transformó en el camino crítico y podía suceder de un momento a otro. Se definió un tiempo aproximado de un mes como  espera razonable antes de plantear replanificar. Pasado ese tiempo el lector/escritor RFID estaba funcionando, pero los plazos ya se habían hecho muy cortos, por lo que se decidió pedir una prórroga y replanificar para redondear lo mejor posible el trabajo. El pedido de prórroga se vio justificado por los dos meses perdidos con la primer SBC (mencionado en ensayos), y en la replanificación finalmente se sacrificó la comunicación con el servidor, que desde el primer hito se veía como la parte más firme a descartar de la planificación original.

Finalmente los tiempos se agotaron sin que se llegara a completar la comunicación con el lector de tarjetas de contacto.