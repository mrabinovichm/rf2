\chapter{Tiempos}

Por lo general los proyectos no siguen estrictamente la planificación inicial,
éste no fue la excepción a la regla. Varias son la causas que llevaron a no 
cumplirse con el cronograma establecido, algunas de ellas tenidas en cuenta 
durante el análisis de riesgos y otras impensadas desde todo punto de vista.
Entre estas últimas se encuentran las placas Hawkboard defectuosas, este 
imprevisto absolutamente infortuito forzó una replanificación del cronograma,
siendo además el factor preponderante al momento de solicitar prórroga. Lo 
anterior trajo aparejado el uso de una nueva SBC, que a su vez tuvo repercusión
en el diseño hardware que se tenía hasta ese momento, nuevos componentes de 
hardware tuvieron que ser estudiados para compatibilizar los cambios necesarios.

\bigskip
El retraso alcanzado a causa de lo ya explicado rondó los tres meses.

\bigskip
Sumado a lo que se ha descrito antes, hubo un atraso de alrededor de un mes desde el momento que se seleccionó la SBC GESBC-9G20 hasta que se decidió utilizar la SBC Hawkboard, por falta de respuesta del soporte técnico de la primera. El otro camino crítico que tuvo que
enfrentarse estuvo asociado al diseño de la antena RFID, las notas teóricas
suministradas por el mismo fabricante del chip contenían errores en los
cálculos de algunos componentes que conforman el circuito, provocando que 
la fecha indicada para la culminación de las pruebas de hardware de este 
componente se vieran pospuestas por tres meses aproximadamente. 

\bigskip
Entre las causas que se previeron en el análisis de riesgos, se encuentra el 
financiamiento del proyecto por parte de los integrantes del grupo, esto 
debido a que la parte interesada, IM, no encontró el mecanismo de compra de 
componentes e insumos que no pueden adquirirse en el mercado local, sus trabas 
burocráticas impiden que se manejen tiempos razonables para un proyecto de estas 
características.

\bigskip
Aún habiendo solicitado una prórroga de tres meses los plazos se vencieron sin 
haber culminado el total de las tareas previstas como ser la integración del 
lector de tarjetas de contacto a la lista de dispositivos soportados por la 
biblioteca pcsclite. 

\bigskip
En lo que sigue se detalla una tabla con las tareas realizadas.


\begin{longtable}{|p{4cm}|p{1.5cm}|p{1.5cm}|p{1.5cm}|p{1.5cm}|c|}
%\begin{tabular}{|p{3cm}|c|c|c|c|c|}
\hline
 & \multicolumn{ 2}{c|}{\textbf{Planificación Teórica}} & \multicolumn{ 2}{c|}{\textbf{Planificación Real}} & \textbf{Fechas Importantes} \\ \hline
\multicolumn{1}{|c|}{\textbf{Tareas}} & \textbf{inicio} & \textbf{fin} & \textbf{inicio} & \textbf{fin} &  \\ \hline
Definir arquitectura & 13/05/10 & 29/05/10 & 13/05/10 & 21/05/10 &  \\ \hline
Definir SBC & 31/05/10 & 05/06/10 & 31/05/10 & 20/07/10 &  \\ \hline
Primera compra &  &  &  &  & 29/08/10 \\ \hline
Hito 1 &  &  &  & \multicolumn{1}{l|}{} & 15/09/10 \\ \hline
Cambio de SBC &  &  &  & \multicolumn{1}{l|}{} & 12/11/10 \\ \hline
Segunda compra &  &  &  & \multicolumn{1}{l|}{} & 10/12/10 \\ \hline
Hito 2 &  &  &  & \multicolumn{1}{l|}{} & 15/02/11 \\ \hline
Estudio y diseño hardware lector/escritor RFID & 26/06/10 & 28/12/10 & 05/06/10 & 31/03/11 &  \\ \hline
Tercera compra &  &  &  &  & 05/04/11 \\ \hline
Estudio y diseño hardware lector tarjetas de contacto & 19/06/10 & 28/12/10 & 02/07/10 & 15/06/11 &  \\ \hline
Diseño hardware prototipo final &  &  &  & \multicolumn{1}{l|}{} & 18/06/11 \\ \hline
Adaptar código de lector/escritor RFID & 05/04/11 & 12/05/11 & 08/03/11 & 13/06/11 &  \\ \hline
Cuarta compra &  &  &  &  & 05/07/11 \\ \hline
Comunicación con tarjeta de contacto (software) & 25/04/11 & 16/05/11 & 16/05/11 & 16/07/11 &  \\ \hline
Interfaz de usuario (software) & 13/05/11 & 27/05/11 & 13/05/11 & 04/06/11 &  \\ \hline
Integración de módulos (software) & 27/05/11 & 20/06/11 & 13/06/11 & 16/7/2011 (*) &  \\ \hline
Programa base para demo & 01/07/11 & 10/07/11 & 13/06/11 & 30/06/11 &  \\ \hline
Pruebas del sistema & 20/06/11 & 30/06/11 & 20/06/11 & 16/07/11 &  \\ \hline
Documentación final & 11/07/11 & 31/07/11 & 18/07/11 & 15/08/11 &  \\ \hline
%\end{tabular}
\caption{Descripción de tareas más importantes}
\label{}
\end{longtable}

*no terminado.