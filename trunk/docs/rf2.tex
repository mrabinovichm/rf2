%%%%%%%%%%%%%%%%%%%%%%%%%%%%%%%%%%%%%%%%%%%%%%%%%%%%%%%%%%%%%%%%%%%%%%%
%                                                                     %
%                      UNIVERSIDAD de la Repblica                     %
%               Facultad de Ingeniera			             		  %
%                                                                     %
%                              ------------------                     %
%                             | FORMATO DE TESIS |                    %
%                              ------------------                     %
%                                                                     %
%                           (C) Samuel Oporto Daz                     %
%                         (C) Brucele Tejeda Medrano                  %
%                                                                     %
%                                     2006                            %
%                                                                     %
%%%%%%%%%%%%%%%%%%%%%%%%%%%%%%%%%%%%%%%%%%%%%%%%%%%%%%%%%%%%%%%%%%%%%%%


\documentclass[12pt,oneside,final]{rf2}
\hyphenation{Ber-ge-ret} %se indican las sílabas para que lo "recorte" bien cuando termina un renglón
\hyphenation{con-fi-gu-ra-ción}
\hyphenation{li-bre-rí-as}
\hyphenation{dis-po-ni-bles}
\hyphenation{si-guien-te} 
\hyphenation{a-co-pla-mien-to} 
\hyphenation{Cla-ssic} 
\hyphenation{com-pa-ti-bi-li-dad}
\hyphenation{Primary-Status}
\hyphenation{me-dian-te}
\hyphenation{e-fec-tua-ron}
\hyphenation{ins-tan-cia}
\hyphenation{ca-rac-te-res}
\hyphenation{re-gis-tros}
\hyphenation{su-ge-ri-dos}
\hyphenation{si-guien-tes}
\hyphenation{ethernet}
\hyphenation{bi-blio-te-cas}
\hyphenation{bi-blio-te-ca}
\hyphenation{mo-di-fi-car}
\hyphenation{rea-li-zar}
\hyphenation{re-gis-tro}
\hyphenation{fa-bri-ca-do}
\hyphenation{re-so-nan-cia}
\hyphenation{mo-di-fi-có}  %por alguna razón lo corta así modif-icó

 \usepackage{hyperref}
 \hypersetup{colorlinks=true,    % false: boxed links; true: colored links
    linkcolor=black,          	 % color of internal links
    citecolor=blue,        		 % color of links to bibliography
    filecolor=black,      		 % color of file links
    urlcolor=blue          		 % color of external links
}

\usepackage{array}
\usepackage{longtable} 
\usepackage{multicol} 
\usepackage{epsfig,bm,epsf,float}

%\usepackage[usenames,dvipsnames]{color}

\usepackage[utf8]{inputenc} %si esta no funciona descomentar la de abajo
%\usepackage[latin1]{inputenc} %si esta no funciona descomentar la de arriba
\usepackage[activeacute,spanish]{babel}

\usepackage{graphicx}

\usepackage[usenames]{color}
\usepackage{times}
\usepackage{subfigure}
\usepackage{longtable}
\setlength{\arrayrulewidth}{1pt}
\setlength{\doublerulesep}{0mm}
\usepackage{multirow}
\usepackage{pdfpages} %para insertar archivos pdf
\usepackage{listings} %para insertar código

%\usepackege{tabulary}

\begin{document}

% Portada:

% Nombre de tesis
\title{RF$^{2}$}

% Autor
\author{Daniel Aicardi, Melina Rabinovich, Edgardo Vaz}

% Mes
\degreemonth{Agosto}

% Año
\degreeyear{2011}

% Título profesional
\degree{}

% Asesor
\advisor{Ing. Juan Pablo Oliver, Ing. Andr\'es Aguirre}

\maketitle











\begin{abstract}
\begin{itshape}
\dsp

El presente documento describe el prototipo Recarga Fácil por Radio Frecuencia, RF$^{2}$, realizado como proyecto de fin de carrera de Ingeniería Eléctrica en la Universidad de la República entre marzo de 2010 y julio de 2011. El mismo consiste en un sistema embebido para recarga y consulta de tarjetas RFID, como las que se utilizan hoy día en el sistema de transporte metropolitano, y fue diseñado para operar de forma autónoma interactuando directamente con el usuario.

\bigskip
El hardware fue enteramente diseñado por el grupo de trabajo a excepción de la single board computer. Las herramientas de software utilizadas son open source, así como también las bibliotecas usadas para desarrollar la aplicación final. El diseño, la fabricación, y el armado del prototipo fue realizado en su totalidad en Uruguay.

\end{itshape}
\end{abstract}

\begin{acknowledgments}

En primer lugar queremos agradecer a nuestras familias y amigos. Agradecemos al grupo mina del INCO, a Leonardo Steinfeld, Nicolás Barabino, Francisco Lanzari, María Eugenia Corti, Santiago Reyes, Viterbo Rodríguez, Christian Gutierrez, Andrés Bergeret, Gonzalo Tabares, Klaus Rotzinger, Marcelo Fiori, Pablo Cancela, Ana y Claudia Rabino. Y a todos los que de alguna forma u otra colaboraron con nosotros.

\end{acknowledgments}
% -----------------------------------------------------------------------------------------------------------------
% Cambiar dedicatoria...
% -----------------------------------------------------------------------------------------------------------------


\dedication

\begin{quote}

para cuchu,\\
chuchuchu,\\
y pirimpirim jajaja

\end{quote}
\begin{prefacio}

\begin{itshape}
“El ciudadano Línea saca su billetera, extrae su tarjeta y la introduce en la máquina registradora; una serie de gestos automáticos. Unas mandíbulas de aluminio se cierran sobre ella, unos dientes de cobre buscan la clave magnética, y una lengua electrónica saborea la vida del ciudadano Línea.
Lugar y fecha de nacimiento. Padres. Raza. Religión. Historial educativo, militar y de servicios civiles. Estado. Hijos. Ocupaciones, desde el comienzo hasta el presente. Asociaciones. Medidas físicas, huellas digitales, retínales, grupo sanguíneo. Grupo psíquico básico. Porcentaje de lealtad, índice de lealtad en función del tiempo hasta el momento del último análisis...
... El ciudadano Línea se encuentra en la ciudad donde, la noche anterior, dijo que estaría, así que no ha tenido que hacer una corrección.
Los nuevos informes se añaden al historial del ciudadano Línea. Toda su vida regresa al banco de datos. Desaparece de la unidad exploradora y la unidad comparativa, para que éstas atiendan la próxima llegada.
La máquina ha tragado y digerido otro día. Está satisfecha.”

\rightline{Sam Hall (1953), Poul Anderson}
\end{itshape}
\bigskip

La narración anterior es parte de un cuento de ciencia ficción llamado “Sam Hall”, escrita por Poul Anderson en 1953. En esta historia el autor describe un mundo donde cada persona tiene asignada una tarjeta conteniendo datos que la caracterizan, y puede ser controlado su accionar a través de una super computadora que almacena y procesa los datos de toda la humanidad. 
En nuestros días este cuento de ciencia ficción no está tan alejado de la realidad, las tarjetas “inteligentes” (smart cards) son cada vez más usadas en múltiples aplicaciones como ser, pasaporte electrónico, pago electrónico, sistemas de transporte, controles de acceso y sistemas de seguridad, entre otros.
El siguiente proyecto se desarrolla con la intención de aprender las bases del mundo de las tarjetas “inteligentes” y que sirva como punto de partida para que otros entiendan su funcionamiento. 
No es intención de los autores que se use el contenido de este documento con fines como los que se indicaban en la narrativa de ciencia ficción, muy por el contrario, el empleo de esta tecnología debe estar en favor de las personas y no en su contra.

\end{prefacio}
\newpage
\addcontentsline{toc}{section}{Tabla de contenidos}

\tableofcontents

\listoffigures

\listoftables
\newpage
%\startarabicpagination

\addcontentsline{toc}{section}{Glosario}
\begin{glosario}

{\bf{AFE}} – Artefacto Feo de Exhibir, dispositivo precedente realizado en IM por el grupo de electrónica.

{\bf{APDU}} – es la unidad de comunicación entre un lector de tarjetas inteligentes y una tarjeta inteligente, según la sigla Application Protocol Data Unit.

{\bf{ASCII}} – Código Estadounidense Estándar para el Intercambio de Información, según su sigla American Standard Code for Information Interchange.

{\bf{ASIC}} – circuito integrado de aplicación específica, según su sigla application-specific integrated circuit.

{\bf{ASK}} – codificación por desplazamiento de amplitud, según su sigla amplitude shift key.

ATR – respuesta al reset, según su sigla answer to reset.

Authent – comando, usado en el integrado CL RC632, para autenticación de tarjetas RFID, que accede a fifo.

baud rate – número de unidades de señal por segundo, tasa de baudios.

bootloader – aplicación usada para iniciar el sistema operativo.

CCID – controlador genérico de dispositivos lectores de smart card con interfaz USB, según su sigla Chip/Smart Card Interface Devices.

CL RC632 – integrado para lector de tarjetas RFID de protocolo múltiple.

compilar – traducción de un código fuente (escrito en un lenguaje de programación de alto nivel) a lenguaje máquina.

croscompilar – compilar software para una plataforma diferente a la usada, usando un entorno especial (compilador, librerias, etc.).

ext3 – tercer sistema de archivos extendido , es un sistema de archivos con registro por diario, el más usado en Linux.

FAT32 – es un tipo de sistema de archivos FAT (File Allocation Table), desarrollado para MS-DOS.

FIFOData – registro, del integrado CL RC632, en el que se escribe/lee para almacenar/quitar un byte del buffer FIFO, e incrementa/decrementa el puntero de escritura/lectura de dicho buffer.

FIFOLength – registro, del integrado CL RC632, que almacena la distancia entre los punteros de lectura y escritura del buffer.

fileSystem – estructura la información guardada en una unidad de almacenamiento.

filtro EMC – filtro para reducir ruidos o interferencias.

GDB – depurador estándar para el sistema operativo GNU, según su sigla GNU Debugger.

GPIO – entrada/salida de propósito general, según su sigla general purpose input/output.

HiAlertIRq – bandera (bit1) del registro InterruptRq, del integrado CL RC632, que seteada a 1 indica que HiAlert del registro PrimaryStatus fue seteada, indicando overflow o underflow en el buffer fifo de acuerdo a: $64 - FIFOLength \leq WaterLevel $ donde WaterLevel es el nivel de alerta para overflow y underflow y se fija en el registro FifoLevel.

I/O – pines de entrada/salida, según su sigla input/output.

I2C – bus de comunicaciones en serie, que utiliza dos líneas para transmitir la información: una para los datos y por otra la señal de reloj.

IDE – entorno de desarrollo integrado, según su sigla integrated development environment.

IdleIRq – bandera (bit2) del registro InterruptRq, del integrado CL RC632, que seteada a 1 indica que terminó un comando por si mismo. Si comienza un comando desconocido también se setea a 1.

IIE – Instituto de Ingeniería Eléctrica.

InterruptEn – registro de bits de control, del integrado CL RC632, para habilitar/deshabilitar peticiones de interrupción.

InterruptRq – registro de banderas, del integrado CL RC632, de petición de interrupción.

IRQPinConfig – registro, del integrado CL RC632, que configura el estado de salida del pin IRQ.

ISO7816 – estándar internacional relacionado con tarjetas de identificación electrónicas, en particular con tarjetas inteligentes.

ISO 14443 – estándar internacional relacionado con tarjetas de identificación electrónicas, en particular con tarjetas inteligentes. Este estándar define una tarjeta RFID utilizada para identificación y pagos.

ISR – rutina de atención a interrupción, según su sigla interrupt service routine.

JTAG – interfaz especial de cuatro o cinco pines agregadas a un chip, diseñada de tal manera que varios chips en una tarjeta puedan tener sus líneas JTAG conectadas en daisy chain, de manera tal que una sonda de testeo JTAG necesita conectarse a un solo "puerto JTAG" para acceder a todos los chips en un circuito impreso.

kernel – núcleo de linux, es el corazón de este sistema operativo.

kernel panic – error interno en el sistema.

LCD16x2 – display de dos líneas de 16 caracteres cada una.

lector mudo – lector que dialoga directamente con la tarjeta, sin que haya un intermediario.

LoadKey – comando, usado en el integrado CL RC632, que lee una llave del buffer FIFO y, si está en el formato correcto, la almacena en el buffer de llaves.

LoAlertIRq – bandera (bit0) del resgistro InterruptRq, del integrado CL RC632, que seteada a 1 indica que el bit LoAlert del registro PrimaryStatus fue seteado, indicando que la cantidad de bytes en el buffer FIFO cumplen $FIFOLength \leq WaterLevel $, donde WaterLevel es el nivel de alerta para overflow y underflow y se fija en el registro FifoLevel.

PIC – familia de microcontroladores tipo RISC (reduced instruction set computer) fabricados por Microchip Technology Inc.

Mifare – tecnología de tarjetas inteligentes sin contacto (TISC) más ampliamente instalada en el mundo, es propiedad de NXP Semiconductores.

Mifare Classic – dispositivos de almacenamiento de memoria, generalmente de 1K y 4K.

open-firmware – 

open-hardware – dispositivos de hardware cuyas especificaciones y diagramas esquemáticos son de acceso público.

open source – desarrollado y distribuído en forma libre.

OpenPCD – dispositivo lector de tarjetas RFID.

PC – computador personal, según su sigla personal computer.

PCB – placa de circuito impresa, según su sigla printed circuited board.

pcsclite – biblioteca para smart cards.

PrimaryStatus – registro de banderas de estado del receptor del integrado CL RC632, transmisor y buffer FIFO.

protocolo T=0 – protocolo de datos orientado a byte.

protocolo T=1 – protocolo de datos orientado a bloques.

LDO – salida de baja caída, según su sigla Low Drop Output

RF – radio frecuencia.

RF$^{2}$ – Recarga Fácil por Radio Frecuencia.

RS232 – interfaz que designa una norma para el intercambio serie de datos binarios entre un DTE (Equipo terminal de datos) y un DCE (Data Communication Equipment, Equipo de Comunicación de datos).

RSTPD – pin de reseteo y apagado del integrado CL RC632, según su sigla reset and power down.

RxIRq – bandera (bit3) del registro InterruptRq, del integrado CL RC632, que es seteada a 1 cuando el receptor termina de recibir.

SBC – computadora completa hecha en un único PCB, según su sigla single board computer.

SC – tarjeta inteligente, según su sigla smart card.

SCUI – PCB lector/escritor de tarjetas de contacto e interfaz de usuario, según su sigla smart card user interface.

SDK – kit de desarrollo de software, según su sigla software development kit.

SIMO – entrada esclavo salida maestro, según su sigla slave in master out.

sistema embebido – Dispositivos electrónicos que incorporan una computadora (normalmente un microprocesador usado principalmente para simplificar el diseño del sistema y darle flexibilidad ) en su implementación. 

software libre – software que una vez obtenido puede ser usado, copiado, estudiado, modificado y redistribuido libremente.

SOMI – salida esclavo entrada maestro, según su sigla slave out master in.

SPI – bus de interfaz de periféricos serie.

tarjeta de contacto SAM – tarjeta que interactúa por contactos.

tarjeta RFID – tarjeta capáz de interactuar sin contacto.

Transceive – comando, usado en el integrado CL RC632, que transmite datos desde el buffer FIFO a la tarjeta RFID y al terminar activa automáticamente el receptor.

TimerIRq – bandera (bit5) del registro InterruptRq, del integrado CL RC632, que es seteada a 1 cuando el registro TimerValue decrementa a cero.

TxIRq – bandera (bit4) del registro InterruptRq, del integrado CL RC632, que es seteada a 1 cuando: el comando Transceive, Auth1 o Auth2 termina, el comando WriteE2 programa todos los datos, el comando CalcCRC procesa todos los datos.

UART – Transmisor-Receptor Asíncrono Universal, controla los puertos y dispositivos serie, según su sigla Universal Asynchronous Receiver-Transmitter.

UID – código único de identificación de tarjeta RFID.

USB – bus universal en serie, puerto que sirve para conectar periféricos, según su sigla Universal Serial Bus.

USB OTG – especificación que permite a dispositivos USB actuar como host permitiendo adjuntar dispositivos USB (mouse, teclado, etc.).

USB4ALL – interfaz USB genérica para comunicación con dispositivos electrónicos.

VLT – PCB conversor de voltajes, según su sigla voltage level translator.

\end{glosario}

\startarabicpagination
% Capítulos ---> esto define el índice
\part{Introducción}
\chapter{Descripci\'on del proyecto}

\section{Definici\'on}

A partir de la puesta en marcha del STM, surge la necesidad de consultar y recargar tarjetas RFID (utilizadas en dicho sistema) en l\'inea (con conexi\'on a un servidor de la Intendencia de Montevideo), de forma r\'apida, segura y auto-gestionada por parte del usuario, en diversos puntos de Montevideo.

\section{Antecedentes}

\begin{itemize}

\item AFE: Prototipo de sistema embebido capaz de cargar y consultar tarjetas RFID como las utilizadas en el STM. El mismo se compone de varios m\'odulos: una SBC (single board computer), un lector-escritor de tarjetas RFID, un lector de tarjetas de contacto, un m\'odem 3G/GPRS y una interfaz con el usuario que consta de un display,leds y buzzer.

\item OpenPCD: Dise'no de hardware libre para dispositivos de proximidad de acoplamiento (PCD) basado en comunicaci\'on RF de 13,56MHz. Este dispositivo es capaz de desplegar informaci\'on desde Tarjetas de proximidad de Circuito Integrado (PICC) que se ajusten a las normas de proveedores independientes, tales como ISO 14443, ISO 15693, as\'i como los protocolos propietarios como Mifare Classic.

\end{itemize}
\chapter{Objetivo general del proyecto}

\section{?`Qu\'e y para qu\'e?}

El objetivo primero del proyecto es la interacción con tarjetas inteligentes sin contacto, 
a través de un dispositivo que permita tanto su lectura como escritura.
Esto implica el diseño y la fabricación de un prototipo de sistema embebido
que está compuesto por dos lectores/escritores de tarjetas inteligentes, uno para tarjetas
sin contacto de tipo Mifare y otro para tarjetas con contacto; cuenta también con una
interfaz para el usuario capaz de informar el estado de la transacción mediante mensajes adecuados.

El diseño hardware del lector/escritor de tarjetas sin contacto se basa en el dispositivo
OpenPCD, a nivel de software es compatible con la misma biblioteca que este último y busca
disminuir el elevado costo de emplear un lector/escritor OpenPCD dentro del prototipo.

En relación al dispositivo AFE mencionado en los antecedentes, el objetivo es mejorar
su arquitectura rompiendo dependencias tecnológicas con su actual lector/escritor de
tarjetas sin contacto y hacer uso de una single board computer que permita una mayor
flexibilidad que la usada por este dispositivo hasta el momento.


\section{?`Por qu\'e cambiar la arquitectura actual?}

Como arquitectura precedente existe la del prototipo AFE (Artefacto Feo de Exhibir), realizada por el grupo de electr\'onica de la Intendencia de Montevideo. La misma consiste en una SBC, que se fabrica con otro prop\'osito y es utilizada en esta aplicaci\'on puesto que es la \'unica forma de adquirir este tipo de hardware en plaza. A \'esta se conectan a trav\'es de puertos USB, un lector/escritor de tarjetas Mifare, un lector de tarjetas de contacto, un modem 3G y un dispositivo dise'nado a partir de un microcontrolador PIC, llamado USB4ALL \cite{usb4all}, el cual es open-hardware y open-firmware, en el que se pueden conectar otro tipo de dispositivos cuya interfaz nativa no sea USB, como ser display, buzzer, leds, sensores, etc, los cuales no pueden ser conectados directamente a la SBC porque la misma no cuenta con los puertos de expansi\'on necesarios.

Surge entonces la necesidad de cambiar la configuraci\'on de dicha arquitectura. Se hace necesario romper dependencias tecnol\'ogicas con el lector/escritor de tarjetas Mifare, ya que dej\'o de ser soportado por la biblioteca pcsclite \cite{pcsclite} (a pedido del fabricante); y con la SBC, que es empleada en una aplicaci\'on espec\'ifica y puede dejar de fabricarse o sufrir cambios dr\'asticos que ya no permitan su uso.


\section{Alcance}

\begin{itemize}

\item Hardware: Se fabricar\'a un m\'odulo donde se insertar\'a la tarjeta de contacto (SAM).
Se agregar\'a un display, leds y buzzer como interfaz para el usuario. Se fabricar\'a un m\'odulo de lectura/escritura RFID. Se estudiar\'a la forma de conectar los perif\'ericos a la placa de la SBC.

\item Software: Se har\'a lo necesario para que el lector/escritor RFID funcione como un dispositivo soportado por la biblioteca librfid \cite{librfid}. Se hará lo posible para lograr compatibilidad hacia atrás con el AFE.
    
\end{itemize}

\section{Especificaci\'on funcional}

El prototipo final deber\'a ser capaz de interactuar con tarjetas RFID a trav\'es de la antena del dispositivo lector/escritor RFID, y con una tarjeta de contacto (SAM). Luego de los controles correspondientes y autenticaci\'on de la tarjeta (con datos encriptados), comenzar\'a la interacci\'on con el usuario mediante un display que ser\'a la interfaz de comunicaci\'on con el mismo. El display informar\'a al usuario de las tareas que se est\'en realizando, mensajes cortos y descriptivos. Los tiempos de recarga y consulta deber\'an ser menores a un minuto.

\section{Criterios de \'exito}

\begin{itemize}

\item Lograr recargar y consultar tarjetas RFID mediante el dispositivo embebido.

\item Los tiempos de recarga y consulta deber\'an ser menores a un minuto.

\end{itemize}
\part{Diseño}
\chapter{Funcionamiento del prototipo}

\section{Requerimientos}
El principal requerimiento a cumplir es la interacción con tarjetas RFID (ver apéndice \ref{anx_sc}), tanto para su lectura como escritura.
La comunicación con tarjetas de contacto (ver apéndice \ref{anx_sc}) es necesaria para la interacción con un módulo de seguridad que permita, la generación de las claves utilizadas para autenticarse con las tarjetas RFID, y una transacción segura con un servidor. En ambos casos es necesario cumplir con las normas y estándares adecuados  (tarjetas RFID - ISO 14443 y tarjetas de contacto - ISO7816).
Por último mantener informado al usuario de lo que sucede durante una transacción a través de una interfaz visual y sonora.


\section{Descripción del prototipo}\label{2.2}
Este prototipo integra la lista de dispositivos que hoy en día se denominan sistemas embebidos. Su hardware está integrado por un sistema basado en un microprocesador que  recibe el nombre de Single Board Computer (SBC), a la que se agrega un conversor de niveles (VLT) que permite interconectarla con, un lector/escritor de tarjetas RFID a través de un puerto SPI, un lector de tarjetas de contacto a través de un puerto serial (UART), y la interfaz de usuario compuesta por un buzzer, tres leds (rojo, amarillo, verde) y un display conectado a través de puertos de entrada/salida de propósito general (GPIO).

\begin{figure}[H]
\centering
  \begin{center}
   \includegraphics[scale=.15]{Imagenes/prototipo_s_nombres.jpg}
  \end{center}
  \caption{Vista general del prototipo}\label{prototipo} 
\end{figure}

\begin{figure}[H]
  \centering
  \subfigure[Vista anversa del prototipo]{\label{protoF}
  \includegraphics[scale=.13]{Imagenes/prototipo_f.jpg} } 
  \subfigure[Vista reversa del prototipo]{\label{protoB} 
  \includegraphics[scale=.13]{Imagenes/prototipo_b.jpg} }

  \caption{Vista anversa y reversa del prototipo}\label{protoFB}
\end{figure}

En cuanto al software, está basado en bibliotecas de código abierto que permiten desarrollar la aplicación que asegura el manejo del hardware y el funcionamiento de todo el sistema en conjunto.

\bigskip
Debe indicarse que las funcionalidades que brinda el módulo de seguridad no son usadas en la aplicación final. Si bien el hardware que compone el lector/escritor de tarjetas de contacto está operativo, ya que fue probado con una pequeña aplicación que permite enviar comandos APDU a la tarjeta de seguridad, no fue posible integrar este lector/escritor a la lista de dispositivos compatibles con la biblioteca PCSC-Lite.

En cuanto a la comunicación con el servidor indicado en el bloque de la figura \ref{HW_gral}, no está implementada por quedar fuera del alcance del proyecto.

\section{Funcionamiento general del prototipo}
Una vez que el prototipo RF$^{2}$ se encuentra operativo, el dispositivo despliega en el display el mensaje “Aproxime su tarjeta”, permaneciendo en dicho estado hasta que algún usuario acerque una tarjeta al lector/escritor RFID. 
En la primera transacción entre lector y tarjeta se obtiene el identificador único (UID) de ésta última, que será enviado al módulo de seguridad SAM (previa autenticación exitosa), para que a partir de éste, se generen las claves de acceso que permitan la lectura y escritura de la tarjeta RFID.
Mientras se lleva a cabo la operación, se despliega en el display el mensaje, “No retire su tarjeta” a la vez que el led amarillo es encendido para indicar precaución ya que se están procesando datos.
La siguiente acción a llevar a cabo es verificar que la tarjeta del usuario tenga saldo pendiente de acreditar, en caso afirmativo se indica al usuario el saldo a acreditar a través del display con el mensaje “Saldo a acreditar \$...”. Si todo fue exitoso, se borra el saldo transferido de la lista de saldos pendientes a acreditar para que no se transfiera saldo indefinidas veces.
A continuación se despliega en el display el nuevo monto almacenado en la tarjeta, “Su saldo es de \$...”, se enciende el led verde y se emite un pitido mediante el buzzer en señal que la operación fue satisfactoria.
Por último se muestran en el display los mensajes “Transacción finalizada”, “Gracias” y vuelve al inicio para comenzar un nuevo ciclo.

En caso que la tarjeta no tuviera saldo pendiente de acreditar, el prototipo RF$^{2}$ funciona en modo consulta y despliega en el display el saldo disponible en la tarjeta, “Su saldo es de \$...”, encendiendo el led verde y emitiendo un pitido, seguido de los mensajes “Transacción finalizada”, “Gracias” y vuelve al inicio para comenzar un nuevo ciclo.

En caso de ocurrir un error durante alguno de los pasos anteriores, ya sea porque
el usuario retiró la tarjeta en un momento inadecuado, o simplemente porque el prototipo RF$^{2}$
no logró leer o escribir la tarjeta en forma correcta, se enciende el led rojo, se emite un
doble pitido mediante el buzzer, y el display muestra el mensaje “Error, vuelva a intentarlo”,
acto seguido el ciclo vuelve a comenzar. 

\begin{figure}[H]
\centering
  \begin{center}
   \includegraphics[scale=.35]{Imagenes/flujo.jpg}
  \end{center}
  \caption{Diagrama de flujo}\label{Fig:HW} 
\end{figure}
\chapter{Hardware}

%sección 4.1
\section{Arquitecturas estudiadas}\label{arqEst}
Se plantearon varias alternativas como posible solución. A medida que se encontraron limitantes o que no se cumplían los requerimientos exigidos, se fueron descartando dichas opciones.

A continuación se describen algunas de las arquitecturas consideradas:

\begin{itemize}
\item[1 -] OpenPCD + lector de tarjetas de contacto + display + buzzer + leds
\bigskip

\begin{figure}[H]
\centering
  \begin{center}
  \includegraphics[scale=.4]{Imagenes/0.jpg} 
  \end{center}
  \caption{Solución considerada 1}\label{Fig:HW1} 
\end{figure}

\newpage
Al dispositivo OpenPCD, se conecta el resto del hardware a través de su único puerto de entrada/salida disponible que es de tipo I2C.

\bigskip
\item[2 -] SBC + OpenPCD + microcontrolador + lector de tarjetas de contacto + display + buzzer + leds
\bigskip

\begin{figure}[H]
\centering
  \begin{center}
  \includegraphics[scale=.3]{Imagenes/1.jpg} 
  \end{center}
  \caption{Solución considerada 2}\label{Fig:HW2} 
\end{figure}

Tanto el dispositivo OpenPCD como el microcontrolador se conectan directamente por USB a la SBC. El microcontrolador maneja el resto de los dispositivos (lector de tarjetas de contacto, display, buzzer y leds).

\bigskip
\item[3 -] SBC + OpenPCD + lector de tarjetas de contacto + display + buzzer + leds
\bigskip

\begin{figure}[H]
\centering
  \begin{center}
  \includegraphics[scale=.25]{Imagenes/2.jpg} 
  \end{center}
  \caption{Solución considerada 3}\label{Fig:HW3} 
\end{figure}

El dispositivo OpenPCD se conecta por USB a la SBC. La SBC maneja los dispositivos (lector de tarjetas de contacto, display, buzzer y leds) a través de sus interfaces nativas.

\bigskip
\item[4 -] SBC + lector de tarjetas RFID + lector de tarjetas de contacto + display + buzzer + leds
\bigskip

\begin{figure}[H]
\centering
  \begin{center}
  \includegraphics[scale=.25]{Imagenes/3.jpg} 
  \end{center}
  \caption{Solución considerada 4}\label{Fig:HW4} 
\end{figure}

Todos los periféricos se conectan a la SBC a través de sus interfaces nativas, esto incluye también el integrado CL RC632 de Philips \cite{RC632} (ver hoja de datos en el apéndice \ref{HD}). Se debe diseñar la antena para propagar la señal RF hacia las tarjetas.

\bigskip
\item[5 -] microcontrolador + lector de tarjetas RFID + lector de tarjetas de contacto + display + buzzer + leds
\bigskip

\begin{figure}[H]
\centering
  \begin{center}
  \includegraphics[scale=.25]{Imagenes/4.jpg} 
  \end{center}
  \caption{Solución considerada 5}\label{Fig:HW5} 
\end{figure}

Consta de un único PCB, que posee un microcontrolador como sistema central al cual se
conectan el resto de los dispositivos. Dicho PCB tiene incorporada la antena para la
propagación de RF.

\end{itemize}

\section{Arquitectura seleccionada}
En una primera instancia se pretendía utilizar únicamente el dispositivo OpenPCD (ver figura \ref{Fig:HW1}), ya que el mismo cuenta con un microcontrolador de la familia ARM, el AT91SAM7S128. Una vez estudiado se llegó a la conclusión de que no permitía la instalación de un sistema operativo GNU/Linux, ya que el mismo precisa más de 4 MB de RAM para poder hacer algo útil. Otra desventaja encontrada fue que sólo tiene un puerto I2C como forma de conectar periféricos.

Surgió entonces la necesidad de usar una SBC como dispositivo capaz de ejecutar un sistema operativo y las aplicaciones necesarias para que el dispositivo cumpla con los requerimientos exigidos. El dispositivo OpenPCD pasaría entonces a cumplir la función de lector/escritor de tarjetas RFID (ver figuras \ref{Fig:HW2} y \ref{Fig:HW3}), conectado a la SBC a través de su puerto USB, mientras que para el resto de los periféricos se diseñaría un PCB que fuera capaz de ser conectado a la SBC a través de sus interfaces nativas. Esta arquitectura fue descartada por el incremento en el costo del proyecto.

Fue necesario entonces descartar el uso del dispositivo OpenPCD y dar lugar a un diseño propio del lector/escritor de tarjetas RFID (ver figura \ref{Fig:HW4}), utilizando para esto el integrado CL RC632 de Philips.

La última opción y la más ambiciosa, plantea el diseño completo de un PCB (ver figura \ref{Fig:HW5}) conteniendo un microcontrolador y memoria capaz de ejecutar un sistema operativo, los lectores de tarjetas, tanto de contacto como RFID, y el resto de los periféricos (display, leds, buzzer). Esta opción fue dejada de lado por entender que excedería los plazos de tiempo del proyecto.

Se pensó entonces en diseñar la arquitectura 4 indicada en la figura \ref{Fig:HW4}, SBC + lector de tarjetas RFID + lector de tarjeta de contacto + display + buzzer + leds, y dado que se cuenta con un OpenPCD, la opción 3 mostrada en la figura \ref{Fig:HW3}, SBC + OpenPCD + lector de tarjetas de contacto + display + buzzer + leds, se dejaría como arquitectura alternativa si no se alcanzaran buenos resultados con el lector/escritor de tarjetas RFID.

\bigskip
Luego de estudiar ventajas y desventajas de las arquitecturas planteadas, se eligió la indicada en el ítem 4 en la sección \ref{arqEst}, que es la que más se adaptó a los requerimientos necesarios:

\begin{itemize}
\item SBC +  lector de tarjetas RFID + lector de tarjeta de contacto + display + buzzer + leds
\end{itemize}

En la figura \ref{Fig:HW_GRAL} se muestra un diagrama de bloques correspondiente a la arquitectura seleccionada:

\begin{figure}[H]
\centering
  \begin{center}
  \includegraphics[scale=.5]{Imagenes/diagrama_rf2.jpg} 
  \end{center}
  \caption{Diagrama de bloques de la arquitectura seleccionada}\label{Fig:HW_GRAL} 
\end{figure}


Los bloques oscuros serán implementados, no así los claros.


\newpage
\section{Elecci\'on de hardware}

\subsection{SBC}
En primera instancia se confeccionó una lista con posibles candidatas de SBC disponibles
en el mercado internacional, teniendo en cuenta factores como: precio, puertos de I/O, memoria RAM, memoria Flash, puertos USB, soporte para GNU/Linux, entre otros.
Se definieron una serie de requisitos mínimos necesarios para seleccionar de la lista la SBC que más se adecuara a la arquitectura definida.
Para la comunicación con el resto de los módulos será necesario: una interfaz UART para el módulo de seguridad (SAM); una interfaz SPI para el módulo lector/escritor RFID (CL RC632 de Philips); 20 GPIO para display, leds, buzzer, otros; 1 USB host para una posible conexión de un modem 3G (intercambio de datos con un servidor). En cuanto a la memoria disponible, tomando como referencia el AFE, debe ser de 32 MB de RAM y 8 MB de flash para un funcionamiento aceptable. Es conveniente, pensando a futuro, que el procesador trabaje a una frecuencia no menor a 200 MHz.
Dado el presupuesto estimado para el proyecto, el precio no debe superar los 150 dólares en origen.
Como requisito adicional se exigió que existiera un foro actualizado y soporte técnico que permitiera evacuar dudas.


Aplicados los requisitos mínimos a la lista previamente confeccionada de SBC candidatas, se optó por dos: GESBC-9G20 \cite{9G20} y Hawkboard \cite{Hawk}.
En cuanto a la primera opción, GESBC-9G20, los fabricantes no respondieron consultas, por tanto se descartó. Se optó entonces por la segunda opción, Hawkboard, puesto que respondieron a las consultas en tiempos razonables y se logró evacuar dudas desde el foro.


Luego de comprar dos Hawkboard, ambas resultaron defectuosas a nivel de hardware, después de varios meses de pruebas sin resultados y sin respuestas concretas por parte del proveedor y fabricante y con la intención de cumplir con los plazos del proyecto, se optó por utilizar una SBC (Beagleboard \cite{Beagle}) que se consiguió en préstamo por medio del INCO. Esta SBC cumplió con los requisitos mínimos, aunque en ese momento tenía un costo del doble de la Hawkboard, teniéndose que diseñar un módulo hardware adicional.
Finalmente, la SBC seleccionada para trabajar fue la Beagleboard rev.C4.

Las características generales de la BeagleBoard son: cuenta con un procesador \\
OMAP3530 de 720 MHz con arquitectura ARM. Posee  memoria NAND-flash de 256 MB y memoria ROM de igual tamaño. Tiene una ranura adicional para extender la memoria a través de una memoria SD. Entre otras cosas cuenta con un puerto USB OTG, un puerto USB host, un bloque de expansión de 28 pines (con señales a 1,8 Volt), puerto JTAG, conector RS232, etc.\\
En lo que respecta a la potencia disipada, la Beagleboard tiene un consumo de pico de 2W, y un consumo promedio de 560mW \cite{consumo1} \cite{consumo2}.

\begin{figure}[H]
  \centering
  \subfigure[Vista anversa de la SBC]{\label{sbcF}
  \includegraphics[scale=.08]{Imagenes/SBC_f.jpg} } 
  \subfigure[Vista reversa de la SBC]{\label{sbcB} 
  \includegraphics[scale=.08 ]{Imagenes/SBC_b.jpg} }

  \caption{Vista anversa y reversa de la SBC}\label{sbcFB}
\end{figure}


\subsection{VLT - Conversor de Voltajes}
Este módulo no fue tenido en cuenta en la primera etapa del diseño de la arquitectura hardware, sino que surge como necesidad debido al cambio de SBC. Como consecuencia de lo anterior se vio la ventaja de incorporar una placa que permite la conexión entre la SBC y el resto del hardware, el cual puede permanecer inalterado por más que no ocurra lo mismo con la SBC, ya que ésta puede cambiar de versión o dejar de fabricarse en un breve lapso de tiempo. El único elemento a cambiar sería entonces la placa VLT, que es más simple y barata de fabricar que las restantes partes.
La placa de circuito impreso VLT consta básicamente de dos conectores, uno de ellos permite la conexión con la Beagleboard y el otro la conexión con el restante hardware el cual se encuentra intergrado en un PCB llamdo SCUI. Ambos conectores no se encuentran directamente interconectados entre sí a través de pistas, pues para el caso particular de Beagleboard fue necesario incorporar conversores de tensión que permitieran el traslado del nivel de tensión desde 1,8 Volt que usa esta SBC, a las tensiones con las que operan los periféricos, ya sea 3,3 o 5 Volt.
El último elemento, no menos importante, es un regulador de tensión LDO que permite generar 3,3 Volt a partir de la fuente de tensión de 5 Volt de la propia Beagleboard.

Por más detalles ver esquemático en la figura \ref{Fig:VLT}.

\begin{figure}[H]
  \centering
  \subfigure[Vista anversa de la SBC]{\label{vltF}
  \includegraphics[scale=.08, angle=90]{Imagenes/VLT_f.jpg} } 
  \subfigure[Vista reversa de la SBC]{\label{vltB} 
  \includegraphics[scale=.08, angle=90]{Imagenes/VLT_b.jpg} }

  \caption{Vista anversa y reversa de VLT}\label{vltFB}
\end{figure}

\subsection{SCUI - Lector de tarjetas de contacto e Interfaz de Usuario}
El módulo SCUI puede dividirse en dos partes, una de ellas es un lector de tarjetas de contacto basadas en la norma ISO7816, y la otra es una simple interfaz para el usuario.
El lector de tarjetas de contacto (smart cards), está compuesto por un conversor full duplex a half duplex el cual se encuentra conectado a uno de los puertos UART de la SBC a través del módulo VLT, que se describió en el punto anterior. Este conversor permite la transmisión de datos directamente entre la tarjeta y la SBC, sin necesidad de intercalar un ASIC para el manejo de tarjetas del tipo ISO7816. Cuenta también con un oscilador para alimentar la entrada de reloj de las tarjetas. La entrada de control (OE) del oscilador operada desde la SBC permite poner la salida de reloj en tercer estado, cosa muy útil a la hora de cumplir con la secuencia de inicialización de las tarjetas descritas en el estándar. El lector permite operar con tarjetas clase A (alimentadas a 5 Volt) y clase B (alimentadas a 3,3 Volt) haciendo uso de un jumper que permite intercambiar la tensión de alimentación suministrada a la tarjeta. Se cuenta con un zócalo para insertar la tarjeta de contacto.
Por más detalles ver esquemático en la figura \ref{Fig:SAM}.

Por otra parte, la intefaz de usuario está compuesta por tres leds (verde, amarillo y rojo), buzzer y un display LCD16x2 donde son desplegados los mensajes que indican al usuario la operación que se efectúa sobre su tarjeta Mifare.
Por más detalles ver esquemático en la figura \ref{Fig:UI}.

El último elemento a describir aquí es un conector receptáculo 5x2 (100mils) en el que se conecta el módulo lector/escritor RFID que opera con las tarjetas RFID Mifare.
Por más detalles ver esquemático en la figura \ref{Fig:SCUI}.

\begin{figure}[H]
  \centering
  \subfigure[Vista anversa de SCUI]{\label{scuiF}
  \includegraphics[scale=.07]{Imagenes/SCUI_f.jpg} } 
  \subfigure[Vista reversa de SCUI]{\label{scuiB} 
  \includegraphics[scale=.08]{Imagenes/SCUI_b.jpg} }

  \caption{Vista anversa y reversa de SCUI}\label{scuiFB}
\end{figure}

\newpage
\subsection{Lector/Escritor RFID}
Este módulo es el encargado de la comunicación con las tarjetas RFID que cumplen con la norma ISO14443. Consta básicamente de cuatro secciones entre las que se encuentran: el integrado CL RC632; el filtro EMC, el circuito de adaptación de impedancia (matching); y el inductor de la antena. 
El ASIC CL RC632 permite, por un lado la comunicación digital con un microprocesador a través de su puerto de datos y por el otro lado la transmisión de datos hacia la antena que emitirá la señal RF para la comunicación con las tarjetas ISO14443.
Por más detalles ver esquemático en la figura \ref{Fig:RFID}.

Lo que se llama propiamente antena RF está conformada por el circuito de adaptación de impedancia (matching) y por el inductor, que propaga el campo magnético para lograr el acoplamiento necesario entre lector y tarjeta, de aquí la sigla PCD (Proximity Coupling Device).
Por más detalles ver esquemático en la figura \ref{Fig:RFID2}.

Los principios básicos de funcionamiento de la antena se detallan en el apéndice \ref{anx_antena}.

\begin{figure}[H]
  \centering
  \subfigure[Vista anversa del lector/escritor RFID]{\label{scuiF}
  \includegraphics[scale=.08]{Imagenes/ant_f.jpg} } 
  \subfigure[Vista reversa del lector/escritor RFID]{\label{scuiB} 
  \includegraphics[scale=.08]{Imagenes/ant_b.jpg} }

  \caption{Vista anversa y reversa del lector/escritor RFID}\label{l/eRFID}
\end{figure}

\bigskip
\bigskip
Por detalles de esquemáticos y listas de componentes referirse al apéndice \ref{docHW}.

\newpage
\section{Funcionamiento de m\'odulos}

\subsection{SBC}
La SBC está formada por un SOC y memoria suficiente para ejecutar un sistema operativo GNU/Linux orientado a desarrollar sistemas embebidos. Sobre el sistema operativo se instalan los módulos y bibliotecas necesarias para hacer uso del hardware que contiene la SBC. En la aplicación se utilizará uno de sus puertos SPI para la comunicación con el lector/escritor de tarjetas RF, un puerto UART para la comunicación de datos con el lector de tarjetas de contacto y varias salidas GPIO para el control de la interfaz de usuario.

\subsection{VLT - Conversor de Voltajes}
El corazón de esta placa son los integrados TXB0108 \cite{HD_VLT} (ver hoja de datos en el apéndice \ref{HD}) que permiten la interconexión de dispositivos que operan en distintos niveles de tensión. Básicamente el integrado está constituído por dos puertos, puerto A y puerto B cada uno de 8 bits. El puerto A opera con la tensión de 1,8 Volt que permite ser conectado a la Beagleboard, el puerto B opera con la tensión de 3,3 Volt cuando se encuentra conectado al CL RC632, y de 5 Volt para los restantes periféricos.
Cada I/O de un puerto es sensible a los flancos de subida o bajada, trasladando estos cambios a la I/O correspondiente del puerto opuesto. 
Este integrado posee también una entrada de control para poner los puertos en estado de alta impedancia.
Una ventaja es que no poseen entrada de control de dirección de flujo de datos, de modo que se ahorran pines de control que no se tienen disponibles en la Beagleboard.
En la figura \ref{Fig:Celda_TXB0108} se puede observar como están constituídas cada una de las entradas/salidas del integrado.
Otra pieza que compone esta placa es el regulador de tensión LDO implementado a partir del integrado LM1117 \cite{LDO} (ver hoja de datos en el apéndice \ref{HD}), éste se utiliza para convertir la entrada de tensión de 5 Volt en una salida de tensión de 3,3 Volt y así poder alimentar el periférico correspondiente.

Por más detalles ver esquemático en la figura \ref{Fig:VLT}.

\bigskip
\bigskip
\begin{figure}[H]
\centering
  \begin{center}
  \includegraphics[scale=.3]{Imagenes/TXB0108.png} 
  \end{center}
  \caption{Arquitectura de una celda I/O del TXB0108}\label{Fig:Celda_TXB0108} 
\end{figure}


\subsection{SCUI - Lector de tarjetas de contacto e Interfaz de Usuario}

\leftline{\bf{Lector de tarjetas de contacto ISO7816}}

Es un lector muy simple de implementar, su construcción se basa en un conversor full a half duplex construido a partir de un circuito transistorizado trabajando en zona de corte y saturación. Los transistores empleados son el NPN 2N3904 (ver hoja de datos en el apéndice \ref{HD}) y el PNP 2N3906 (ver hoja de datos en el apéndice \ref{HD}) los cuales fueron seleccionados en base a su rápida característica de conmutación que es del orden de algunas decenas de nanosegundos. Dada la característica del circuito, es posible recibir el eco de la transmisión de datos generados por la SBC. 
Un elemento fundamental que compone el circuito del lector es el oscilador de frecuencia 3,579545 MHz, este valor no es antojadizo sino que permite generar la base de tiempo adecuada para la transmisión de datos entre la tarjeta y la SBC. Otras frecuencias de reloj fueron empleadas, como ser 4 MHz y 5 MHz, con resultados inciertos en la recepción de los datos, aún cuando sería posible usar estos valores según la referencia \cite{SCHb} para los parámetros obtenidos desde el ATR de la tarjeta. 
El circuito cuenta también con protección de descarga ESDA6V1W5 (ver hoja de datos en el apéndice \ref{HD}) para los contactos de la smart card.

Por más detalles ver esquemático en la figura \ref{Fig:SAM}.

\newpage
\leftline{\bf{Interfaz de usuario}}

El elemento a destacar es un display LCD16x2 que basa su funcionamiento en el controlador Hitachi HD44780 \cite{dpy} (ver hoja de datos en el apéndice \ref{HD}). La transferencia de datos hacia el display se hace a través de un puerto con 4/8 bits de datos y 3 bits de control. Debido a que no se cuenta con la cantidad de pines disponibles en la Beagleboard para operar en el modo de 8 bits, se empleó en su lugar el modo 4 bits del display. El bit de control RS indica si el byte a enviar por el puerto de datos es una palabra de control o un caracter ASCII a ser almacenado en la memoria interna del display. El bit R/W por su parte indica si se efectuará una lectura o una escritura de la memoria interna del display. Por último en el bit E se indica mediante flanco de bajada que se ejecute la operación indicada con los anteriores dos bits de control, previo a este flanco las señales en el puerto de datos deben permanecer fijas.
El display cuenta también con una entrada para calibrar el contraste del LCD, la calibración se realiza a partir de un divisor resistivo implementado con resistencias y un preset.
El backlight del display es accionado desde uno de los pines de la SBC a partir de un circuito transistorizado que opera en zona de corte/saturación.
Los restantes elementos que componen la interfaz de usuario son leds y buzzer que son accionados directamente desde los pines del puerto de expansión de la SBC.

Por más detalles ver esquemático en la figura \ref{Fig:UI}.

\subsection{Lector/Escritor RFID}
En el corazón del lector/escritor de tarjetas RFID, se encuentra el chip CL RC632 que forma parte de una familia de integrados empleados para la comunicación con tarjetas sin contacto, pertenecientes a la norma ISO14443 las cuales operan a la frecuencia 13,56 MHz.
El CL RC632 soporta todas las capas del esquema de comunicación que se establecen en la mencionada norma, incluyendo el algoritmo de seguridad (CRYPTO1) para autenticar las tarjetas Mifare Classic. En lo que sigue se describen algunas de las características principales del integrado.

Por más detalles ver esquemático en la figura \ref{Fig:RFID}.

\bigskip
\leftline{\bf{Interfaz}}

Los comandos, bits de configuración y las banderas se acceden a través de la interfaz con un microprocesador. El puerto elegido para la comunicación desde la SBC es el SPI, aunque es posible la comunicación a través de su puerto paralelo. 

\bigskip
\leftline{\bf{Registros}}\label{Registros}

La configuración del chip se lleva a cabo a partir de un mapa de registros de control que se encuentra dividido en 8 páginas con 8 registros cada una. La manera de alcanzar estos registros es mediante el intercambio de página, mecanismo que puede ser deshabilitado mediante escritura de un “1” en el bit 7 del registro 0 en la página 0, logrando direccionamiento plano. La función de cada uno de sus registros puede ser observada en la hoja de datos del integrado \cite{RC632} (ver apéndice \ref{HD}).

\bigskip
\leftline{\bf{Memoria EEPROM}}

La memoria está dividida en 32 bloques con 16 bytes cada uno.
El contenido de memoria EEPROM en los bloques 1 y 2 (dirección 10hex a 2Fhex) se utilizan para configurar los registros del CL RC632 durante la fase de inicialización, de forma automática.
La configuración por defecto soporta la comunicación Mifare ISO 14443 A, aunque los usuarios pueden especificar la inicialización para I-Code1, ISO 15693 o ISO 14443 B, mediante los bloques de memoria 3 al 7.
Se reservan 384 bytes para almacenar las claves CRYPTO1 que son usadas para la autenticación con las tarjetas. El formato de una de estas claves puede verse en \cite{RC632} (ver apéndice \ref{HD}) y tiene una longitud de 12 bytes, por tanto es posible almacenar en memoria las 32 claves que posee una tarjeta.

\bigskip
\leftline{\bf{Buffer FIFO}}

El integrado contiente un buffer FIFO de 64 bytes para flujo de datos con un microprocesador.
La entrada y salida del buffer de datos está conectado con el registro FIFOData. Escribir en este registro almacena un byte en el buffer e incrementa el puntero de escritura del buffer. La lectura de este registro muestra el contenido del buffer e incrementa el puntero de lectura. La distancia entre el puntero de escritura y lectura se puede obtener mediante la lectura del registro FIFOLength, indicando así la cantidad de bytes que se llevan almacenados. Es posible observar y controlar el estado del buffer mediante varios registros, para evitar que se produzcan errores de comuncicación con el microprocesador.

\bigskip
\leftline{\bf{Interrupciones}}

El CL RC632 indica ciertos eventos estableciendo el bit IRQ en el registro \\
PrimaryStatus, y además, por la activación del pin IRQ. La señal en el pin IRQ se puede utilizar para interrumpir un micrprocesador. 
Las posibles fuentes de interrupción son: 

\begin{itemize}

\item Timer, a través de su bandera TimerIRq 
\item Transmisor, coprocesador CRC y memoria E2PROM, a través de su bandera TxIRq 
\item Receptor, a través de su bandera RxIRq 
\item Registro de comando, a través de su bandera IdleIRq 
\item Buffer FIFO, a través de sus banderas HiAlertIRq y LoAlertIRq 

\end{itemize}

El CL RC632 informa al microprocesador sobre el origen de una interrupción mediante el establecimiento del bit adecuado en el registro InterruptRq. La relevancia de cada bit de petición de interrupción como fuente de una interrupción puede ser enmascarada con el bit de habilitación de interrupciones en el registro InterruptEn. 
Si alguna bandera de solicitud de interrupción se establece en 1 (una solicitud de interrupción está pendiente) y la correspondiente bandera de habilitación de interrupción está en "1", la bandera de estado IRq en el registro PrimaryStatus se establece en 1. 
Por otra parte diferentes fuentes de interrupción pueden estar activas al mismo tiempo. Por lo tanto, se hace un OR con todos los bits de solicitud de interrupción, el resultado se envía a la bandera IRq y se conecta al pin IRQ. 
Los bits de petición de interrupciones están seteados de forma automática por las máquinas de estado internas del CL RC632. Adicionalmente, el microprocesador tiene acceso para setearlos o borrarlos. 
Una implementación especial de los registros InterruptRQ y InterruptEn permiten el cambio de un único bit de estado sin tocar el resto. 

\bigskip
Configuración del Pin IRQ:
El nivel lógico de la bandera de estado IRq es visible por el pin IRQ. Además, la señal en el pin puede ser controlada por los siguientes bits del registro IRQPinConfig 
\begin{itemize}
\item IRQInv: Si este bit es 0, la señal en el pin IRQ es igual al nivel lógico del bit IRq. 
Si es 1, la señal en el pin IRQ está invertida con respecto al bit IRq. 
\item IRQPushPull:  Si este bit es 1, el pin IRQ tiene características de una salida estándar 				   CMOS, de otra manera la salida es open drain y un resistor externo es necesario para alcanzar un nivel alto en este pin. 
\end{itemize}

Para poder hacer uso de lo descrito anteriormente se previó y reservó una entrada en el conector de expansión de la Beagleboard (ver figura \ref{Fig:VLT}), sin embargo el software empleado no hace uso del mecanismo de interrupciones sino que opera mediante polling.

\bigskip
\leftline{\bf{Transmisor, pines Tx1 y Tx2}}

La señal en Tx1 y Tx2 es la portadora, centrada en 13,56 MHz, modulada ASK 100\% con los datos a transmitir. Estos pines son conectados directamente a la antena para propagar la señal RF hacia las tarjetas RFID. La distancia de operación alcanzada es de hasta 10cm de longitud, dependiendo de la geometría de la antena, así como también adaptación de impedancia lograda, entre otros (ver \cite{MRICF} y \cite{RFIDPA} incluídas en el apéndice \ref{HD}).
Algunos registros del integrado permiten la configuración del transmisor, posibilitando entre otras cosas apagar la señal portadora en caso de ser necesario.

\bigskip
\leftline{\bf{Conjunto de comandos}}

El CL RC632 opera como una máquina de estado capaz de interpretar y ejecutar un conjunto de comandos pre establecidos. La ejecución de uno de ellos es posible escribiendo su código correspondiente en el “Registro de Comandos”, si fuera necesario el pasaje de parámetros, éstos se colocarán en el buffer FIFO mencionado antes. 
Una lista detallada de comandos junto con los parámetros necesarios es mostrada en la hoja de datos (ver apéndice \ref{HD}), entre ellos se pueden destacar los siguientes: Authent, Transceive, LoadKey.

\bigskip
\leftline{\bf{Antena RF}}

En lo que sigue se describen algunas de las partes que integran la antena RF que se conecta directamente a los pines Tx1 y Tx2 del integrado descrito antes.

\bigskip
\leftline{\bf{Filtro EMC}}

La frecuencia de la portadora de la señal transmitida se centra en 13,56 MHz, sin embargo se generan también armónicos de mayor frecuencia. Para cumplir con la regulación internacional EMC es que se agrega este filtro pasa bajos, cuya frecuencia de corte debe ubicarse en 14,4 MHz, o sea 13,56 MHz más 847,5 KHz para permitir el ancho de banda necesario que logre el baud rate requerido en la transmisión de los bits. 
En síntesis el filtro ayuda a mejorar la relación señal a ruido para la señal recibida y decrementa el sobretiro en los pulsos transmitidos mejorando la calidad de la señal transmitida.
Los valores propuestos para los componentes de este filtro se encuentran en las notas de aplicación \cite{MRICF}.

\bigskip
\leftline{\bf{Matching}}

Por su parte, el circuito de adaptación de impedancia permite que la antena resuene a la frecuencia deseada, en este caso 13,56 MHz. Los valores de los elementos que conforman este circuito deben ser estimados y sintonizados a partir del diseño del inductor de la antena.
El factor de calidad total de la antena debe ser tenido en cuenta para cumplir con los requerimientos establecidos en la norma ISO14443. 
El mecanismo para el cálculo de los elementos que foman este circuito se detallan en las notas de aplicación \cite{MRICF}.

\bigskip
\leftline{\bf{Inductor}}

El inductor de la antena es quien propaga el campo magnético para la transmisión de datos hacia las tarjetas. El diseño de la antena comienza a partir de este elemento.
El cálculo detallado del valor del inductor se encuentra en las notas de aplicación \cite{ACD}, aunque su costo y tiempo en la práctica son considerables; una estimación del valor de la inductancia puede verse en el apéndice \ref{anx_antena}, en el que se deben tener en cuenta los siguientes elementos: geometría de la antena, ancho y espesor del conductor del PCB, longitud de una espira, número de vueltas, etc.

\bigskip
\leftline{\bf{Receptor}} 

El circuito receptor de la antena se encuentra bien detallado en las notas de aplicación \cite{MRICF} y no fue necesario efectuar ningún cambio para lograr buenos resultados en este diseño particular. 

Por más detalles ver esquemático en la figura \ref{Fig:RFID2}.
\chapter{Documentos y esquem\'aticos del hardware}

\section{Herramientas de diseño}
\subsection{SBC}
\subsection{VLT - Conversor de Voltajes}
\subsection{SCUI - Lector de tarjetas de contacto e Interfaz de Usuario}
\subsection{Lector-Escritor RFID}


%capítulo 6
\chapter{Software}

%sección 6.1
\section{Introducción}
Se debe destacar que todo el desarrollo de software se basa exclusivamente en herramientas de software libre. La distribución Linux elegida para el sistema embebido se llama Angström. Ésta distribución es muy usada en aplicaciones que usan una Beagleboard y cuenta con una gran cantidad de bibliotecas implementadas en lenguaje C, que permiten una gran escalabilidad a la hora de incorporar nuevos periféricos en la aplicación.

%sección 6.2
\section{Arquitectura de Software}
%sección 6.2.1
\subsection{Descripción}
Un sistema linux se compone de diferentes partes que interactúan entre sí, formando capas ordenadas con distintos grados de abstracción respecto al hardware. Esto lo podemos apreciar en la figura \ref{Fig:SW} donde se muestra a grandes rasgos el sistema implementado. 

\begin{figure}[H]
\centering
  \begin{center}
  \includegraphics[scale=.4]{Imagenes/SW.jpg} 
  \end{center}
  \caption{Sistema RF${^{2}}$}\label{Fig:SW} 
\end{figure}

El bootloader es la parte del sistema más primitiva y su función es la de cargar el
kernel en memoria RAM para su ejecución. En general el bootloader en sistemas embebidos
es una aplicación que se divide en dos etapas, la primera etapa es fuertemente dependiente
del CPU con que cuenta la placa y su función es buscar en particiones activas para luego cargar 
en memoria RAM la segunda etapa del bootloader. Esta segunda etapa se encarga de descomprimir
en memoria RAM la imagen comprimida del kernel para luego ser ejecutado y que éste tome el
control del sistema.
El kernel se encarga a grandes rasgos de habilitar interrupciones, configurar la memoria y montar un sistema de archivos primitivo que permite a su vez cargar los módulos necesarios para la interfaz con periféricos. Luego se monta el verdadero sistema de archivos (fileSystem). En este nuevo sistema de archivos es donde se instalarán diferentes programas y bibliotecas para una correcta ejecución de nuestra aplicación.
En funcionamiento toda la comunicación con periféricos se realiza a través del kernel que es la parte más cercana al hardware.
Cada vez que se ejecuta una aplicación, ésta hace uso de las bibliotecas para poder comunicarse con el kernel, y éste se encarga de la comunicación con los periféricos. Las bibliotecas pueden ser nativas como es el caso de la biblioteca de lenguaje C o desarrolladas para que nuestra aplicación funcione correctamente.


%sección 6.2.2
\subsection{Sistema Operativo}
La Beagleboard al arrancar tiene la posibilidad de buscar el bootloader en NAND o en dispositivos extraíbles tales como memorias USB o memorias SD, lo mismo sucede con el kernel. Para nuestro sistema, elegimos un arranque a través de una memoria SD ya que es más fácil de manipular.

En la figura \ref{Fig:SD} se puede ver como queda distribuída la memoria SD con las distintas partes
que conforman el sistema operativo. 

\begin{figure}[H]
\centering
  \begin{center}
  \includegraphics[scale=.4]{Imagenes/sd.jpg} 
  \end{center}
  \caption{Memoria SD}\label{Fig:SD} 
\end{figure}

En la memoria SD se pueden distinguir dos particiones, una en formato FAT32 y otra
en formato ext3. La partición en FAT32 es llamada “de arranque” y es donde se encuentra 
el bootloader (MLO, u-boot.bin) y la imagen comprimida del kernel (uImage). 
La partición en ext3 es donde se encuentra el sistema de archivos (fileSystem).

El MLO es el equivalente al bootloader de la primera etapa; en general ya viene precargado en la memoria NAND de la Beagleboard. Es posible generarlo o incluso bajar una versión ya compilada desde la web de Angström. Como característica principal tiene la capacidad de buscar el u-boot.bin en dispositivos extraíbles como memorias SD o USB.

El u-boot.bin es equivalente al bootloader de la segunda etapa. Al igual que el MLO, es posible generarlo o incluso bajarlo de la web de Angström. En nuestro sistema fue necesario generarlo ya que configura el bloque de expansión de la Beagleboard.

El uImage es el kernel del sistema. Fue necesario generarlo ya que se debieron modificar sus fuentes para que queden habilitadas las interfaces de comunicación con los dispositivos periféricos.

El fileSystem es el correspondiente a una distribución linux llamada Angström. Se pueden llegar a precargar distintos programas y bibliotecas dependiendo de la forma en que lo generemos.
Angström es una distribución linux diseñada específicamente para sistemas embebidos desarrollados
para SBCs como la usada para este prototipo. Esto lo hace más eficiente que otros sistemas operativos. La elección de esta distribución se debió a que es de los más recomendados y utilizados en la documentación y foros de Beagleboard.

%sección 6.2.3
\subsection{Bibliotecas}
librfid es una biblioteca de software libre para manejo de lectores/escritores RFID. Implementa, en el dispositivo lector/escritor, el stack de protocolos ISO 14443A, ISO 14443B, ISO 15693, Mifare Ultralight y Mifare Classic.
Entre los lectores soportados están OpenPCD y algunos modelos Omnikey, estos lectores tienen  una interfaz de conexión USB. Además la librfib tiene soporte para cualquier otro lector con comunicación directa con el CL RC632 mediante la interfaz SPI y es por esta razón que se tuvo en cuenta.

\bigskip
Existe una herramienta desarrollada que ayuda a entender el funcionamiento de la biblioteca, librfid-tool.

%sección 6.3
\section{Herramientas utilizadas en el desarrollo del sistema}

%sección 6.3.1
\subsection{Introducción}
Para el desarrollo de sistemas, existe una gran variedad de herramientas útiles, algunas de software libre y otras privativas. El hecho de tener tantas opciones disponibles, a pesar de ser una ventaja algunas veces dificulta la elección de las herramientas correctas.

Para la elección de las herramientas se tomó como primer criterio de decisión el hecho de que sean libres así como las experiencias de otras personas que ya han transitado caminos comunes, consultando y participando en foros activos.

A continuación se detallan las herramientas utilizadas para el desarrollo del sistema. 
Primero se da una descripción de las herramientas elegidas y luego se comentan otras que se probaron con igual o peor resultado que las herramientas elegidas en última instancia.

%sección 6.3.2
\subsection{Generación de MLO, u-boot.bin y uImage}
No fue necesario generar el MLO debido a su simpleza, puesto que el binario precompilado realiza bien su función.

El u-boot.bin y el uImage fueron generados con la herramienta de desarrollo y compilación OpenEmbedded-Bitbake que es una fusión de dos herramientas: OpenEmbedded (herramienta para construcción y mantenimiento de distribuciones) y Bitbake (herramienta de compilación similar al Make que automatiza la construcción de ejecutables entre otros). OpenEmbedded utiliza Bitbake para su objetivo. Es una herramienta muy potente y difícil de aprender al principio. Luego de entendido su principio de funcionamiento se hace muy simple su uso, para lo que es necesario tener acceso a una buena conexión a internet.
Con esta herramienta también se pueden generar el MLO y el filesystem, aunque se prefirió utilizar otras herramientas por sobre ésta. 
Su instalación, configuración, estructura y uso se pueden ver en el apéndice \ref{anx_sw_oe}.

%sección 6.3.3
\subsection{Generación de FS}
Para la generación del fileSystem de Angström, se utilizó la herramienta web Narcissus.
Esta herramienta permite seleccionar entre diferentes dispositivos entre los cuales está Beagleboard, los programas que se quieran instalar, el formato de la imágen seleccionada e incluso se puede generar un kit de desarrollo (SDK) para el host. Debido a la facildad de uso y a los buenos resultados obtenidos, se decidió utilizar esta opción por sobre la del filesystem generado por la herramienta OpenEmbedded-Bitbake.

%sección 6.3.4
\subsection{Croscompilación}
Para la croscompilación se utilizó el SDK generado por Narcissus y la herramienta Make para generar los archivos necesarios. La instalación del SDK se encuentra en el APÉNDICE.

%sección 6.3.5
\subsection{Depuración de código}
Para la depuración, se utilizó la herramienta GDB del proyecto GNU. 
Al momento de compilar, es necesario agregar la opción -g para que la aplicación pueda ser depurada. Esta opción agrega información en el código de la aplicación.
La interfaz del GDB es por consola, aunque existen algunos programas que utilizan GDB y además ofrecen una interfaz gráfica (DDD[referencia]).
Algunos de los comandos útiles y sus usos más comunes son: 

\bigskip
breakpoint: para colocar un breakpoint. En general se lo llama seguido del nombre de una función de la aplicación.

print: seguido del nombre de una variable, muestra el contenido de la variable durante el proceso de depuración. Si la variable es local a alguna función, el valor de la variable se pierde al salir de la función.

next o “n”: sirve para ir línea a línea en modalidad step-over (sin entrar a las funciones).

step o “s”: sirve para ir línea a línea en modalidad step-into (entrando a las funciones).

backtrace o “bt”: despliega el stack de llamadas a funciones, sirve para saber por donde se pasó y donde estamos.

\bigskip
Sin olvidarnos del hecho de que la aplicación RF$^{2}$ está diseñada para una arquitectura distinta a la del PC de desarrollo, para el depurado de la aplicación existen dos alternativas. 

La primera es lo que se podría llamar depuración local, esto es, instalar GDB en la Beagleboard y depurar la aplicación en ésta. Para saber lo que sucede es necesario acceder de forma remota a ésta desde el PC de desarrollo. 

La segunda opción es la depuración remota. La depuración remota consiste en realizar la depuración de la aplicación desde el PC de desarrollo. Para esto, es necesario instalar GDBServer en la Beagleboard y tener instalado el GDB específico de la Beagleboard en el PC de desarrollo. Luego se establece una conexión que puede ser serial o ethernet entre la Beagleboard y el PC de desarrollo. Por más detalles sobre la configuración referirse al apéndice.

\bigskip
La primera opción no es posible para sistemas embebidos chicos en los cuales no se puede instalar GDB, aunque éste no es el caso de la Beagleboard. 
Se tienen más y mejores herramientas en el PC de desarrollo, por ejemplo programas con interfaz gráfica que ayudan a entender mejor lo que está pasando. Por algunas de estas razones, se prefiere el uso de la depuración remota. 

\bigskip
Se utilizaron indistintamente tanto la primera opción como la segunda.

%sección 6.3.6
\subsection{Bibliotecas}
librfid-tool es una herramienta de uso por línea de comandos que da acceso de bajo nivel RFID utilizando los lectores soportados por la librfid.

\bigskip
A continuación se detallan: la forma de llamar a la aplicación y algunas de las opciones soportadas.

\bigskip
librfid-tool -[opción]

\bigskip
Dentro de las opciones:

s: realiza una búsqueda de tarjetas RFID hasta encontrar una.

S: loop infinito con la opción -s, muestra información sobre la tarjeta RFID encontrada en cada paso.

p: especifica el protocolo RFID a utilizar, entre las opciones se encuentran tcl, mifare-classic y  mifare-ultralight.

l: especifica el protocolo de capa 2 a utilizar, entre las opciones se encuentran ISO14443a, ISO14443b y ISO15693.

h: ayuda.

\bigskip
Ejemplos de uso recomendados:

\bigskip
\$ librfid-tool -p mifare-classic

Si encuentra una tarjeta con protocolo Mifare-classic, devuelve la lectura completa de los bloques de memoria de la tarjeta.

\bigskip
\$ librfid-tool -S

Devuelve el UID y el protocolo soportado por la tarjeta.

%sección 6.4
\section{Desarrollo}

%sección 6.4.1
\subsection{MLO}

%sección 6.4.2
\subsection{Multiplexado de pines}
El microprocesador OMAP3530 tiene muchos pines con distintas interfaces entre las
que se cuentan puertos UART, SPI, GPIO, etc., pero no todos son accecibles desde la BeagleBoard. 
Para poder acceder a algunos de estos puertos del microprocesador, existe en la placa de la
BeagleBoard un bloque de expansión de 28 pines.

\bigskip
Por defecto en el bloque de expansión no se encuentran las señales que se quieren. Esto 
lleva a que se tenga que modificar el estado inicial de los pines. 
Existen dos formas de modificar los pines de modo de tener las señales que se precisan. Una de ellas es modificar el bootloader, la otra es modificar el kernel. Esto implica cambios
en los archivos fuentes y posterior compilación que genere los nuevos binarios u-boot o uImage. 

\bigskip
Para la modificación de las señales disponibles en el bloque de expansión se decidió modificar el u-boot ya que la modificación por u-boot es más intuitiva y por experiencia se sabe que lo que más se actualiza y/o modifica es el kernel. 

%sección 6.4.3
\subsection{u-boot}
Como se mencionó anteriormente en el u-boot se realiza la configuración de los pines del bloque de expansión de la Beagleboard. “hacer referencia a tabla en algún lado”
Cada pin del bloque de expansión tiene varias funcionalidades asociadas, y la configuración de una 
funcionalidad depende de un multiplexado modificable a nivel de software. Esto es, dependiendo del “modo de pin” elegido, la función que se obtiene en dicho pin.
Para que los cambios hechos en el u-boot tengan el efecto esperado al arrancar el sistema, es necesario que en la configuración del kernel esté la opción CONFIG\_OMAP\_MUX=no, lo que imposibilita al kernel de realizar este mismo cambio. Esta opción no está activada por defecto en ninguna versión actual del kernel. 

\bigskip
Antes que pueda ser modificado el estado de los pines del bloque de expansión,
es necesario obtener los archivos fuentes con los cuales se genera el archivo binario u-boot.

\bigskip
Nota: Puede que cuando se realiza la instalación de OpenEmbedded-Bitbake (apéndice), se genere un directorio relacionado con u-boot en /stuff/build/tmp/work/beagleboard-angstrom-linux-gnueabi/, si esto es así, no es necesario volver a obtener los fuentes.

\bigskip
\leftline{Se configura el bitbake para poder utilizarlo:}

\centerline{\$ export BBPATH=/stuff/build:/stuff/openembedded}

\centerline{\$ export PATH=/stuff/bitbake/bin:\$PATH}

\bigskip
\leftline{Se obtienen los fuentes:}

\centerline{\$ cd /stuff/build}

\centerline{\$ bitbake -f -c clean -b ../openembedded/recipes/u-boot/u-boot\_git.bb}

\centerline{\$ bitbake -f -c compile -b ../openembedded/recipes/u-boot/u-boot\_git.bb}

\bigskip
Los fuentes se encuentran en 

/stuff/build/tmp/work/beagleboard-angstrom-linux-gnueabi/u-boot.../git/.

\bigskip
En los fuentes del u-boot dentro de board/ti/beagle/ se encuentra el archivo beagle.h que es donde se establece la configuración de los pines del bloque de expansión de la Beagleboard.

Si se abre este archivo se ven líneas del estilo: 

\begin{verbatim}
MUX_VAL(CP(MCBSP3_DX), (IEN | PTD | DIS | M4)) /*GPIO_140*/\
\end{verbatim}

MUX\_VAL indica que se va a modificar el valor de multiplexado de lo que está entre paréntesis. 

\bigskip
CP(MCBSP3\_DX) es el Control\_PadConf, esto es el registro del microprocesador asociado con el 
pin a modificar. 

\bigskip
(IEN $|$ PTD $|$ DIS $|$ M4) esta es la configuración del pin en cuestión: 


La opción IEN (input enable) hace que el pin sea bidireccional. 

La opción PTD y PTU, indica si el pin tiene un pull down o pull up respectivamente. 

La opción DIS y EN, indica si se deshabilitan o no las opciones PTD y PTU. 

La opción M4 es el modo seleccionado para del pin. Para la Beagleboard existen 4 modos: M1, M2, M3 y M4.

GPIO\_140 es el nombre de la señal (solo es un comentario). 

\bigskip
El Control\_PadConf es un registro de 32bits el cual controla el estado de dos pines, esto es, la parte 
baja del registro controla un pin y la parte alta controla otro. 

\bigskip
Analizando el “manual de referencia BeagleBoard del usuario” (adjunto en el apéndice \ref{HD})(“Expansion connector signals” – tabla 20), “manual técnico de referencia OMAP35x” (“SCM functional description” – capítulo 7.4.4) y agregando las opciones que interesan para los pines, se obtuvo la siguiente tabla: 

\bigskip
{\bf{agregar la tabla!}}


\bigskip
Al modificar el archivo beagle.h hay que tener mucho cuidado ya que al sustituir los valores no se deben repetir pines ni registros, no deben haber incoherencias, un registro por cada pin y un pin por cada registro. 
Dentro del archivo hay un macro definido MUX\_BEAGLE\_C(), donde se deben realizar las modificaciones ya que el modelo utilizado de Beagleboard es el C4 y éste macro es el utilizado para dicho modelo.
En una primera instancia se sustituyeron los valores de la TABLA buscando los equivalentes del 
PadConf en el beagle.h. 


\begin{verbatim}
\#define MUX_BEAGLE_C() \
MUX_VAL(CP(MCBSP3_DX),   (IEN  | PTD | DIS | M4))/*GPIO_140*/\
MUX_VAL(CP(MCBSP3_DR),   (IEN  | PTD | DIS | M4))/*GPIO_142*/\
MUX_VAL(CP(MCBSP3_CLKX), (IEN  | PTD | DIS | M4))/*GPIO_141*/\
MUX_VAL(CP(MCBSP3_FSX),  (IEN  | PTD | DIS | M1))/*UART2_RX*/\
MUX_VAL(CP(UART2_TX),    (IDIS | PTD | DIS | M0))/*UART2_TX*/\
MUX_VAL(CP(MMC2_DAT7),   (IEN  | PTD | EN  | M4))/*GPIO_139*/\
MUX_VAL(CP(UART2_CTS),   (IEN  | PTD | DIS | M4))/*GPIO_144*/\
MUX_VAL(CP(MMC2_DAT6),   (IEN  | PTD | EN  | M4))/*GPIO_138*/\
MUX_VAL(CP(MMC2_DAT5),   (IEN  | PTD | EN  | M4))/*GPIO_137*/\
MUX_VAL(CP(MMC2_DAT4),   (IEN  | PTD | EN  | M4))/*GPIO_136*/\
MUX_VAL(CP(UART2_RTS),   (IEN  | PTD | EN  | M4))/*GPIO_145*/\
MUX_VAL(CP(MCBSP1_DX),   (IEN  | PTD | EN  | M4))/*GPIO_158*/\
MUX_VAL(CP(MMC2_DAT2),   (IEN  | PTD | EN  | M4))/*GPIO_134*/\
MUX_VAL(CP(MCBSP1_CLKX), (IEN  | PTD | EN  | M4))/*GPIO_162*/\
MUX_VAL(CP(MMC2_DAT1),   (IEN  | PTU | EN  | M4))/*GPIO_133*/\
MUX_VAL(CP(MCBSP1_FSX),  (IEN  | PTD | EN  | M4))/*GPIO_161*/\
MUX_VAL(CP(MCBSP1_DR),   (IEN  | PTD | EN  | M4))/*GPIO_159*/\
MUX_VAL(CP(MCBSP1_CLKR), (IEN  | PTD | EN  | M4))/*GPIO_156*/\
MUX_VAL(CP(MCBSP1_FSR),  (IEN  | PTD | EN  | M4))/*GPIO_157*/\
MUX_VAL(CP(I2C2_SDA),    (IEN  | PTD | EN  | M4))/*GPIO_183*/\
MUX_VAL(CP(I2C2_SCL),    (IEN  | PTU | EN  | M4))/*GPIO_168*/\
MUX_VAL(CP(MMC2_DAT3),   (IEN  | PTD | EN  | M1))/*SPI3_CS0*/\
MUX_VAL(CP(MMC2_DAT0),   (IEN  | PTU | EN  | M1))/*SPI3_SOMI*/\
MUX_VAL(CP(MMC2_CMD),    (IEN  | PTU | DIS | M1))/*SPI3_SIMO*/\
MUX_VAL(CP(MMC2_CLK),    (IEN  | PTU | DIS | M1))/*SPI3_CLK*/
\end{verbatim}

Luego es necesario compilar para obtener el u-boot.bin.

\bigskip
\centerline{\$ cd /stuff/build}

\centerline{\$ bitbake -f -c compile -b ../openembedded/recipes/u-boot/u-boot\_git.bb}

\centerline{\$ bitbake -f -c deploy -b ../openembedded/recipes/u-boot/u-boot\_git.bb}

\bigskip
Nota: Cada vez que se introduzca un nuevo cambio, no es necesario ejecutar el comando con la opción clean (lo que implica volver a bajar los fuentes), solo basta con recompilar.

\bigskip
El archivo generado (u-boot.bin) se encuentra en 

/stuff/build/tmp/deploy/glibc/images/beagleboard/ aunque con su nombre seguido de un número identificatorio, el cual debe ser borrado para poder mantener el nombre u-boot.bin.

\bigskip
Pese a que en la literatura y foros, se plantea lo contrario, no fue posible establecer los atributos valor y dirección de los pines GPIO mediante la modificación planteada (una posible solución puede verse en el apéndice \ref{anx_sw_uIm}). Lo que sí cambia efectivamente es el modo del pin, permitiendo obtener las interfaces adecuadas en el bloque de expansión.

%sección 6.4.4
\subsection{uImage}
La versión del kernel elegida fue la 2.6.32 que en el momento del desarrollo era la versión más estable.(ver si en algún lado puse compatibilidad con distro) Aunque también se hicieron pruebas con las versiones 2.6.29 y 2.6.37.
Durante el inicio, el kernel carga los módulos y controladores necesarios para el funcionamiento del 
hardware que forma parte del sistema embebido. También se montan las interfaces para poder interactuar con los distintos dispositivos a ser conectados a la Beagleboard como lo son: SPI, GPIO, UART, etc. Estas interfaces se encuentran bajo el directorio /dev en el sistema de archivos. En algunos casos, no aparecen algunas de las interfaces configuradas en /dev lo que lleva a modificar los fuentes del kernel para que esto así suceda. Este fue el caso de la interfaz SPI que no quedó mapeada en /dev pese a que había sido configurada en los fuentes del u-boot; también hubo problemas con los atributos, valor y dirección de los GPIO como se mencionó anteriormente; adicionalmente hacía falta un módulo para simular una conexión ethernet sobre una interfaz USB para establecer una conexión entre la Beagleboard y un PC como si fuera un enlace de red. 
Todo esto llevó a que se tuvieran que modificar los archivos fuentes del kernel como se muestra en el apéndice \ref{anx_sw_uIm}.
A continuación se detallan los pasos a seguir para la modificación de los fuentes del kernel:
 
\bigskip 
Comando necesarios para el desarrollo del uImage:

\bigskip
\centerline{\$ bitbake virtual/kernel -c comando}

\bigskip
virtual/kernel: refiere a que estamos generando un kernel.

\bigskip
Entre los comandos:

\bigskip
clean: borra el contenido del directorio work. Borra todos los cambios hechos en la configuración del uImage fuente y parches agregados.

patch: genera los archivos de configuración y el fuente del uImage. Además le aplica los parches.

menuconfig: abre el editor de la configuración del kernel.

compile: compila todo.

deploy: genera los archivos referidos en este caso al uImage (.config, módulos, uImage) y los guarda en el directorio /stuff/build/tmp/deploy/glibc/images/beagleboard/.

\bigskip
Nota: Es necesario que todos estos comandos sean ejecutados en el orden adecuado para que todo funcione correctamente.

\bigskip
Primero se configura bitbake para poder utilizarlo:

\centerline{\$ export BBPATH=/stuff/build:/stuff/openembedded}

\centerline{\$ export PATH=/stuff/bitbake/bin:\$PATH}

\bigskip
Se comienza el desarrollo:

\centerline{\$ cd /stuff/build}

\centerline{\$ bitbake virtual/kernel -c clean}

\centerline{\$ bitbake virtual/kernel -c patch}

\centerline{\$ bitbake virtual/kernel -c menuconfig}

\bigskip
Luego de ejecutar este comando se abre el editor de la configuración del kernel (ver figura XXX). Es este editor es donde se indican qué módulos cargar y cuales no. En este caso un cambio de  configuración es necesario para el buen funcionamiento de la interfaz SPI y de la conexión USB-Ethernet con la Beagleboard.

\begin{figure}[H]
\centering
  \begin{center}
  \includegraphics[scale=.3]{Imagenes/kernel.png} 
  \end{center}
  \caption{Editor de configuración del kernel}\label{Fig:kernel} 
\end{figure}

Para configurar la interfaz SPI, se debe configurar como sigue:

Device Drivers – SPI Support=y y luego como en la figura XX.

\begin{figure}[H]
\centering
  \begin{center}
  \includegraphics[scale=.4]{Imagenes/spi_chica.png} 
  \end{center}
  \caption{Configuración SPI}\label{Fig:spi} 
\end{figure}


Para poder establecer la conexión por USB con la Beagleboard, se debe configurar como sigue: 

Device Drivers – USB Support=y – USB Gadget Support=y y luego como en la figuraXX.

\begin{figure}[H]
\centering
  \begin{center}
  \includegraphics[scale=.4]{Imagenes/usb_chica.png} 
  \end{center}
  \caption{Configuración USB Gadget}\label{Fig:usb} 
\end{figure}

Luego, es necesario modificar el archivo board\_omap3beagle.c que se encuentra en /stuff/build/tmp/work/beagleboard-angstrom-linux-gnueabi/linux-omap-.../git/arch/arm/mach-omap2/. En este archivo está toda la inicialización de las interfaces. Los detalles de los cambios introducidos en este archivo se pueden observar en el apéndice XXXX

\bigskip
Ahora se compila y genera el archivo uImage:

\centerline{\$ bitbake virtual/kernel -c compile}

\centerline{\$ bitbake virtual/kernel -c deploy}

\bigskip
Dentro de /stuff/build/tmp/deploy/glibc/images/beagleboard/ se encuentra el archivo uImage generado.

\bigskip
Nota: Respecto al nombre del archivo, al igual que con el caso del u-boot.bin el nombre que aparece es un nombre más largo y necesita ser renombrado a uImage para que se pueda ejecutar correctamente.

%sección 6.4.5
\subsection{FileSystem}
\leftline{Desarrollo de fileSystem}

Como se nombró anteriormente, el filesystem se generó a partir de la herramienta web Narcissus.
En el fileSystem es donde se encuentran los paquetes y programas ya instalados. Cuanto más programas se instalen más grande será en tamaño el fileSystem.

\bigskip
Para utilizar la herramienta Narcissus debemos acceder a la siguiente dirección web: http://narcissus.angstrom-distribution.org/ 

\bigskip
A continuación se detallan las diferentes características y opciones a elegir para crear un filesystem a medida para nuestra SBC y un kit de desarrollo para el PC de desarrollo:

\bigskip
Select Machine: Beagleboard.

Image Name: el nombre que se le quiera dar.

Complexity: complejidad, se eligió advanced ya que la opción simple no brinda libertad de configuración.

Release: versión, aquí hay varias opciones disponibles, se eligió unstable ya que es la más estable de las disponibles. La primera opción disponible es la más estable.

Base System: aquí se elige el soporte de drivers y paquetes que se pretenden. bare bones es la opción con menos soporte y extended es la de mayor soporte. Cuanto más soporte, más pesado se hace el filesystem. Una opción interesante es la opción regular, y es la que se eligió.

/dev manager: esto es el manejador de /dev, se recomienda udev.

Type of Image: formato en el que se quiere descargar el filesystem. Se eligió tar.gz ya que es la opción más versátil.

Software manifest: se genera un archivo en la web con todos los paquetes que se instalaron en detalle.

SDK type: esta opción permite generar un kit de desarrollo para el PC de desarrollo compatible con el filesystem generado. Esto es sumamente útil por ejemplo para croscompilar. Aquí se eligió la opción Full SDK.

User environment section: aquí se indica que tipo de sistema operativo se quiere, básicamente se tienen dos opciones; una es un filesystem sin interfaz gráfica y las otra con entorno gráfico. Se eligió la opción console (sin entorno gráfico) ya que la aplicación no exige entorno gráfico.

\bigskip
Luego se permiten seleccionar programas que se desean instalar.
Las aplicaciones elegidas son: nano editor (editor de texto) ya que hace las cosas más fáciles que el programa vi, GDB y GDBServer necesarios para la depuración de la aplicación, toolchain para tener herramientas de compilación nativas en la Beagleboard.

\bigskip
Cuando todo fue seleccionado, se da un click en build me (demora un poco).
Cuando el proceso termina, se generan dos archivos comprimidos, un archivo con el nombre elegido para la imagen en un formato .tar.gz y el SDK para el PC de desarrollo en formato .tar.bz2. 

%sección 6.4.6
\subsection{Bibliotecas}

\leftline{\bf{Software para el manejo de GPIO}}

El módulo de software para el uso de los puertos de propósito general, GPIO, en principio puede resultar poco importante a simple vista, pero esta porción de código es usada por el resto de los módulos que conforman la aplicación completa del prototipo RF$^{2}$. 
Este módulo cuenta básicamente con una estructura que permite almacenar el estado de cada puerto, una macro y 4 funciones que se datallan a continuación.
La primera de las funciones se llama config\_gpio\_pin() y permite exportar desde el espacio kernel al espacio usuario las funcionalidades necesarias para hacer uso del puerto que se indica como argumento. Al momento en que se exporta, se indica la dirección, o sea si será un puerto de entrada o salida, a través de un parámetro que es pasado a la función.
La función que permite leer el valor actual de un puerto se llama read\_gpio\_pin(), es necesario pasarle como argumento el indicador del puerto del cual queremos conocer su valor. El valor del puerto es guardado en la estructura que almacena el estado de cada puerto para posteriores consultas, sin tener que volver a llamar a dicha función.
Las últimas dos funciones son contrapuestas, set\_gpio\_pin() y clear\_gpio\_pin(), éstas permiten poner el valor de un puerto específico en el valor lógico “1” o “0” respectivamente. Previo a establecer o borrar el valor del puerto ambas funciones verifican que la dirección del puerto sea de salida; como mecanismo de seguridad no es posible cambiar el valor de un puerto de entrada.
Por su parte la macro reset\_status\_gpio() permite borrar el estado de un puerto que ya no esté en uso.

\bigskip
\leftline{\bf{SC}}

Hoy en día la mayoría de los lectores de tarjetas de contacto tienen una interfaz USB para ser conectado en un PC en aplicaciones de escritorio. Para el uso de este tipo de lectores sobre Linux existe un controlador genérico llamado CCID.
Sin embargo las tarjetas de contacto no poseen un puerto USB sino un puerto serie para establecer la comunicación con algún dispositivo, es por esto que en la nueva generación de lectores siempre hay un  ASIC para lograr la interacción, por un lado con la tarjeta de contacto y por el otro la comunicación con el PC.
Como fue descrito en la sección de hardware, el lector de tarjetas de contacto tiene una interfaz serial pura para la transferencia de datos con las tarjetas. En base al diseño hardware elegido, las capas de software sobre las que se decidió trabajar son las que se detallan en la figura XXXX. 

\begin{figure}[H]
\centering
  \begin{center}
  \includegraphics[scale=.4]{Imagenes/SW_sc1.jpg} 
  \end{center}
  \caption{Capas de software de trabajo}\label{Fig:capas} 
\end{figure}

En una primera etapa y para simplificar el desarrollo y la depuración del software, la capas empleadas fueron las que se muestran en la figura XXXX.


\begin{figure}[H]
\centering
  \begin{center}
  \includegraphics[scale=.4]{Imagenes/SW_sc2.jpg} 
  \end{center}
  \caption{Capas de software en una primera etapa}\label{Fig:capas0} 
\end{figure}

\bigskip
\leftline{Descripción de las capas:}

\bigskip
\leftline{controlador}
El kernel es el encargado de manipular directamente los registros del puerto serial, las interrupciones que desde éste se generan y la ISR para atender las interrupciones.
La implementación del controlador del lector de tarjetas se basó en el controlador serial de Linux a través de su estructura “termios”. Esta estructura nos permite configurar todos los parámetros necesarios para la comunicación serial como ser, baud rate, cantidad de bits por byte, bit de paridad, bit de parada entre otros. Las funciones read y write permiten la lectura y escritura de los bytes de datos que son recibidos y transmitidos por el puerto serial.

\bigskip
\leftline{CT/API (Card Terminal / Application Programming Interface)}
Por encima del controlador serial se encuentra CT/API[ref], una interfaz definida por varias emprasas entre las que se incluye Telekom Alemania en la década de los noventa, que permite encapsular el controlador específico de cada lector de tarjetas, de manera que la aplicación final no se vea afectada al cambiar un lector por otro.
Esta interfaz de programación está formada tan solo por 3 funciones, CT\_init, CT\_data y CT\_close, que permiten la inicialización del lector, la transferencia de datos entre host/lector o host/tarjeta (host se refiere a la SBC o PC donde se encuentra conectado el lector de tarjetas de contacto) directamente y el cierre de la comunicación.
CT\_init se encarga del pasaje de parámetros a la capa del controlador, para la configuración del puerto de comunicación entre el host y el lector de tarjetas. Los parámetros en uso aquí son: la tasa de transferencia de datos, el número de bits por cada byte, el tipo de paridad empleado y el puerto serie a ser utilizado.
CT\_data es la función encargada de transferir comandos y datos hacia y desde la tarjeta o hacia y desde el lector (en caso que el mismo esté formado por un ASIC o microprocesador). La manera de diferenciar desde donde es enviado el dato, es a través de un parámetro pasado a esta función, y de forma análoga se determina el destino del mensaje. El protocolo usado para la transferencia de datos es T=0, orientado a bytes y del cual pueden conocerse más detalles en [handbook sc]
CT\_close es la contracara de  CT\_init, se encarga de cerrar la comunicación con el lector. Lo que hace básicamente es liberar el handle (puntero) asociado al puerto serial.
Para el caso en que los comandos y/o datos estén dirigidos hacia el lector, existe otra especificación, llamada CT/BCS (Card Terminal / Basic Command Set), donde se encuentran definidos una serie de comandos básicos para el manejo del lector. Estos comandos se numeran a continuación y se da una breve descripción, por más detalles referirse a la mencionada especificación [ref]:

RESET CT permite reiniciar el lector o tarjeta (en el caso de ser un lector mudo); de manera opcional puede devolver el ATR.

REQUEST ICC tiene como objeto devolver el ATR de la tarjeta una vez que la misma se encuentra ubicada en el zócalo del lector.

GET STATUS es empleado para conocer información sobre el lector o si la tarjeta está insertada y eléctricamente conectada en el lector. 

EJECT ICC genera la desactivación eléctrica de la tarjeta.

\bigskip
\leftline{IFDHandler}
El siguiente componente en este stack de capas es ifdhandler[ref], no es otra cosa que un conjunto de funciones formando una API, empleada por pcsclite para encapsular el manejo del hardware de  lectores cuyos fabricantes quieran cumplir con las especificaciones PC/SC[ref]. Una venaja importante de esta API es que le permite a pcsclite operar tanto con lectores de puerto serial como con lectores de puerto USB.
Esta capa de software podría usarse directamente sobre el controlador del lector, prescindiendo de CT/API, aunque se decidió mantenerla por motivos de simplicidad ya que sólo es necesario sustituir la capa de aplicación por las restantes capas superiores como se indica en las figuras anteriores.
En lo que sigue se enumeran algunas de las funciones de esta API y se describen brevemente. Por más detalles ver el manual ifdhandler [ref].
IFDHCreateChannel establece el canal de comunicación con el lector. Para conseguirlo usa un parámetro llamado Channel, que indica cual es el puerto serial a usar, por ejemplo para el caso de Linux /dev/ttySx (x es el número que corresponda).
IFDHCloseChannel implementa la acción opuesta a la función anterior, cerrando el canal de comunicación con el lector de tarjetas.
IFDHGetCapabilities permite obtener las capacidades específicas del lector o de la tarjeta insertada en el mismo.
IFDHPowerICC se encarga del control de las señales de alimentación y reset que el lector suministra a la tarjeta. Desempeña tres acciones posibles, encendido, reset y apagado de la tarjeta.
IFDHTransmitToICC se encarga de la transferencia de datos con la tarjeta a través de alguno de los protocolos disponibles, como ser T=0 o T=1.
IFDHICCPresence retorna el estado de la tarjeta insertada en el zócalo del lector.

\bigskip
\leftline{PCSCLite}
Por arriba de ifdhandler se encuentra la librería pcsclite, ésta contiente todas las funciones necesarias para establecer la comunicación con un lector y la tarjeta conectada a éste último. Para usar el controlador encapsulado por ifdhandler desde pcsclite es necesario seguir los pasos de configuración detallados en el apéndice.

\bigskip
\leftline{Aplicación final}
Por arriba de toda las capas descritas antes se encuentra la aplicación del prototipo, que hace uso de las funciones suministradas por pcsclite y donde se encuentran definidos los comnados APDU específicos con los que opera la tarjeta de contacto.


\bigskip
\leftline{\bf{RFID}}
A continuación se detallan los cambios introducidos en la librfid para el correcto funcionamiento del lector/escritor RFID utilizando librfid-tool.

\bigskip
Se analizó el código principal de la aplicación librfid-tool y se vio que dentro del directorio utils en el archivo common.c se encuentra la función reader\_init la cual busca un lector/escritor entre los soportados. Esta función no tenía implementada una búsqueda para dispositivos conectados por SPI. Por lo tanto se tuvieron que agregar algunas líneas a la función para que el funcionamiento fuera posible:

\begin{verbatim}
rh = rfid_reader_open("/dev/spidev3.0", RFID_READER_SPIDEV);
if (!rh) {
    fprintf(stderr, "No SPIDEV found\n");
    return -1;
}
\end{verbatim}

Este cambio permitió detectar la interfaz SPI (spidev3.0). Algo a tener en cuenta, es que como en Linux las interfaces están asociadas a un archivo, el hecho de abrir el archivo no implica que haya nada conectado en esa interfaz. Por esta razón, la búsqueda de un lector/escritor conectado por interfaz SPI se realiza en última instancia.

\bigskip
Otro cambio fundamental es en la frecuencia de trabajo del SPI, se modifica a 10MHz, ya que con la frecuencia establecida en la librfid (1MHz) el lector/escritor RFID no funciona correctamente. Toda la configuración de la comunicación por SPI se encuentra en rfid\_reader\_spidev.c que está en /src.
La función que se modificó es spidev\_open y el cambio se muestra a continuación:

\begin{verbatim}
tmp = 10e6; /* 10 MHz */
if (ioctl(spidev_fd, SPI_IOC_WR_MAX_SPEED_HZ, &tmp) < 0)
    goto out_rath;
\end{verbatim}

\subsection{Aplicación final}

Para el desarrollo de la aplicación RF$^{2}$, se decidió trabajar sobre los fuentes de la herramienta librfid-tool ya que maneja varias funciones de utilidad y es de ayuda a la hora de compilar para el armado de una aplicación completa. Se mantuvieron todas las opciones de la herramienta ya que son muy útiles y pueden ayudar en un futuro para establecer orígenes de fallas. No se modificó ninguna función de la aplicación original y cuando fue necesaria alguna modificación, se procedió a implementar una nueva.

\bigskip
Antes de seguir fue necesario entender el funcionamiento de las reglas de compilación creadas para la aplicación librfid-tool sobre librfid. En el directorio raíz, se encuentran los siguientes archivos importantes para el desarrollo del sistema: autogen.sh, configure.in, configure, Makefile.am, Makefile.flags.am, Makefile.in, Makefile.

\bigskip
A continuación se describe a grandes rasgos la utilidad de cada uno de estos archivos:

\bigskip
configure.in es el archivo de configuración con el cual se crea el archivo configure.
Makefile.am es el archivo que establece las reglas de compilación (orden, dependencias, etc) y es con el cual se crea el archivo Makefile.in. Este archivo se encuentra en todos los subdirectorios de la librfid, por lo que se crea un Makefile.in dentro de cada subdirectorio.
Makefile.flags.am es un archivo con banderas establecidas para el sistema. Este archivo está incluido en todos los Makefile.am de la aplicación.
autogen.sh es un script que se encarga de crear los archivos configure y Makefile.in, cada uno de ellos creados a partir de los archivos correspondientes antes mencionados.
Al ejecutar el archivo configure, este establece la configuración que necesita el Makefile.in para conocer las reglas de compilación que se van a usar y algunos parámetros extra. Luego de este proceso se generan los Makefile.
El archivo Makefile es el que conoce las reglas de compilación, las dependencias y las opciones elegidas para el desarrollo. Sabe incluso donde se encuentran los demás Makefile para ejecutarlos cuando sea necesario.

\bigskip
Cada subdirectorio tiene reglas locales de construcción de sus objetos y a su vez se ayudan mutuamente para lograr una aplicación más completa. Desde el Makefile principal se controla que todo funcione correctamente.
Para saber más sobre reglas de compilación ver JHHJHJHJH

\bigskip
La aplicación RF$^{2}$ utiliza otras bibliotecas además de librfid, además estas bibliotecas van a interactuar entre sí, lo que lleva a que aparezcan nuevas dependencias. Por lo tanto, es necesario modificar las reglas de compilación.

\bigskip
Se agregó en el raíz de la librfid el directorio rf2 en el cual se encuentran subdirectorios con los fuentes necesarios para la comunicación con el display, leds, buzzer, tarjeta de contacto y otros utilitarios, formando la estructura que se muestra en la figura \ref{est_RF2}. 


\begin{figure}[H]
\centering
  \begin{center}
  \includegraphics[scale=.5]{Imagenes/estructura_librfid.png} 
  \end{center}
  \caption{Estructura de árbol de aplicación RF$^{2}$}\label{est_RF2} 
\end{figure}

gpio incluye fuentes para el manejo de los GPIO.

lcd incluye fuentes para el manejo del display.

rf incluye fuentes con funciones extra de utilidad para el manejo del lector/escritor RFID diseñado.

sam incluye los fuentes para la comunicación con la tarjeta de contacto.

utiles incluye fuentes con funciones útiles de la aplicación.

\bigskip
Dentro de cada uno de estos directorios se encuentra un archivo Makefile con las reglas de compilación correspondiente a cada uno.

\bigskip
Las modificaciones a los archivos de construcción de la aplicación para que contemplen el agregado del directorio rf2 como la creación de nuevas reglas para la construcción dentro de éste, se detallan a continuación.

\bigskip
Antes que nada se decidió modificar los archivos Makefile.am para que sepan de la existencia del nuevo directorio. Se modifican estos archivos debido a que nunca son borrados. Por ejemplo, si los cambios se hacen sobre los Makefile.in, puede pasar que en elgún momento se regeneren borrando los cambios que se hicieron. Además, luego de entender la estructura y funcionamiento de estos archivos, es más fácil incluir una modificación en Makefile.am que en un Makefile.in o Makefile directamente.
Dentro del directorio utiles, se creó el archivo Variables\_Make que establece el valor de algunas variables específicas de la aplicación RF$^{2}$ como ser CC\_arm que hace referencia al gcc asociado con la herramienta de croscompilación para ARM. 

\bigskip
Makefile.flags.am:
Aquí se indicó que se agregue rf2 a la ruta de búsqueda de archivos encabezados.

\begin{verbatim}
INCLUDES = \$(all_includes) -I$(top_srcdir)/include 
-I\$(top_srcdir)/rf2
\end{verbatim}

Makefile.am en el raíz:

\begin{verbatim}
include rf2/utiles/Variables_Make
\end{verbatim}
Con esto se aseguró que el resto de los subdirectorios pertenecientes a la librfid, sepan los valores de las variables específicas de la aplicación RF$^{2}$.

Luego se incluye el directorio rf2 a la aplicación.
\begin{verbatim}
if ENABLE_SPIDEV 
SUBDIRS += rf2 
endif
\end{verbatim}

\bigskip
Makefile.am en utils:
Este es el cambio más dificil de entender y surge de la observación del Makefile.in generado cada vez (prueba y error). Como la aplicación RF$^{2}$ se basa en la modificación de los fuentes de la herramienta librfid-tool, se deben agregar todas las dependencias con archivos del nuevo subdirectorio rf2.

\begin{verbatim}
librfid_tool_SOURCES = librfid-tool.c librfid-tool.h 
common.c common.h ../rf2/gpio/gpio.c ../rf2/gpio/gpio.h 
../rf2/rf/rc632_utils.c ../rf2/rf/rc632_utils.h 
../rf2/lcd/lcd16x2.c ../rf2/lcd/lcd16x2.h ../rf2/sam/sam.c 
../rf2/sam/sam.h ../rf2/sam/sam_util.c ../rf2/sam/sam_util.h 
../rf2/utiles/utiles.c ../rf2/utiles/utiles.h
\end{verbatim}

Cada .h y .c se traduce en un .o al crear el Makefile.in.

\bigskip
Otro cambio necesario, ya que si no se incluye provoca errores de compilación es el siguiente.

\begin{verbatim}
mifare_tool_SOURCES = mifare-tool.c common.c 
../rf2/gpio/gpio.c ../rf2/rf/rc632_utils.c 
../rf2/lcd/lcd16x2.c ../rf2/sam/sam.c 
../rf2/sam/sam_util.c ../rf2/utiles/utiles.c
\end{verbatim}

Mifare-tool es otra herramienta de la librfid.

\bigskip
Luego se creó dentro del directorio rf2 un archivo Makefile que es el que ordena la construcción de todos los objetivos dentro de cada subdirectorio.

\begin{verbatim}
include utiles/Variables_Make 

all: gpio/gpio.o rf/rc632_utils.o lcd/lcd16x2.o 
sam/sam.o utiles/utiles.o 

gpio/gpio.o: 
	$(MAKE) -C gpio 

rf/rc632_utils.o: 
	$(MAKE) -C rf 

lcd/lcd16x2.o: 
	$(MAKE) -C lcd 

sam/sam.o: 
	$(MAKE) -C sam 
	 
utiles/utiles.o: 
	$(MAKE) -C utiles 

install: 

clean: 
	rm -f *.o 
	$(MAKE) -C gpio clean 
	$(MAKE) -C rf clean 
	$(MAKE) -C lcd clean 
	$(MAKE) -C sam clean 
	$(MAKE) -C utiles clean 

distclean: clean
\end{verbatim}

\bigskip
En este caso \$(MAKE) -C “directorio” indica que la regla de construcción del objetivo se encuentra en “directorio”.

\bigskip
Se agregó una nueva opción (n) en el main de la aplicación librfid-tool que llama a la función principal(). Ésta es la función principal de la aplicación RF$^{2}$. De este modo no se modifica el main original de librfid-tool y se dejan las opciones por defecto.
Se hizo uso de algunas funciones ya escritas y se crearon otras. En este punto no se modificó ninguna función de la aplicación original y cuando fue necesaria alguna modificación, se procedió a implementar una nueva función con las modificaciones previstas.
Luego viene la etapa de croscompilación de la aplicación para ser probada en la SBC.

\bigskip
Croscompilación de la aplicación final:
Para guardar el resultado de la croscompilación se creó un directorio work en el home del usuario.

\bigskip
\centerline{\$ ./autogen.sh}
Este paso es necesario siempre que se modifiquen los Makefile.am, en otro caso no. Se recuerda que este script genera los archivos Makefile.in y configure.

\bigskip
\centerline{\$ ./configure --enable-spidev --host=arm-angstrom-linux-gnueabi --prefix=/home/proyecto/work}
Se configura el sistema para lectores/escritores RFID con interfaz SPI, se indica la arquitectura de la SBC para la cual compilamos y se indica el directorio donde se instala la aplicación. Luego de hecho esto ya no es necesario volver a ejecutarlo ya que los Makefile ya quedan creados con esta configuración.

\centerline{\$ make clean \&\& make -j5 \&\& make install}
Se construye la aplicación.

Luego, en el directorio work se encuentran cuatro directorios creados:

\bigskip
bin: incluye los binarios construidos.

include: directorio con los .h necesarios para correr la aplicación.

lib: la biblioteca en sí.

share: nada de utilidad.

\bigskip
El paso siguiente es el de copiar estos directorios en el sistema de archivos de la SBC.

\bigskip
Copia de archivos a la SBC:

\bigskip
Para realizar la copia conviene comprimir el resultado de la croscompilación y luego enviarlo a la SBC.

\centerline{\$ tar -czf rf2.tar.gz bin include lib share}

\bigskip
Luego de enviado el archivo comprimido, es necesaria la copia de estos directorios en el sistema de archivos de la SBC.

\bigskip
Instalación en la SBC:

\centerline{\$ tar -xf rf2.tar.gz -C /usr}
Esto descomprime rf2.tar.gz bajo /usr.

\bigskip
Se ejecuta la aplicación RF$^{2}$:

\centerline{\$ librfid-tool -n}


%sección 6.5
\section{Ejecución de programa principal}

%sección 6.5.1
\subsection{Script para ejecución autónoma}

\part{Ensayos}
\chapter{Ensayos}

\section{SBC}
Las Hawkboard fabricadas entre el 1º de agosto y el 20 de octubre de 2010 fueron vendidas en el mercado con 
un error a nivel de hardware que no había sido constatado por el fabricante y que no fue reconocido por éste
hasta el mes de noviembre. La solución al problema fue liberada en la fecha 20 de diciembre de 2010 y constaba de sustituir en el circuito, los ferrites FB12 y FB13 por un puente de soldadura de estaño (el uso de jumper 0R fue probado sin obtener buenos resultados). Mayores detalles de la solución pueden encontrarse en el documento Hawkboard\_Press\_Release\_Solution.pdf[anexo].
El inconveniente mencionado antes evitaba que el sistema operativo Linux iniciara correctamente, generándose un mensaje de “kernel panic” indicando que el sistema operativo no podía ser ejecutado. Esto evitó que se pudieran probar las partes de hardware y software que se tenían desarrolladas hasta ese entonces, teniendo que recurrirse a mecanismos alternativos como el uso de un microprocesador rabbit para efectuar pruebas sobre el lector/escritor RFID.

\bigskip
\bigskip
{\bf{Pruebas sobre las interfaces}}

Testeo de GPIO: Para el testeo de los GPIO, se compiló y probó el archivo led.c (ver Anexo VI) el
cual cambia el valor del pin 13 del bloque de expansión cada un segundo. Si se coloca un led entre
este pin y la tierra, se puede ver como el led se prende y apaga.


Testeo de UART: Para el testeo de la interfaz serial UART, se compiló y probó el programa uart.c
(ver Anexo VI) que envía una serie de caracteres por uart\_tx y luego lee por uart\_rx. Para verificar
el correcto funcionamiento se debe cortocircuitar uart\_tx con uart\_rx.


Testeo de SPI: Para el testeo del SPI se consiguió un ejecutable (spidev\_test) que hace algo parecido
a lo que realiza el archivo de testeo de la uart. En este caso debemos ejecutar el archivo con los
parámetros correspondientes ./spidev\_test -D /dev/spidev3.0 (spidev3.0 porque se está usando spi3
con cs0). Cuando se ejecuta, deben aparecer en pantalla varias filas con “FF”. Si
cortocircuitamos el SIMO y el SOMI algunas filas deberían cambiar (no “FF”) y con esto queda
verificado el buen funcionamiento del SPI.


\section{VLT - Conversor de Voltajes}
No existieron problemas en este módulo, y dadas las características del circuito
no hay demasiados puntos de falla. Si fuera necesario verificar los valores de
tensión en el regulador de tensión, la tensión de entrada puede ser medida desde
el conector CONN\_14x2 y la de salida desde el conector CONN\_20x2, ver Figura 5.1.
Un detalle a tener en cuenta a la hora de medir los valores de tensión de las
señales que pasan a través de los conversores de nivel, cuando las mismas se 
encuentren en estado ocioso (estáticas), es que no debe hacerse con multímetros 
de mala calidad, o se obtendrán valores incorrectos durante la medición. Se 
recomienda para una correcta medición el empleo de osciloscopio con puntas x10. 
Como se mencionó antes no se tuvieron inconvenientes con este módulo, pero generó 
conflictos en el circuito conversor full a half duplex del lector de tarjetas de 
contacto que serán detallados más adelante.


\section{SCUI - Lector de tarjetas de contacto e Interfaz de usuario}

{\bf{Lector de tarjetas de contacto ISO7816}}

Las primeras pruebas realizadas sobre el lector de tarjetas de contacto se efectuaron sobre una
placa de circuito impreso de fabricación propia, conectándose el lector directamente sobre el 
conector de expansión de la Beagleboard. La intención de esta prueba era más que nada la de probar
el circuito conversor full a half duplex, transmitiendo una serie de bytes por el canal Tx y recibiendo
el eco mediante el canal Rx, cotejando que los bytes recibidos coincidieran con los transmitidos. 
El primer problema encontrado aquí estuvo asociado a una falla en uno de los transistores, el PNP 3906, 
que debió ser sustituido por encontrarse defectuoso.
El software usado aquí para efectuar las pruebas sobre el hardware se basa en un controlador serial 
desarrollado por el grupo de robótica del INCO, el cual fue mínimamente modificado ya que uno de los 
parámetros, CSIZE, en la configuración del puerto afectaba el número de bits que conforman un byte recibido. 
La línea de código que hacía referencia a este parámetro fue comentada ya que modificaba el valor del parámetro 
csN, con N=5 en lugar de N=8 (donde N es el número de bits que forman el byte). 
El cambio anterior permitió que los bytes recibidos en el canal Rx coinicidieran con los transmitidos en Tx, 
validando en una primera instancia el hardware conversor full a half duplex del lector de tarjetas.
El siguiente paso fue intercalar entre la Beagleboard y el lector de tarjetas de contacto, el conversor de niveles (VLT) para realizar las mismas pruebas que se datallaron antes, aunque en este caso los resultados no fueron alentadores ya que los bytes recibidos no coincidían con los transmitidos. Todo indicaba que el conversor de nivel afectaba el conversor full a half duplex. Luego de algunas pruebas más sobre el circuito, sin cambios favorables, se decidió consultar al foro de Texas Instruments (fabricante del integrado TXB0108). Desde el soporte técnico solicitaron se les enviara una imagen capturada con osciloscopio de las señales en el puerto serial para observar la forma de los pulsos. En la Figura “X” debajo se puede ver la deformación de los pulsos en la señal Rx (canal 1 del osciloscopio) cuando el circuito contaba con un valor de 500 Ohms en la resistencia R9 (ver Figura 5.3); la solución encontrada fue disminuir el valor de R9 y no aumentarlo como se había intentado anteriormente sin beneficio alguno. Al usar valores entre 90 Ohms y 180 Ohms para la resistencia R9, la forma de los pulsos recibidos en Rx (canal 1) fueron la copia de los pulsos transmitidos en Tx (canal 2), como puede verse en la Figura “Y” para un valor de 90 Ohms. 
Aquí puede verse el hilo de discusión en el foro: ${http://e2e.ti.com/support/interface/etc\_interface/f/391/t/114719.aspx}$.

Una vez superados los obstáculos anteriores fue posible probar el circuito completo del lector, incluyendo 
la tarjeta de contacto en su zócalo correspondiente. El software usado en tal fin se basa en un controlador
serial, implementado por David Corcoran (uno de los desarrolladores de pcsclite), el cual debió ser modificado para usarse en el lector de tarjetas de contacto del prototipo ${RF^{2}}$. Una de las mayores dificultades encontrada en esta etapa fue el hallar los parámetros adecuados de inicialización del puerto serial, que debe cumplir con las opciones 8E2 (8 bits por byte, bit de paridad par, y dos bits de parada) para operar con las tarjetas de contacto compatibles
con la norma ISO7816; sin embargo la opciones adecuadas elegidas en la configuración del puerto serial fueron 8E1.
Adicionalmente al problema de encontrar las opciones correctas mencionadas antes, fue que los bytes de datos
recibidos como el ATR de la tarjeta no coincidían en su totalidad con los valores esperados(leídos con un lector
Omnikey 3121 y la herramienta pcsc\_scan de pcsclite), sólo algunos bytes y algunos nibbles bajos eran correctos.
Esta diferencia estuvo asociada a la frecuencia usada para alimentar la señal de reloj de la tarjeta de contacto;
se usaron frecuencias de 4 Mhz y 5 Mhz que si bien podrían usarse según se indica en [handbook SC] para
los parámetros especificados en el ATR de las tarjetas empleadas, estos valores no fueron adecuados según ya indicamos,
teniendo que usar en su lugar un oscilador de frecuencia 3,579545 Mhz, valor que no se conseguió cuando se realizó
la primer compra de componentes.
Una dificultad adicional tuvo que ser sorteada en este módulo de hardware, el diseño del PCB que se envió a fabricar
tenía un error, las pistas de datos de Rx y Tx estaban intercambiadas. El diseño tuvo que ser corregido y se 
envió a fabricar un nuevo PCB.


\bigskip
\bigskip
{\bf{Interfaz de usuario}}


No existieron mayores inconvenientes con la interfaz para el usuario, sí fue 
necesaria la corrección en el valor de una resistencia en el circuito que calibra 
el contraste del LCD, ya que los caracteres se observaban muy tenues.

\bigskip
\bigskip
Al momento de probar el display imprimía caracteres extraños, salvo cuando se enviaban mensajes conteniendo una única palabra. Se probó cambiando los mensajes a desplegar en el mismo, y el problema persistía, pero se llegó a la conclusión de que era provocado por los espacios (“ ”) puesto que cuando se envió un mensaje omitiéndolos fue deplegado en forma correcta. Luego simplemente se 
modificó el código fuente, para que cada vez que recibiera un caracter espacio, enviara al display el código ASCII correspondiente solucionando el problema.

Cuando se comenzaron a imprimir los saldos de las tarjetas, volvió a imprimir caracteres extraños, esta vez el problema eran los caracteres “0”. La solución más rápida encontrada fue imprimir “O” cada vez que llegara un caracter “0”, por lo que se modificó el código para que así sea.


\section{Lector/Escritor RFID}
Como se mencionó antes, cuando se tuvo pronto el PCB del lector/escritor RFID aún la SBC no estaba pronta para poder conectarlo y hacer las pruebas necesarias. Por lo tanto, como se tenía un kit de desarrollo rabbit 4000, se conectó el lector/escritor al rabbit y se comenzó el testeo de hardware. Para esto, previamente, hubo que configurar los pines del rabbit de modo de obtener un puerto SPI a través del cual pudiera comunicarse con el CL RC632 soldado en el PCB. El compilador Dynamic C ya incluye una librería SPI que se utilizó para facilitar el trabajo.

Todo el software implementado para pruebas fue realizado a partir de lo estudiado en la librería 
librfid (http://openmrtd.org/projects/librfid/).

Una vez que el SPI estuvo operativo, se comenzó con pruebas muy básicas como enviar un comando específico al CL RC632 y esperar la respuesta correcta según la hoja de datos del CL RC632. En principio no se recibía lo esperado, pero el problema era que el puerto no estaba configurado en forma correcta, luego de reconfigurarlo en varias oportunidades (testeándolo con un led) se llegó a la configuración correcta y el CL RC632 comenzó a responder lo esperado. Fue entonces que se probaron otros comandos y surgieron otros errores. Se diseñó entonces una función para enviar comandos a registros, previo comprender el funcionamiento de los mismos (8 páginas, con 8 registros cada una). 
Se logró leer y escribir la memoria fifo, también resetearla (borrarla por completo y reseteando todas sus banderas). Se probó leer la eeprom, leyendo información conocida. Luego se escribió en la misma y se leyó lo escrito como prueba de que funciona. 
Se le dio el formato necesario a la clave de una tarjeta, para almacenar en memoria, y se cargó en la memoria correspondiente. 
Por último se intentó dialogar con una tarjeta, sin éxito. Aunque se verificó que el lector/escritor RFID modulaba, puesto que se pudo observar en el osciloscopio (se puede ver foto!). 
Se implementó un detector de campo magnético (una bobina con un led) (se puede sacar una foto :) ) para comprobar la presencia de la portadora en los alrededores de la antena. Gracias al mismo, se observó que no había campo generado por el inductor del circuito impreso (no había portadora), por lo que se desconectó dicho inductor y se realizó otro que se soldó al PCB y es el que logró modular.

Para observar la forma de las señales se usó un osciloscopio, verificando que las mismas fueran correctas y cumplieran con las condiciones indicadas en el manual del CL RC632 .

Luego, estuvo pronta la SBC. Se conectó entonces el PCB por SPI y se comenzaron las pruebas ya en el hardware del prototipo ${RF^{2}}$. Si bien con el rabbit se pudo verificar parte del hardware, la antena propiamente dicha no estaba verificada, puesto que no se logró leer tarjeta alguna.
Se revieron entonces todas las pruebas efectuadas desde el rabbit, ahora en la SBC. Luego de varios días de trabajo sin lograr leer tarjetas, se midió (por sugerencia de Juan Pablo Oliver) en el Instituto de Ingeniería Eléctrica la impedancia de la antena. La misma no era la que se obtuvo en los cálculos teóricos, estaba muy por encima (ver figura). Se procedió entonces, siguiendo las recomendaciones de las notas de diseño (APP NOTES, ver anexo supongo), a modificar el circuito de matcheo. Se sustituyeron y agregaron condensadores hasta lograr mejorar la frecuencia de resonancia del inductor.
Luego, mejorado el circuito de matcheo, se re-fabricó el PCB. El mismo seguía sin lograr leer tarjetas. Entonces se volvió a revisar la configuración del puerto SPI en la Beagleboard, y luego de varias pruebas con el osciloscopio (Ed y Dan si quieren detallen porque no sé bien que hicieron, yo estaba con el rabbit, dijo Daniel:  ./spidev\_test -D /dev/spidev3.0 frecuencia) se llegó a la conclusión de que el problema era de software. Se encontró entonces que el pin de reset no se estaba poniendo a nivel bajo como se debía hacer. Luego de corregir el error, se comenzó a leer parte de las tarjetas, aunque aparecían varios errores. Fue entonces que se pensó que también existía un problema de velocidad de recepción en los datos, se varió la frecuencia del puerto SPI hasta que se logró leer completamente cada una de las tarjetas.
Por último se probó escribir las tarjetas con un programa de testeo existente en la librería librfid, sin problemas. Luego, cuando se implementó el software específico del prototipo ${RF^{2}}$ surgieron problemas pero se debían a que el acceso a la tarjeta para escribir se realiza con la claveB y se estaba usando la claveA.
\part{Compras}
\chapter{Compras}

La compra de componentes electrónicos se efectuó casi exclusivamente en empresas 
de Estados Unidos, entre las que figuran Newark, Digikey y Special Computing. Esto
se debió a la dificultad que existe en plaza para conseguir los insumos
necesarios para la fabricación del prototipo RF$^{2}$. 

Entre los pocos elementos comprados en plaza se encuentran las placas de cirucito
impreso que fueron fabricadas por la empresa Eneka. 

En lo que sigue se mustran las listas de componentes separadas por el módulo de 
hardware al que pertenecen. Los precios que se detallan son en origen,
en dólares americanos y no incluyen impuestos, excepto los PCB que incluyen IVA.

\newpage
\section{SBC}
\begin{longtable}{|l|l|c|c|c|}
\hline
\multicolumn{1}{|c|}{\textbf{Componente}} & \multicolumn{1}{c|}{\textbf{Descripción}} & \textbf{Cantidad} & \textbf{Precio x1} & \textbf{Total} \\ \hline
SBC & Beagleboard  RevC4 & 1 & 125 & 125 \\ \hline
Memoria SD & 4GB SDHC Class 6 SD Card & 1 & 15 & 15 \\ \hline
Cable serial DB9 nulo & DB9F Null Modem (RS-232) (6-ft) & 1 & 4 & 4 \\ \hline
Cable conversor usb–serial & USB to DB9M RS-232 (PL-2302) & 1 & 10 & 10 \\ \hline
Cable USB & USB Mini-A to USB A Female, OTG & 1 & 9 & 9 \\ \hline
Cable USB & USB Mini-B Male to USB A Male & 1 & 5 & 5 \\ \hline
Fuente  & 5VDC/2,5A & 1 & 10 & 10 \\ \hline
 &  & \multicolumn{1}{l|}{} & \multicolumn{1}{l|}{} & 178 \\ \hline
\caption{Single board computer y lista de accesorios.}
\label{}
\end{longtable}


\section{PCBs}
\begin{longtable}{|l|l|c|c|c|}
\hline
\multicolumn{1}{|c|}{\textbf{Componente}} & \multicolumn{1}{c|}{\textbf{Descripción}} & \textbf{Cantidad} & \textbf{Precio x1} & \textbf{Total} \\ \hline
VLT & Interfaz entre SBC y SCUI & 1 & 28 & 28 \\ \hline
SCUI & Interfaz de usuario y lector de tarjetas ISO7816 & 1 & 53 & 53 \\ \hline
RWD RFID & Lector/Escritor de tarjetas RFID ISO14443 & 1 & 64 & 64 \\ \hline
 &  & \multicolumn{1}{l|}{} & \multicolumn{1}{l|}{} & 145 \\ \hline
\caption{Lista de placas de circuito impreso.}
\label{}
\end{longtable}

\newpage
\section{VLT}
\begin{longtable}{|l|p{3cm}|p{2cm}|c|c|c|c|}
\hline
\multicolumn{1}{|c|}{\textbf{Componente}} & \multicolumn{1}{c|}{\textbf{Descripción}} & \textbf{ Footprint} & \textbf{Valor} & \textbf{Cantidad} & \textbf{Precio x1} & \textbf{Total} \\ \hline
C1 & Polarized Capacitor (Tantal) & 6032[2312] & 10uF, 25V & 1 & 1,09 & 1,09 \\ \hline
C2 & Polarized Capacitor (Tantal) & 6032[2312] & 100uF, 6V3 & 1 & 1,16 & 1,16 \\ \hline
U4 & Regulador LM1117-3.3 & SOT-223 & 3.3V, 800mA & 1 & 1,1 & 1,1 \\ \hline
U1, U2, U3 & Voltage Level Translator & TSSOP20 & - & 3 & 2,24 & 6,72 \\ \hline
P1 & RECEPTACLE, 28WAY, 2ROW & SMD  Pitch 2,54 & 28 pines & 1 & 4,19 & 4,19 \\ \hline
P2 & RECEPTACLE, 40WAY, 2ROW & SMD  Pitch 2,54 & 40 pines & 1 & 4,36 & 4,36 \\ \hline
P1b & HEADER, 28WAY, 2ROW & T H Pitch 2,54 & 28 pines & 1 & 2 & 2 \\ \hline
P2b & HEADER, 40WAY, 2ROW & T H Pitch 2,54 & 40 pines & 1 & 1,94 & 1,94 \\ \hline
 &  & \multicolumn{1}{l|}{} & \multicolumn{1}{l|}{} & \multicolumn{1}{l|}{} & \multicolumn{1}{l|}{} & 22,56 \\ \hline
\caption{Lista de componentes de la placa de circuito impreso VLT.}
\label{}
\end{longtable}

\newpage
\section{SCUI}
\begin{longtable}{|l|p{2.5cm}|p{2cm}|p{2cm}|c|c|c|}
\hline
\multicolumn{1}{|c|}{\textbf{Componente}} & \multicolumn{1}{c|}{\textbf{Descripción}} & \textbf{ Footprint} & \textbf{Valor} & \textbf{Cantidad} & \textbf{Precio x1} & \textbf{Total} \\ \hline
R9 & Resistor 100W 1/4W 1\%  & 3216[1206] & 100W 1/4W  1\% & 1 & 0,07 & 0,07 \\ \hline
R10, R13 & Resistor 100KW 1/4W 5\%  & 3216[1206] & 100KW 1/4W   5\% & 2 & 0,09 & 0,18 \\ \hline
R11, R12, R14 & Resistor 10KW 1/4W 5\%  & 3216[1206] & 10KW 1/4W   5\% & 3 & 0,08 & 0,24 \\ \hline
Q2 & TRANSISTOR, NPN, 300MHZ & SOT23 & MMBT3904 & 1 & 0,125 & 0,125 \\ \hline
Q3 & TRANSISTOR, PNP, 250MHZ & SOT23 & MMBT3906 & 1 & 0,18 & 0,18 \\ \hline
J2 & SIM socket (6 contacts) & SMD & - & 1 & 1,25 & 1,25 \\ \hline
JP3, JP4 & HEADER, 1ROW, 3WAY & T H Pitch 2,54 & 3 pines & 2 & 0,11 & 0,22 \\ \hline
X1 & Oscillator 3.579545MHz & SMD & 3.579545 Mhz & 1 & 5,25 & 5,25 \\ \hline
ESD1 & Anti ESD & SOT323 & 6V / 150W & 1 & 0,45 & 0,45 \\ \hline
 &  & \multicolumn{1}{l|}{} & \multicolumn{1}{l|}{} & \multicolumn{1}{l|}{} & \multicolumn{1}{l|}{} & 7,965 \\ \hline
\caption{Lista de componentes del lector de tarjetas de contacto, SC.}
\label{}
\end{longtable}

%LCD
\begin{longtable}{|l|p{3cm}|p{2cm}|p{2cm}|c|c|c|}
\hline
\multicolumn{1}{|c|}{\textbf{Componente}} & \multicolumn{1}{c|}{\textbf{Descripción}} & \textbf{ Footprint} & \textbf{Valor} & \textbf{Cantidad} & \textbf{Precio x1} & \textbf{Total} \\ \hline
R1 & Resistor 4K7    1/10W     1\% & 1608[0603] & 4,7KW  1/10W   1\% & 1 & 0,05 & 0,05 \\ \hline
R2, R8 & Resistor 3R3    1/10W     1\% & 1608[0603] & 3,3W    1/10W   1\% & 2 & 0,09 & 0,18 \\ \hline
R3, R4, R5 & Resistor 680R  1/10W     1\% & 1609[0603] & 680W   1/10W   1\% & 3 & 0,05 & 0,15 \\ \hline
R6, R7 & Resistor 10K    1/10W     1\% & 1608[0603] & 10KW  1/10W   1\% & 2 & 0,05 & 0,1 \\ \hline
RV1 & Preset 15K        1/10W  25\% & SMD & 15KW   1/10W  25\% & 1 & 0,71 & 0,71 \\ \hline
Q1 & TRANSISTOR, NPN, 300MHZ & SOT23 & MMBT3904 & 1 & 0,125 & 0,125 \\ \hline
S1 & LCD MODULE 16X2 CHARACTER & Pitch 2,54 & - & 1 & 10,85 & 10,85 \\ \hline
CONN1 & HEADER FEMALE 16POS.1" TIN & Through Hole & 16 pines & 1 & 1,25 & 1,25 \\ \hline
CONN2 & HEADER, 1ROW, 16WAY & T H Pitch 2,54 & 16 pines & 1 & 0,155 & 0,155 \\ \hline
LED1 & Led green 5mm & Through Hole & 1,9V,  2mA & 1 & 0,11 & 0,11 \\ \hline
LED2 & Led red 5mm & Through Hole & 1,9V,  2mA & 1 & 0,1 & 0,1 \\ \hline
LED3 & Led yellow 5mm & Through Hole & 2,4V,  2mA & 1 & 0,13 & 0,13 \\ \hline
BUZZ1 & Buzzer & Through Hole & 3~20Vdc, 3~16mA & 1 & 5,31 & 5,31 \\ \hline
 &  & \multicolumn{1}{l|}{} & \multicolumn{1}{l|}{} & \multicolumn{1}{l|}{} & \multicolumn{1}{l|}{} & 19,17 \\ \hline
\caption{Lista de componentes para la interfaz de usuario, LCD.}
\label{}
\end{longtable}


\section{Lector/Escritor RFID}
\begin{longtable}{|l|p{2cm}|p{2cm}|p{2.5cm}|c|c|c|}
\hline
\multicolumn{1}{|c|}{\textbf{Componente}} & \multicolumn{1}{c|}{\textbf{Descripción}} & \textbf{ Footprint} & \textbf{Valor} & \textbf{Cantidad} & \textbf{Precio x1} & \textbf{Total} \\ \hline
C1, C2 & Capacitor & 1608[0603] & 10pF,   Ceramic NPO, 2\% & 2 & 0,135 & 0,27 \\ \hline
C3, C4 & Capacitor & 1609[0603] & 100pF, Ceramic NPO, 2\% & 2 & 0,194 & 0,388 \\ \hline
C5, C6, C7, C8 & Capacitor & 1608[0603] &  NC & 4 & - & 0 \\ \hline
R1,  R2 & Resistor & 1608[0603] & 0W,  1/10W,  1\% & 2 & 0,015 & 0,03 \\ \hline
 &  &  &  &  &  & 0,688 \\ \hline
\caption{Lista de componentes del lector/escritor RFID. Antena.}
\label{}
\end{longtable}



\begin{longtable}{|p{1cm}|p{2.5cm}|p{2cm}|p{2.5cm}|c|c|c|}
\hline
\multicolumn{1}{|c|}{\textbf{Componente}} & \multicolumn{1}{c|}{\textbf{Descripción}} & \textbf{ Footprint} & \textbf{Valor} & \textbf{Cantidad} & \textbf{Precio x1} & \textbf{Total} \\ \hline
C10 & Capacitor & 1610[0603] & 10pF, Ceramic NPO, 2\% & 1 & 0,135 & 0,135 \\ \hline
C1, C2 & Capacitor & 1608[0603] & 15pF, Ceramic NPO, 5\% & 2 & 0,03 & 0,06 \\ \hline
C12, C13 & Capacitor & 1608[0603] & 56pF, Ceramic NPO, 2\% & 2 & 0,194 & 0,388 \\ \hline
C14, C15 & Capacitor & 1608[0603] & 68pF, Ceramic NPO, 1\% & 2 & 0,197 & 0,394 \\ \hline
C9 & Capacitor & 1609[0603] & 100pF, Ceramic NPO,  2\% & 1 & 0,194 & 0,194 \\ \hline
C16 & Capacitor & 1608[0603] & 1nF, Ceramic NPO, 10\% & 1 & 0,08 & 0,08 \\ \hline
C4, C5, C7, C8, C11, C17 & Capacitor & 1608[0603] & 100nF, Ceramic X7R, 10\% & 6 & 0,074 & 0,444 \\ \hline
C3, C6, C18 & Capacitor & 1608[0603] & 10uF, Ceramic X5R, 20\% & 3 & 0,195 & 0,585 \\ \hline
L1, L2, L3, L6 & Inductor & 2012[0805] & 22nH, 700mA, 5\% & 4 & 0,454 & 1,816 \\ \hline
L4, L5 & Inductor & 3225[1210] & 1uH, 400mA, 5\% & 2 & 0,29 & 0,58 \\ \hline
R3 & Resistor & 1608[0603] & 50W, 1/10W   1\% & 1 & 0,268 & 0,268 \\ \hline
R2 & Resistor & 1608[0603] & 820W, 1/10W   5\% & 1 & 0,027 & 0,027 \\ \hline
R1 & Resistor & 1608[0603] & 2,2KW, 1/5W   1\% & 1 & 0,08 & 0,08 \\ \hline
U1 & Reader ISO14443 & SO32 & CL RC632 & 1 & 14,22 & 14,22 \\ \hline
U2 & Crystal Oscillator, HC49 US SMD & 49USMXL & 13.56MHz, 10pF & 1 & 0,98 & 0,98 \\ \hline
U3 & Operational Amplifier (up to 7.5V) & SOT23-5 & OPA354 & 1 & 2,8 & 2,8 \\ \hline
CONN1, CONN2 & U.FL-R Connector & U.FL-R-SMT & - & 2 & 1,76 & 3,52 \\ \hline
J1 & HEADER, 10WAY, 2ROW & T H Pitch 2,54 & 10 pines & 1 & 0,389 & 0,389 \\ \hline
J1b & RECEPTACLE, 10WAY, 2ROW & SMD  Pitch 2,54 & 10 pines & 1 & 2,27 & 2,27 \\ \hline
 &  & \multicolumn{1}{l|}{} & \multicolumn{1}{l|}{} & \multicolumn{1}{l|}{} & \multicolumn{1}{l|}{} & 29,23 \\ \hline
\caption{Lista de componentes del lector/escritor RFID. Módulo digital + filtro EMC.}
\label{}
\end{longtable}





% Anexos:
\part{Anexos}
\appendix
\chapter{Tarjetas “inteligentes” (Smart Cards)}

Una tarjeta inteligente (smart card), o tarjeta con circuito integrado (ICC, de su sigla en inglés), es cualquier tarjeta del tamaño de un bolsillo con circuitos integrados que permiten la ejecución de cierta lógica programada. 
Aunque existe un diverso rango de aplicaciones, hay dos categorías principales de ICC. Las tarjetas de memoria contienen sólo componentes de memoria no volátil y posiblemente alguna lógica de seguridad. Las tarjetas microprocesadoras contienen memoria y microprocesador.
La percepción estándar de una smart card es una tarjeta microprocesadora de las dimensiones de una tarjeta de crédito (o más pequeña, como por ejemplo, tarjetas SIM para GSM) con varias propiedades especiales (ej. un procesador criptográfico seguro, sistema de archivos seguro, características legibles por humanos) y es capaz de proveer servicios de seguridad (ej.     confidencialidad de la información en la memoria).
Las tarjetas no contienen baterías; la energía es suministrada por los lectores de tarjetas.

\section{Clasificaciones}

\subsection{Tipos de tarjetas según su capacidad}

\bigskip
Según las capacidades de su chip, las tarjetas más habituales son:

\begin{itemize}
\item Memoria: tarjetas que únicamente son un contenedor de ficheros pero que no albergan aplicaciones ejecutables. Por ejemplo, MIFARE. Éstas se usan generalmente en aplicaciones de identificación y control de acceso sin altos requisitos de seguridad. 
\item Microprocesadas: tarjetas con una estructura análoga a la de una computadora (procesador, memoria volátil, memoria persistente). Éstas albergan ficheros y aplicaciones y suelen usarse para identificación y pago con monederos electrónicos. 
\item Criptográficas: tarjetas microprocesadas avanzadas en las que hay módulos hardware para la ejecución de algoritmos usados en cifrados y firmas digitales. En estas tarjetas se puede almacenar de forma segura un certificado digital (y su clave privada) y firmar documentos o autenticarse con la tarjeta sin que el certificado salga de la misma, ya que es el procesador de la propia tarjeta el que realiza la firma.
\end{itemize}



\subsection{Tipos de tarjetas según la estructura de su sistema operativo}

\begin{itemize}
\item Tarjetas de memoria. Tarjetas que únicamente son un contenedor de datos pero que no albergan aplicaciones ejecutables. Disponen de un sistema operativo limitado con una serie de comandos básicos de lectura y escritura de las distintas secciones de memoria y pueden tener capacidades de seguridad para proteger el acceso a determinadas zonas de memoria. 
\item Basadas en sistemas de ficheros, aplicaciones y comandos. Estas tarjetas disponen del equivalente a un sistema de ficheros compatible con el estándar ISO/IEC 7816 parte 4 y un sistema operativo en el que se incrustan una o más aplicaciones (durante el proceso de fabricación) que exponen una serie de comandos que se pueden invocar a través de API de programación. 
\item Java Card. Tarjeta capaz de ejecutar mini-aplicaciones Java. En este tipo de tarjetas el sistema operativo es una pequeña máquina virtual Java (JVM) y en ellas se pueden cargar dinámicamente aplicaciones desarrolladas específicamente para este entorno. 
\end{itemize}


\subsection{Tipos de tarjetas según el formato (tamaño)}

En el estándar ISO/IEC 7816 parte 1 se definen los siguientes tamaños para tarjetas inteligentes:

\begin{itemize}
\item ID 000: el de las tarjetas SIM usadas para teléfonos móviles GSM. También acostumbran a tener este formato las tarjetas SAM (Security Access Module) utilizadas para la autenticación criptográfica mutua de tarjeta y terminal. 
\item ID 00: un tamaño intermedio poco utilizado comercialmente. 
\item ID 1: el más habitual, tamaño tarjeta de crédito. 
\item ID 1/000: permite remover la tarjeta ID 000 desde la tarjeta ID 1 sin herramientas de corte.
\end{itemize}


\subsection{Tipos de tarjetas según la interfaz}

\bigskip
\leftline{\bf{Tarjeta inteligente de contacto}}

Estas tarjetas disponen de contactos metálicos visibles y debidamente estandarizados (parte 2 de la ISO/IEC 7816). Estas tarjetas, por tanto, deben ser insertadas en una ranura de un lector para poder operar con ellas. A través de estos contactos el lector alimenta eléctricamente a la tarjeta y transmite los datos oportunos para operar con ella conforme al estándar.

\begin{figure}[H]
\centering
  \begin{center}
  \includegraphics[scale=.3]{Imagenes/sc1.jpg} 
  \end{center}
  \caption{Tarjeta de contacto}\label{Fig:HW} 
\end{figure}

La serie de estándares ISO/IEC 7816 e ISO/IEC 7810 definen:

\begin{itemize}
\item La forma física (parte 1). 
\item La posición de las formas de los conectores eléctricos (parte 2). 
\item Las características eléctricas (parte 3). 
\item Los protocolos de comunicación (parte 3).
\item El formato de los comandos (ADPU's) enviados a la tarjeta y las respuestas retornadas por la misma (parte 3).
\item La dureza de la tarjeta.
\item La funcionalidad.
\end{itemize}


\bigskip
\leftline{bf{Tarjetas Inteligentes sin Contacto}}

El segundo tipo es la tarjeta inteligente sin contacto, RFID,  en el cual el chip se comunica con el lector de tarjetas mediante acoplamiento magnético a una tasa de transferencia de 106 a 848 Kbits/s.
El estándar de comunicación de tarjetas inteligentes sin contacto es el ISO/IEC 14443. Define dos tipos de tarjetas sin contacto (A y B), permitidos para distancias de comunicación de hasta 10cm. Las más abundantes son las tarjetas de la familia MIFARE de Philips, las cuales representan a la ISO/IEC 14443-A.
Las tarjetas inteligentes sin contacto son una evolución de la tecnología usada desde hace años por los RFID (identificación por radio frecuencia), añadiéndoles dispositivos que los chip RFID no suelen incluir, como memoria de escritura o microcontroladores.

\bigskip
Tarjetas híbridas y duales

\bigskip
Una tarjeta híbrida es una tarjeta sin contacto (contactless) a la cual se le agrega un segundo chip de contacto. Ambos chips pueden ser chips microprocesadores o simples chips de memoria. El chip sin contacto es generalmente usado en aplicaciones que requieren transacciones rápidas. Por ejemplo el transporte, mientras que el chip de contacto es generalmente utilizado en aplicaciones que requieren de alta seguridad como las bancarias.


\bigskip
Seguridad

La seguridad es una de las propiedades más importantes de las tarjetas inteligentes y se aplica a múltiples niveles y con distintos mecanismos. Cada fichero lleva asociadas unas condiciones de acceso y deben ser satisfechas antes de ejecutar un comando sobre ese fichero.

\bigskip
En el momento de personalización de la tarjeta (durante su fabricación) se puede indicar que mecanismos de seguridad se aplican a los ficheros. Normalmente se definirán:

\begin{itemize}
\item Ficheros de acceso libre.
\item Ficheros protegidos por claves: Pueden definirse varias claves con distintos propósitos. Normalmente se definen claves para proteger la escritura de algunos ficheros y claves específicas para los comandos de consumo y carga de las aplicaciones de monedero electrónico. De ese modo la aplicación que intente ejecutar comandos sobre ficheros protegidos tendrá que negociar previamente con la tarjeta la clave oportuna. 
\item Ficheros protegidos por PIN: El PIN es un número secreto que va almacenado en un fichero protegido y que es solicitado al usuario para acceder a este tipo de ficheros protegidos. Cuando el usuario lo introduce y el programa se lo pasa a la operación que va a abrir el fichero en cuestión, el sistema valida que el PIN sea correcto para dar acceso al fichero. 
\end{itemize}

La negociación de claves se realiza habitualmente apoyándose en un Módulo SAM, que no deja de ser otra tarjeta inteligente en formato ID-000 alojada en un lector interno propio dentro de la carcasa del lector principal o del TPV(Terminal de Punto de Venta) y que contiene aplicaciones criptográficas que permiten negociar las claves oportunas con la tarjeta inteligente del usuario. Operando de este modo se está autenticando el lector, la tarjeta y el módulo SAM involucrados en cada operación.


\bigskip
Programación de aplicaciones para los sistemas en los que se utiliza la tarjeta

\bigskip
Existen varias API de programación estandarizadas para comunicarse con los lectores de tarjetas inteligentes desde un computador. Las principales son:

\begin{itemize}
\item PC/SC (Personal Computer/Smart Card). El proyecto MUSCLE proporciona una implementación casi completa de esta especificación para los sistemas operativos GNU Linux-UNIX. 
\item OCF (OpenCard Framework), especificado por el grupo de empresas OpenCard. Este entorno intenta proporcionar un diseño orientado a objetos fácilmente extensible y modular. El consorcio OpenCard publica el API y proporciona una implementación de referencia en Java. Existe un adaptador para que OCF trabaje sobre PC/SC. 
\end{itemize}


En ambos casos, el modelo de programación que utilizan las tarjetas inteligentes está basado en protocolos de petición-respuesta. La tarjeta (su software) expone una serie de comandos que pueden ser invocados. Estos comandos interactúan con los ficheros que subyacen a cada aplicación de la tarjeta y proporcionan un resultado. Desde el terminal se invocan estos comandos a través de cualquiera de las API antes descritas componiendo un APDU (Application Protocol Data Unit - comandos con parámetros) que son enviados a la tarjeta para que ésta responda.


\section{ISO 14443}

ISO 14443 es un estándar internacional relacionado con las tarjetas de identificación electrónicas, en especial las smart cards, gestionado conjuntamente por la Organización Internacional de Normalización (ISO) y la Comisión Electrotécnica Internacional (IEC).
Este estándar define una tarjeta de proximidad utilizada para identificación y pagos que por lo general utiliza el formato de tarjeta de crédito definida por ISO 7816 - ID 1 (aunque otros formatos son posibles).
El sistema RFID utiliza un lector con un microcontrolador o ASIC y una antena que opera a 13,56MHz (frecuencia RFID). El lector mantiene a su alrededor un campo electromagnético de modo que al acercarse una tarjeta al campo, ésta se alimenta eléctricamente de esta energía inducida y puede establecerse la comunicación lector-tarjeta.

\begin{figure}[H]
\centering
  \begin{center}
  \includegraphics[scale=.2]{Imagenes/sc2.jpg} 
  \end{center}
  \caption{}\label{Fig:HW} 
\end{figure}

El estándar ISO 14443 consta de cuatro partes y se describen dos tipos de tarjetas: tipo A y tipo B. Las principales diferencias entre estos tipos está en los métodos de modulación, codificación de los planes (parte2) y el protocolo de inicialización de los procedimientos (parte3). Las tarjetas de ambos tipos (A y B) utilizan el mismo protocolo de alto nivel (llamado T=CL) que se describe en la parte4. El protocolo T=CL especifica los bloques de datos y los mecanismos de intercambio:

\begin{itemize}
\item[1.] Bloque de datos de encadenamiento.
\item[2.] Tiempo de espera de extensión.
\item[3.] Múltiple activación.
\end{itemize}

Las tarjetas Mifare cumplen con las partes 1, 2 y 3 de tipo A de la especificación ISO/IEC 14443.


\section{Mifare}

Mifare es la tecnología de smart card sin contacto más ampliamente usada en el mundo. Es equivalente a las 3 primeras partes de la norma ISO 14443 Tipo A. La distancia típica de lectura es de hasta 10 cm, depende de la potencia del lector y factores del entorno, existiendo lectores de mayor y menor alcance.
La tecnología Mifare es económica y rápida, razón por la cual es la más usada a nivel mundial hoy día.


\subsection{Operación}

Las tarjetas Mifare son tarjetas de memoria protegida. Están divididas en sectores que a su vez son subdivididos en bloques y poseen mecanismos de seguridad para el control de acceso. Su capacidad de cómputo no permite realizar operaciones criptográficas o de autenticación mutua de alto nivel, estando principalmente destinadas a monederos electrónicos simples, control de acceso, tarjetas de identidad corporativas, tarjetas de transporte urbano o para ticketing.
Cada sector se divide en cuatro bloques, de los cuales tres pueden contener información del usuario, y el cuarto, llamado trailer, contiene elementos de seguridad. La información es almacenada sin un  formato pre establecido, y se puede modificar con comandos simples de lectura y escritura. Mifare provee un formato especial llamado “bloque de valor”(value block); los bloques que tienen información guardada en este formato se comportan de una forma diferente, incluyendo operaciones de incremento y descuento.
Los sectores utilizan dos claves de acceso llamadas 'A' y 'B'. Estas claves se almacenan en el cuarto bloque junto con los permisos de acceso a cada uno de los tres bloques que son parte del mismo sector. Estos permisos pueden ser: lectura, escritura, descuento o incremento (para bloques de valor).

\begin{figure}[H]
\centering
  \begin{center}
  \includegraphics[scale=.6]{Imagenes/sc3.jpg} 
  \end{center}
  \caption{Mifare Classic de 4K}\label{Fig:HW} 
\end{figure}

Una vez que se acerca la tarjeta a un lector, ésta se activa e inicia un proceso de intercambio con el lector para establecer una comunicación cifrada. Este proceso es igual con todas las tarjetas y está diseñado para proveer protección del canal(evitando que se espíe), y no para autenticar la tarjeta o el lector.
Previo a establecer un canal cifrado la tarjeta envía un código de identificación, mediante el algoritmo de anticolisión, que usualmente es el número de serie de la tarjeta, aunque la norma ISO 14443 indica que este número puede ser aleatorio. Con este número de identificación el lector está en condiciones de realizar cualquier operación en la tarjeta, previa autenticación con las claves de acceso en los respectivos sectores.
Se debe destacar que un sistema con claves diversificadas facilita el fortalecimiento de la seguridad, apoyada por una base de datos que pueda monitorear los aumentos de los saldos y demás estrategias operativas y finalmente la autenticación remota por SAM, no todos los sistemas poseen esto.

\bigskip
Variantes

\begin{itemize}
\item Mifare Classic. Son fundamentalmente de los dispositivos de almacenamiento de memoria. Existen tarjetas de 1Kb y de 4Kb. La Mifare Standard de 1KB ofrece unos 768 bytes de almacenamiento de datos, dividida en 16 sectores. La Mifare Standard de 4k ofrece 3 KB dividido en 64 sectores. 
\item Mifare Ultralight. Es semejante a la classic, pero sólo tiene 512 bits de memoria (es decir 64 octetos), sin seguridad. Esta tarjeta es muy barata así que se utiliza a menudo de forma desechable. 
\item Mifare T = CL. Bajo esta denominación se encuadran las tarjetas Mifare ProX y SmartMX. Son tarjetas con microprocesador que incorporan un sistema operativo de tarjeta (Card Operating System - COS) y aplicaciones desarrolladas específicamente para ser ejecutadas en la tarjeta. Estas tarjetas son capaces de ejecutar operaciones complejas de forma rápida y segura, igual que las tarjetas con contactos ISO 7816. 
\item Mifare DESFire. Esta tarjeta es una versión especial de Philips SmartMX. Se vende con un software de propósito general incorporado (el sistema operativo DESFire), que ofrece más o menos las mismas funciones que Mifare Standard (4kB de almacenamiento de datos dividido en 16 bloques), pero con una mayor flexibilidad, una mayor seguridad (triple DES), y con mayor rapidez (protocolo T=CL). 
\item Mifare DESFire EV1. Es la primera evolución de Mifare DESFire, compatible con la versión anterior, pero aún más segura, alcanzando la certificación EAL 4.
\end{itemize}
\chapter{Lector/Escritor RFID}

\section{Reglas y Parámetros de Diseño de una Antena RF}

Pasos para el diseño de la antena RF:

\begin{itemize}
\item[A)] Diseñar el inductor, medir su inductancia L y resistencia R (o factor de calidad Q).
\item[B)] Calcular los capacitores para el circuito resonante, que forman junto con el inductor.
\item[C)] Sintonizar el circuito resonante junto con el filtro pasa bajo a la impedancia requerida.
\item[D)] Conectar el circuito resonante a la salida del integrado(TX1 y TX2), verificar la corriente ITVDD y si es necesario sintonizar los componentes para un desempeño óptimo.
\item[E)] Verificar y ajustar el factor de calidad Q.
\item[F)] Verificar y ajustar el circuito receptor.
\end{itemize}

\subsection{Diseño del inductor}

Se recomienda usar antenas cuyo inductor tenga forma circular o cuadrado. El valor exacto del inductor es difícil de calcular pero puede ser aproximado por la siguiente ecuación:

\centerline{$ L_{1}[nH]=2 \cdot l_{1} \cdot (ln(\frac{l_{1}}{D_{1}})-K) \cdot N^{1,8}_1$}

\leftline{donde:}
	
\leftline{$l_{1}$ ............... Longitud de una vuelta del conductor (en cm).}

\leftline{$D_{1}$ ............. Diámetro del conductor o ancho del conductor del PCB.}

\leftline{$K$ .............. Factor de forma ($K = 1,07$ antena circular y $K = 1,47$ antena cuadrada).}

\leftline{$N_{1}$ ............. Número de vueltas.}

La antena debe ser simétrica, donde el punto central puede estar conectado a GND. Si esto es así, se sugiere mantener este punto lo más cercano posible al conector de la antena.
El radio de la antena deberá ser $r > 5cm$ para una sola vuelta de cada uno de los inductores simétricos $L_{a}$ y $L_{b}$, o $r < 5cm$ para dos vueltas.
El blindaje de campo eléctrico debe ser conectado a GND.

Los valores de inductancia y resistencia del inductor, si se siguen las reglas de diseño, se encuentran entre: 

$ L = 300nH {...} 2\mu H $

$ R_{coil} =0,5\Omega {...} 5\Omega $


\bigskip
\begin{itshape}
\leftline{Resistor externo}
\end{itshape}


En serie con el inductor se agrega un resistor externo. El valor del mismo se encuentra mediante las ecuaciones:

\centerline{$ Q = \frac{w L}{R_{coil}} \rightarrow R_{coil} = \frac{w L}{Q}$}

\leftline{y  $ R = 2 \cdot R_{s} + R_{coil} = R_{Sa} + R_{Sb} + R_{coil} $ , donde $R$ es la resistencia total, $R_{Sa}$ y $R_{Sb}$ son cada uno de los resistores externos simétricos.}

Definiendo $Q$ entre 20 y 30, se pueden hallar los resistores externos:

$R_{Sa} = R_{Sb} = \frac{1}{2} \cdot (R - R_{coil}) = \frac{wL}{2Q} - \frac{R_{coil}}{2}$ con $w = 2 \pi \cdot 13,56MHz$


Dejando de lado la influencia de todos los otros componentes en el factor $Q$, este cálculo sólo da una estimación del valor de $R_{S}$, pero esta estimación es necesaria para el cálculo de los capacitores del circuito de resonancia. 

\bigskip
\bigskip
	En muchas aplicaciones prácticas es posible observar que se prescinde del valor de 	resistencia externa $R_{S}$, teniendo en cuenta sólo el valor de resistencia del inductor $R_{coil}$.	

\subsection{Capacitores del circuito resonante}
Gracias a la simertría del circuito es posible simplificar los calculos operando sólo con la mitad del circuito, por tanto los valores del inductor $L$, la resistencia $R$ (incluyendo el resistor externo) y el valor de impedancia requerida de la antena $Z_{ant}$, empleados en los siguientes cálculos son la mitad del valor correspondiente a la totalidad del circuito. 
Aplicando entonces la suma de impedancias a la mitad del cirucito y sabiendo que el resultado tiene que ser real e igual a $Z_{ant}$, es posible hallar los valores del capacitor paralelo $C_{2}$ y el capacitor serie $C_{1}$ mediante las siguientes igualdades:

$C_{2a} = C_{2b} = \frac{L}{\omega^{2}L^{2} + R^{2}} - \frac{R}{( \omega^{2}L^{2} + R^{2}) \omega \sqrt{\frac{Z_{a}}{( \frac{\omega^{2}L^{2} + R^{2}}{R} - Z_{a})}}}$

$C_{1a} = C_{1b} = \frac{1}{\omega \sqrt{Z_{a} \cdot ( \frac{\omega^{2}L^{2} + R^{2}}{R} - Z_{a})}}$

$Z_{ant} = 250 \Omega$


El valor de $Z = 2 \cdot Z_{ant} = 500 \Omega$ podría ser incrementado hasta  $Z = 2 \cdot Z_{ant} = 800 \Omega$ para incrementar la potencia de salida, pero el límite de corriente de salida desde el integrado no debe ser excedido.


\subsection{Sintonizar el circuito resonante}
El circuito todo (incluyendo el filtro pasa bajos) tiene que ser adaptado a una impedancia de aproximadamente 40 $\Omega$ entre TX1 y TX2 (500 $\Omega$ si no tenemos en cuenta el filtro). Donde los valores propuestos para los componentes del filtro son:

\bigskip
\leftline{$L_{0} = 1 \mu H$ (e.g. TDK NL322522T-1R0J)}
\leftline{$C_{01} = 68pF$ (Ceramic NP0, tolerance $\leq \pm 2\%$)}
\leftline{$C_{02} = 56pF$ (Ceramic NP0, tolerance $\leq \pm 2\%$)}

\bigskip
Con estos valores, la frecuencia de resonancia del filtro se encuentra centrada en $14,4MHz (13,56MHz + 847,5KHz)$. Esto mejora la performance en dos formas:


\begin{itemize}
\item[a)] Incrementa la relación señal a ruido de la señal recibida.
\item[b)] Decrementa el sobretiro de los pulsos transmitidos, mejorando la calidad de la señal transmitida.
\end{itemize}

El procedimiento para sintonizar el circuito, es el siguiente:

Los materiales necesarios son:
\begin{itemize}
\item[-] Generador de señales ($13,56MHz$).
\item[-] Osciloscopio con puntas de prueba.
\item[-] Resistor de referencia ($40 \Omega$).
\end{itemize}

\bigskip
Se debe conectar las puntas del osciloscopio a la salida del generador y en paralelo un resistor de referencia de $40 \Omega$ ($500 \Omega$ en caso de sintonizar el circuito sin conectar el filtro pasa bajos).

\bigskip
\begin{itshape}
\leftline{Calibración}
\end{itshape}

Se genera una señal sinusoidal de frecuencia $13,56MHz$ y de amplitud entre $2V$ y $5V$.
El osciloscopio se configura para observar las figuras de Lissajous, con la escala del eje X dos veces la del eje Y.
Se calibra el capacitor, $C_{cal}$, de la punta de prueba del osciloscopio hasta que la figura de Lissajous sea un segmento de recta, inclinado $45º$.

\bigskip
\begin{itshape}
\leftline{Sintonizado}
\end{itshape}

Luego de la calibración se sustituye el resistor de referencia por la antena y se sintoniza la misma variando los capacitores $C_{1}$ y $C_{2}$, hasta que se obtenga una figura como la obtenida en el caso anterior. En ese momento la antena se encuentra sintonizada.


En caso de contar con un analizador de redes, el método anterior puede ser evitado, ya que es posible sintonizar el circuito buscando que la impedancia en el diagrama de Smith se ubique sobre el eje real al alcanzar la frecuencia de trabajo, en este caso $13,56 Mhz$.


\subsection{Valor de ITVDD}

El integrado de la familia Micore, entrega a la salida una señal cuadrada, con valor de pico a pico $U_{TxAC} = 2,5V_{pp}$ centrada en el valor de continua  $U_{TxDC} = 2,5V$, con una frecuencia $f_{0} = 13,56MHz$ y un máximo de salida de corriente: 


\centerline{$I_{TVDD} \leq 150mA$}


Esto significa que la salida TX oscila entre $0V$ y $5V$. TX1 y TX2 usualmente están desfasados $180º$, dependiendo de la configuración del bit 3 (TX2Inv) del registro Tx-Control (ver hoja de datos del integrado RC632).


\subsection{Factor de calidad Q}

El factor de calidad $Q$ está directamente asociado con la forma de los pulsos modulados, ésto puede ser usado para verificar el valor del factor.

\bigskip
Un osciloscopio con al menos $50MHz$ de ancho de banda puede ser usado para observar la forma de los pulsos; donde los canales son conectados de la siguiente forma:

\bigskip
CH1:   La punta de prueba conectada en este canal forma un loop con su línea de tierra para generar el acoplamiento necesario al estar próxima a la antena.

\bigskip
CH2:   Este canal es conectado a la salida del pin 4 (MFout) de integrado y es usado como canal de disparo.
\bigskip

El registro MFoutSelect (26h) es configurado con los valores:
\begin{itemize}	
\item[]	“2” para que la señal sea modulada con código Miller.
\item[]	“3” para flujo de datos serial (sin código Miller).
\item[]	Ver hoja de datos del integrado RC632 por más detalles.	
\end{itemize}	

Es recomendado verificar que la forma de los pulsos cumpla con lo establecido en la norma ISO14443. La figura debajo muestra como son estos pulsos.
Para garantizar que la antena se encuentre bien sintonizada y el factor $Q$ sea el correcto, debe verificarse que:


\begin{itemize}
\item[i.] La señal caiga debajo del $5\%$ de su valor máximo (sin tener en cuenta el sobretiro).
\item[ii.] El tiempo $t2$ debe estar limitado entre:  $0,7\mu s < t2 < 1,4\mu s$.
\end{itemize}

Si $t2 < 0,7\mu s$, el factor $Q$ es muy alto (mayor que 35), por lo tanto la resistencia externa $R_{ext}$ debe ser incrementada.

\bigskip
Si  $t2 > 1,4\mu s$ el factor $Q$ es muy bajo, la distancia de operación no será cumplida y por lo tanto $R_{ext}$ debe ser decrementada.

\bigskip
La tabla siguiente muestra la duración de los pulsos en $\mu s$ de acuerdo a la norma ISO 14443.

\begin{longtable}{|l|c|c|c|c|r|}
\hline
\multicolumn{1}{|c|}{\textbf{Pulses length}} & \textbf{t1} & \textbf{t2 min} & \textbf{t3 max} & \textbf{t4 max} \\ \hline

%Pulses length 				t1          t2 min	          t3 max          t4 max 
T1 MAX 			 &          3.0    &       0.7    &            1.0 &                0.4 \\ \hline
T1 MIN 			 &          2.0    &       0.7    &            1.0 &                0.4 \\ \hline
\caption{Duración de los pulsos en $\mu s$ - ISO 14443}
\label{}
\end{longtable}

\begin{figure}[H]
\centering
  \begin{center}
  \includegraphics[scale=.4]{Imagenes/anexo1.png} 
  \end{center}
  \caption{Forma de pulso acorde a la norma ISO 14443}\label{Fig:HW} 
\end{figure}


\subsection{Circuito receptor}

Cuando ya se han tomado en cuenta todos los cuidados en el diseño del transmisor, el circuito receptor debe ser conectado y ajustado.
Los valores de los componentes sugeridos en el circuito receptor son los siguientes:

\bigskip
	$C_{3} = 1nF$ 			(Ceramic NP0, tolerance $\leq \pm 10\%$) 
	
	$C_{4} = 100nF$ 			(Ceramic X7R, tolerance $\leq \pm 10\%$) 
	
	$R_{1} = 470 \Omega {...} 4.7 k \Omega $ 
	
	$R_{2} = 820 \Omega $
	
\bigskip
Dos reglas deben ser tenidas en cuenta para este circuito:

\begin{itemize}
\item[i.] El nivel de tensión de continua, DC, en la entrada Rx tiene que ser mantenido a $V_{mid}$ (por eso es necesario $R_{2}$ y $C_{4}$, ver figura debajo).
\item[ii.] El nivel de tensión de alterna, AC, en la entrada Rx debe ser mantenido entre los siguientes límites: $1,5V_{pp} < V_{Rx} < 3,0V_{pp}$.
\end{itemize}	

\bigskip
Si $V_{Rx} > 3,0V_{pp}$,  $R_{1}$ debe ser incrementada.

\bigskip
Si $V_{Rx} < 1,5V_{pp}$,  $R_{1}$ debe ser decrementada.

\bigskip
El voltaje a la entrada Rx debe ser verificado con y sin presencia de una tarjeta entre los límites máximo y mínimo de distancia de operación.

\bigskip
\begin{itshape}
El valor límite  $V_{Rx}= 3,0V_{pp}$ no debe ser excedido, un valor mayor puede causar fallos en la recepción.
\end{itshape}

\bigskip
\bigskip
\leftline{\bf{Otros puntos a tener en cuenta}}

\bigskip
\leftline{PCB}

\bigskip
La parte más crítica de todo el circuito analógico es el directamente conectado al integrado, o sea el filtro pasa bajos y la conexión de TVDD a la fuente de alimentación.
Entonces, por un lado un filtro puede ser usado para la conexión a la fuente de alimentación.
Por otro lado el diseño del filtro a la salida del integrado, formado por $L_{0}$ y $C_{0}$, debe ser considerado con mucho cuidado. \begin{itshape} El área y la distancia del filtro al integrado deben ser mantenidas lo más pequeñas posibles. Es recomendado además un plano de tierra.	
\end{itshape}

\bigskip
La figura siguiente muestra un esquemático del diseño de una antena:

\begin{figure}[H]
\centering
  \begin{center}
  \includegraphics[scale=.4]{Imagenes/anexo2.png} 
  \end{center}
  \caption{Esquema de una antena, identificando sus principales secciones}\label{Fig:HW} 
\end{figure}


\bigskip
\leftline{Filtro de entrada de alimentación}

\bigskip
Aunque no sería necesario, un filtro puede ser conectado a la entrada TVDD para mejorar los siguientes puntos:

\begin{itemize}
\item[a)] suprimir ruido llegado desde la fuente de alimentación.
\item[b)] suprimir armónicos provenientes desde el transmisor.
\end{itemize}

Filtros idénticos pueden ser ubicados en las entradas AVDD y DVDD.

\bigskip
\leftline{Blindaje}

\bigskip
El blindaje eléctrico absorbe el campo eléctrico generado por la antena. Para construir un blindaje, es recomendable usar un PCB de al menos 4 capas, donde el loop del blindaje se encuentra en las 2 capas externas. Este loop no debe ser cerrado y debe estar conectado en su punto central al sistema de tierra mediante una vía. Los extremos de la bobina deben ser ruteados próximos entre sí para evitar inductancias adicionales.  La figura siguiente da una idea de como debe ser un blindaje: 


\begin{figure}[H]
\centering
  \begin{center}
  \includegraphics[scale=.3]{Imagenes/anexo3.png} 
  \end{center}
  \caption{Blindaje de una antena en un diseño de 4 capas}\label{Fig:HW} 
\end{figure}


%capítulo 6
\chapter{Software}

%sección 6.1
\section{Introducción}
Se debe destacar que todo el desarrollo de software se basa exclusivamente en herramientas de software libre. La distribución Linux elegida para el sistema embebido se llama Angström. Ésta distribución es muy usada en aplicaciones que usan una Beagleboard y cuenta con una gran cantidad de bibliotecas implementadas en lenguaje C, que permiten una gran escalabilidad a la hora de incorporar nuevos periféricos en la aplicación.

%sección 6.2
\section{Arquitectura de Software}
%sección 6.2.1
\subsection{Descripción}
Un sistema linux se compone de diferentes partes que interactúan entre sí, formando capas ordenadas con distintos grados de abstracción respecto al hardware. Esto lo podemos apreciar en la figura \ref{Fig:SW} donde se muestra a grandes rasgos el sistema implementado. 

\begin{figure}[H]
\centering
  \begin{center}
  \includegraphics[scale=.4]{Imagenes/SW.jpg} 
  \end{center}
  \caption{Sistema RF${^{2}}$}\label{Fig:SW} 
\end{figure}

El bootloader es la parte del sistema más primitiva y su función es la de cargar el
kernel en memoria RAM para su ejecución. En general el bootloader en sistemas embebidos
es una aplicación que se divide en dos etapas, la primera etapa es fuertemente dependiente
del CPU con que cuenta la placa y su función es buscar en particiones activas para luego cargar 
en memoria RAM la segunda etapa del bootloader. Esta segunda etapa se encarga de descomprimir
en memoria RAM la imagen comprimida del kernel para luego ser ejecutado y que éste tome el
control del sistema.
El kernel se encarga a grandes rasgos de habilitar interrupciones, configurar la memoria y montar un sistema de archivos primitivo que permite a su vez cargar los módulos necesarios para la interfaz con periféricos. Luego se monta el verdadero sistema de archivos (fileSystem). En este nuevo sistema de archivos es donde se instalarán diferentes programas y bibliotecas para una correcta ejecución de nuestra aplicación.
En funcionamiento toda la comunicación con periféricos se realiza a través del kernel que es la parte más cercana al hardware.
Cada vez que se ejecuta una aplicación, ésta hace uso de las bibliotecas para poder comunicarse con el kernel, y éste se encarga de la comunicación con los periféricos. Las bibliotecas pueden ser nativas como es el caso de la biblioteca de lenguaje C o desarrolladas para que nuestra aplicación funcione correctamente.


%sección 6.2.2
\subsection{Sistema Operativo}
La Beagleboard al arrancar tiene la posibilidad de buscar el bootloader en NAND o en dispositivos extraíbles tales como memorias USB o memorias SD, lo mismo sucede con el kernel. Para nuestro sistema, elegimos un arranque a través de una memoria SD ya que es más fácil de manipular.

En la figura \ref{Fig:SD} se puede ver como queda distribuída la memoria SD con las distintas partes
que conforman el sistema operativo. 

\begin{figure}[H]
\centering
  \begin{center}
  \includegraphics[scale=.4]{Imagenes/sd.jpg} 
  \end{center}
  \caption{Memoria SD}\label{Fig:SD} 
\end{figure}

En la memoria SD se pueden distinguir dos particiones, una en formato FAT32 y otra
en formato ext3. La partición en FAT32 es llamada “de arranque” y es donde se encuentra 
el bootloader (MLO, u-boot.bin) y la imagen comprimida del kernel (uImage). 
La partición en ext3 es donde se encuentra el sistema de archivos (fileSystem).

El MLO es el equivalente al bootloader de la primera etapa; en general ya viene precargado en la memoria NAND de la Beagleboard. Es posible generarlo o incluso bajar una versión ya compilada desde la web de Angström. Como característica principal tiene la capacidad de buscar el u-boot.bin en dispositivos extraíbles como memorias SD o USB.

El u-boot.bin es equivalente al bootloader de la segunda etapa. Al igual que el MLO, es posible generarlo o incluso bajarlo de la web de Angström. En nuestro sistema fue necesario generarlo ya que configura el bloque de expansión de la Beagleboard.

El uImage es el kernel del sistema. Fue necesario generarlo ya que se debieron modificar sus fuentes para que queden habilitadas las interfaces de comunicación con los dispositivos periféricos.

El fileSystem es el correspondiente a una distribución linux llamada Angström. Se pueden llegar a precargar distintos programas y bibliotecas dependiendo de la forma en que lo generemos.
Angström es una distribución linux diseñada específicamente para sistemas embebidos desarrollados
para SBCs como la usada para este prototipo. Esto lo hace más eficiente que otros sistemas operativos. La elección de esta distribución se debió a que es de los más recomendados y utilizados en la documentación y foros de Beagleboard.

%sección 6.2.3
\subsection{Bibliotecas}
librfid es una biblioteca de software libre para manejo de lectores/escritores RFID. Implementa, en el dispositivo lector/escritor, el stack de protocolos ISO 14443A, ISO 14443B, ISO 15693, Mifare Ultralight y Mifare Classic.
Entre los lectores soportados están OpenPCD y algunos modelos Omnikey, estos lectores tienen  una interfaz de conexión USB. Además la librfib tiene soporte para cualquier otro lector con comunicación directa con el CL RC632 mediante la interfaz SPI y es por esta razón que se tuvo en cuenta.

\bigskip
Existe una herramienta desarrollada que ayuda a entender el funcionamiento de la biblioteca, librfid-tool.

%sección 6.3
\section{Herramientas utilizadas en el desarrollo del sistema}

%sección 6.3.1
\subsection{Introducción}
Para el desarrollo de sistemas, existe una gran variedad de herramientas útiles, algunas de software libre y otras privativas. El hecho de tener tantas opciones disponibles, a pesar de ser una ventaja algunas veces dificulta la elección de las herramientas correctas.

Para la elección de las herramientas se tomó como primer criterio de decisión el hecho de que sean libres así como las experiencias de otras personas que ya han transitado caminos comunes, consultando y participando en foros activos.

A continuación se detallan las herramientas utilizadas para el desarrollo del sistema. 
Primero se da una descripción de las herramientas elegidas y luego se comentan otras que se probaron con igual o peor resultado que las herramientas elegidas en última instancia.

%sección 6.3.2
\subsection{Generación de MLO, u-boot.bin y uImage}
No fue necesario generar el MLO debido a su simpleza, puesto que el binario precompilado realiza bien su función.

El u-boot.bin y el uImage fueron generados con la herramienta de desarrollo y compilación OpenEmbedded-Bitbake que es una fusión de dos herramientas: OpenEmbedded (herramienta para construcción y mantenimiento de distribuciones) y Bitbake (herramienta de compilación similar al Make que automatiza la construcción de ejecutables entre otros). OpenEmbedded utiliza Bitbake para su objetivo. Es una herramienta muy potente y difícil de aprender al principio. Luego de entendido su principio de funcionamiento se hace muy simple su uso, para lo que es necesario tener acceso a una buena conexión a internet.
Con esta herramienta también se pueden generar el MLO y el filesystem, aunque se prefirió utilizar otras herramientas por sobre ésta. 
Su instalación, configuración, estructura y uso se pueden ver en el apéndice \ref{anx_sw_oe}.

%sección 6.3.3
\subsection{Generación de FS}
Para la generación del fileSystem de Angström, se utilizó la herramienta web Narcissus.
Esta herramienta permite seleccionar entre diferentes dispositivos entre los cuales está Beagleboard, los programas que se quieran instalar, el formato de la imágen seleccionada e incluso se puede generar un kit de desarrollo (SDK) para el host. Debido a la facildad de uso y a los buenos resultados obtenidos, se decidió utilizar esta opción por sobre la del filesystem generado por la herramienta OpenEmbedded-Bitbake.

%sección 6.3.4
\subsection{Croscompilación}
Para la croscompilación se utilizó el SDK generado por Narcissus y la herramienta Make para generar los archivos necesarios. La instalación del SDK se encuentra en el APÉNDICE.

%sección 6.3.5
\subsection{Depuración de código}
Para la depuración, se utilizó la herramienta GDB del proyecto GNU. 
Al momento de compilar, es necesario agregar la opción -g para que la aplicación pueda ser depurada. Esta opción agrega información en el código de la aplicación.
La interfaz del GDB es por consola, aunque existen algunos programas que utilizan GDB y además ofrecen una interfaz gráfica (DDD[referencia]).
Algunos de los comandos útiles y sus usos más comunes son: 

\bigskip
breakpoint: para colocar un breakpoint. En general se lo llama seguido del nombre de una función de la aplicación.

print: seguido del nombre de una variable, muestra el contenido de la variable durante el proceso de depuración. Si la variable es local a alguna función, el valor de la variable se pierde al salir de la función.

next o “n”: sirve para ir línea a línea en modalidad step-over (sin entrar a las funciones).

step o “s”: sirve para ir línea a línea en modalidad step-into (entrando a las funciones).

backtrace o “bt”: despliega el stack de llamadas a funciones, sirve para saber por donde se pasó y donde estamos.

\bigskip
Sin olvidarnos del hecho de que la aplicación RF$^{2}$ está diseñada para una arquitectura distinta a la del PC de desarrollo, para el depurado de la aplicación existen dos alternativas. 

La primera es lo que se podría llamar depuración local, esto es, instalar GDB en la Beagleboard y depurar la aplicación en ésta. Para saber lo que sucede es necesario acceder de forma remota a ésta desde el PC de desarrollo. 

La segunda opción es la depuración remota. La depuración remota consiste en realizar la depuración de la aplicación desde el PC de desarrollo. Para esto, es necesario instalar GDBServer en la Beagleboard y tener instalado el GDB específico de la Beagleboard en el PC de desarrollo. Luego se establece una conexión que puede ser serial o ethernet entre la Beagleboard y el PC de desarrollo. Por más detalles sobre la configuración referirse al apéndice.

\bigskip
La primera opción no es posible para sistemas embebidos chicos en los cuales no se puede instalar GDB, aunque éste no es el caso de la Beagleboard. 
Se tienen más y mejores herramientas en el PC de desarrollo, por ejemplo programas con interfaz gráfica que ayudan a entender mejor lo que está pasando. Por algunas de estas razones, se prefiere el uso de la depuración remota. 

\bigskip
Se utilizaron indistintamente tanto la primera opción como la segunda.

%sección 6.3.6
\subsection{Bibliotecas}
librfid-tool es una herramienta de uso por línea de comandos que da acceso de bajo nivel RFID utilizando los lectores soportados por la librfid.

\bigskip
A continuación se detallan: la forma de llamar a la aplicación y algunas de las opciones soportadas.

\bigskip
librfid-tool -[opción]

\bigskip
Dentro de las opciones:

s: realiza una búsqueda de tarjetas RFID hasta encontrar una.

S: loop infinito con la opción -s, muestra información sobre la tarjeta RFID encontrada en cada paso.

p: especifica el protocolo RFID a utilizar, entre las opciones se encuentran tcl, mifare-classic y  mifare-ultralight.

l: especifica el protocolo de capa 2 a utilizar, entre las opciones se encuentran ISO14443a, ISO14443b y ISO15693.

h: ayuda.

\bigskip
Ejemplos de uso recomendados:

\bigskip
\$ librfid-tool -p mifare-classic

Si encuentra una tarjeta con protocolo Mifare-classic, devuelve la lectura completa de los bloques de memoria de la tarjeta.

\bigskip
\$ librfid-tool -S

Devuelve el UID y el protocolo soportado por la tarjeta.

%sección 6.4
\section{Desarrollo}

%sección 6.4.1
\subsection{MLO}

%sección 6.4.2
\subsection{Multiplexado de pines}
El microprocesador OMAP3530 tiene muchos pines con distintas interfaces entre las
que se cuentan puertos UART, SPI, GPIO, etc., pero no todos son accecibles desde la BeagleBoard. 
Para poder acceder a algunos de estos puertos del microprocesador, existe en la placa de la
BeagleBoard un bloque de expansión de 28 pines.

\bigskip
Por defecto en el bloque de expansión no se encuentran las señales que se quieren. Esto 
lleva a que se tenga que modificar el estado inicial de los pines. 
Existen dos formas de modificar los pines de modo de tener las señales que se precisan. Una de ellas es modificar el bootloader, la otra es modificar el kernel. Esto implica cambios
en los archivos fuentes y posterior compilación que genere los nuevos binarios u-boot o uImage. 

\bigskip
Para la modificación de las señales disponibles en el bloque de expansión se decidió modificar el u-boot ya que la modificación por u-boot es más intuitiva y por experiencia se sabe que lo que más se actualiza y/o modifica es el kernel. 

%sección 6.4.3
\subsection{u-boot}
Como se mencionó anteriormente en el u-boot se realiza la configuración de los pines del bloque de expansión de la Beagleboard. “hacer referencia a tabla en algún lado”
Cada pin del bloque de expansión tiene varias funcionalidades asociadas, y la configuración de una 
funcionalidad depende de un multiplexado modificable a nivel de software. Esto es, dependiendo del “modo de pin” elegido, la función que se obtiene en dicho pin.
Para que los cambios hechos en el u-boot tengan el efecto esperado al arrancar el sistema, es necesario que en la configuración del kernel esté la opción CONFIG\_OMAP\_MUX=no, lo que imposibilita al kernel de realizar este mismo cambio. Esta opción no está activada por defecto en ninguna versión actual del kernel. 

\bigskip
Antes que pueda ser modificado el estado de los pines del bloque de expansión,
es necesario obtener los archivos fuentes con los cuales se genera el archivo binario u-boot.

\bigskip
Nota: Puede que cuando se realiza la instalación de OpenEmbedded-Bitbake (apéndice), se genere un directorio relacionado con u-boot en /stuff/build/tmp/work/beagleboard-angstrom-linux-gnueabi/, si esto es así, no es necesario volver a obtener los fuentes.

\bigskip
\leftline{Se configura el bitbake para poder utilizarlo:}

\centerline{\$ export BBPATH=/stuff/build:/stuff/openembedded}

\centerline{\$ export PATH=/stuff/bitbake/bin:\$PATH}

\bigskip
\leftline{Se obtienen los fuentes:}

\centerline{\$ cd /stuff/build}

\centerline{\$ bitbake -f -c clean -b ../openembedded/recipes/u-boot/u-boot\_git.bb}

\centerline{\$ bitbake -f -c compile -b ../openembedded/recipes/u-boot/u-boot\_git.bb}

\bigskip
Los fuentes se encuentran en 

/stuff/build/tmp/work/beagleboard-angstrom-linux-gnueabi/u-boot.../git/.

\bigskip
En los fuentes del u-boot dentro de board/ti/beagle/ se encuentra el archivo beagle.h que es donde se establece la configuración de los pines del bloque de expansión de la Beagleboard.

Si se abre este archivo se ven líneas del estilo: 

\begin{verbatim}
MUX_VAL(CP(MCBSP3_DX), (IEN | PTD | DIS | M4)) /*GPIO_140*/\
\end{verbatim}

MUX\_VAL indica que se va a modificar el valor de multiplexado de lo que está entre paréntesis. 

\bigskip
CP(MCBSP3\_DX) es el Control\_PadConf, esto es el registro del microprocesador asociado con el 
pin a modificar. 

\bigskip
(IEN $|$ PTD $|$ DIS $|$ M4) esta es la configuración del pin en cuestión: 


La opción IEN (input enable) hace que el pin sea bidireccional. 

La opción PTD y PTU, indica si el pin tiene un pull down o pull up respectivamente. 

La opción DIS y EN, indica si se deshabilitan o no las opciones PTD y PTU. 

La opción M4 es el modo seleccionado para del pin. Para la Beagleboard existen 4 modos: M1, M2, M3 y M4.

GPIO\_140 es el nombre de la señal (solo es un comentario). 

\bigskip
El Control\_PadConf es un registro de 32bits el cual controla el estado de dos pines, esto es, la parte 
baja del registro controla un pin y la parte alta controla otro. 

\bigskip
Analizando el “manual de referencia BeagleBoard del usuario” (adjunto en el apéndice \ref{HD})(“Expansion connector signals” – tabla 20), “manual técnico de referencia OMAP35x” (“SCM functional description” – capítulo 7.4.4) y agregando las opciones que interesan para los pines, se obtuvo la siguiente tabla: 

\bigskip
{\bf{agregar la tabla!}}


\bigskip
Al modificar el archivo beagle.h hay que tener mucho cuidado ya que al sustituir los valores no se deben repetir pines ni registros, no deben haber incoherencias, un registro por cada pin y un pin por cada registro. 
Dentro del archivo hay un macro definido MUX\_BEAGLE\_C(), donde se deben realizar las modificaciones ya que el modelo utilizado de Beagleboard es el C4 y éste macro es el utilizado para dicho modelo.
En una primera instancia se sustituyeron los valores de la TABLA buscando los equivalentes del 
PadConf en el beagle.h. 


\begin{verbatim}
\#define MUX_BEAGLE_C() \
MUX_VAL(CP(MCBSP3_DX),   (IEN  | PTD | DIS | M4))/*GPIO_140*/\
MUX_VAL(CP(MCBSP3_DR),   (IEN  | PTD | DIS | M4))/*GPIO_142*/\
MUX_VAL(CP(MCBSP3_CLKX), (IEN  | PTD | DIS | M4))/*GPIO_141*/\
MUX_VAL(CP(MCBSP3_FSX),  (IEN  | PTD | DIS | M1))/*UART2_RX*/\
MUX_VAL(CP(UART2_TX),    (IDIS | PTD | DIS | M0))/*UART2_TX*/\
MUX_VAL(CP(MMC2_DAT7),   (IEN  | PTD | EN  | M4))/*GPIO_139*/\
MUX_VAL(CP(UART2_CTS),   (IEN  | PTD | DIS | M4))/*GPIO_144*/\
MUX_VAL(CP(MMC2_DAT6),   (IEN  | PTD | EN  | M4))/*GPIO_138*/\
MUX_VAL(CP(MMC2_DAT5),   (IEN  | PTD | EN  | M4))/*GPIO_137*/\
MUX_VAL(CP(MMC2_DAT4),   (IEN  | PTD | EN  | M4))/*GPIO_136*/\
MUX_VAL(CP(UART2_RTS),   (IEN  | PTD | EN  | M4))/*GPIO_145*/\
MUX_VAL(CP(MCBSP1_DX),   (IEN  | PTD | EN  | M4))/*GPIO_158*/\
MUX_VAL(CP(MMC2_DAT2),   (IEN  | PTD | EN  | M4))/*GPIO_134*/\
MUX_VAL(CP(MCBSP1_CLKX), (IEN  | PTD | EN  | M4))/*GPIO_162*/\
MUX_VAL(CP(MMC2_DAT1),   (IEN  | PTU | EN  | M4))/*GPIO_133*/\
MUX_VAL(CP(MCBSP1_FSX),  (IEN  | PTD | EN  | M4))/*GPIO_161*/\
MUX_VAL(CP(MCBSP1_DR),   (IEN  | PTD | EN  | M4))/*GPIO_159*/\
MUX_VAL(CP(MCBSP1_CLKR), (IEN  | PTD | EN  | M4))/*GPIO_156*/\
MUX_VAL(CP(MCBSP1_FSR),  (IEN  | PTD | EN  | M4))/*GPIO_157*/\
MUX_VAL(CP(I2C2_SDA),    (IEN  | PTD | EN  | M4))/*GPIO_183*/\
MUX_VAL(CP(I2C2_SCL),    (IEN  | PTU | EN  | M4))/*GPIO_168*/\
MUX_VAL(CP(MMC2_DAT3),   (IEN  | PTD | EN  | M1))/*SPI3_CS0*/\
MUX_VAL(CP(MMC2_DAT0),   (IEN  | PTU | EN  | M1))/*SPI3_SOMI*/\
MUX_VAL(CP(MMC2_CMD),    (IEN  | PTU | DIS | M1))/*SPI3_SIMO*/\
MUX_VAL(CP(MMC2_CLK),    (IEN  | PTU | DIS | M1))/*SPI3_CLK*/
\end{verbatim}

Luego es necesario compilar para obtener el u-boot.bin.

\bigskip
\centerline{\$ cd /stuff/build}

\centerline{\$ bitbake -f -c compile -b ../openembedded/recipes/u-boot/u-boot\_git.bb}

\centerline{\$ bitbake -f -c deploy -b ../openembedded/recipes/u-boot/u-boot\_git.bb}

\bigskip
Nota: Cada vez que se introduzca un nuevo cambio, no es necesario ejecutar el comando con la opción clean (lo que implica volver a bajar los fuentes), solo basta con recompilar.

\bigskip
El archivo generado (u-boot.bin) se encuentra en 

/stuff/build/tmp/deploy/glibc/images/beagleboard/ aunque con su nombre seguido de un número identificatorio, el cual debe ser borrado para poder mantener el nombre u-boot.bin.

\bigskip
Pese a que en la literatura y foros, se plantea lo contrario, no fue posible establecer los atributos valor y dirección de los pines GPIO mediante la modificación planteada (una posible solución puede verse en el apéndice \ref{anx_sw_uIm}). Lo que sí cambia efectivamente es el modo del pin, permitiendo obtener las interfaces adecuadas en el bloque de expansión.

%sección 6.4.4
\subsection{uImage}
La versión del kernel elegida fue la 2.6.32 que en el momento del desarrollo era la versión más estable.(ver si en algún lado puse compatibilidad con distro) Aunque también se hicieron pruebas con las versiones 2.6.29 y 2.6.37.
Durante el inicio, el kernel carga los módulos y controladores necesarios para el funcionamiento del 
hardware que forma parte del sistema embebido. También se montan las interfaces para poder interactuar con los distintos dispositivos a ser conectados a la Beagleboard como lo son: SPI, GPIO, UART, etc. Estas interfaces se encuentran bajo el directorio /dev en el sistema de archivos. En algunos casos, no aparecen algunas de las interfaces configuradas en /dev lo que lleva a modificar los fuentes del kernel para que esto así suceda. Este fue el caso de la interfaz SPI que no quedó mapeada en /dev pese a que había sido configurada en los fuentes del u-boot; también hubo problemas con los atributos, valor y dirección de los GPIO como se mencionó anteriormente; adicionalmente hacía falta un módulo para simular una conexión ethernet sobre una interfaz USB para establecer una conexión entre la Beagleboard y un PC como si fuera un enlace de red. 
Todo esto llevó a que se tuvieran que modificar los archivos fuentes del kernel como se muestra en el apéndice \ref{anx_sw_uIm}.
A continuación se detallan los pasos a seguir para la modificación de los fuentes del kernel:
 
\bigskip 
Comando necesarios para el desarrollo del uImage:

\bigskip
\centerline{\$ bitbake virtual/kernel -c comando}

\bigskip
virtual/kernel: refiere a que estamos generando un kernel.

\bigskip
Entre los comandos:

\bigskip
clean: borra el contenido del directorio work. Borra todos los cambios hechos en la configuración del uImage fuente y parches agregados.

patch: genera los archivos de configuración y el fuente del uImage. Además le aplica los parches.

menuconfig: abre el editor de la configuración del kernel.

compile: compila todo.

deploy: genera los archivos referidos en este caso al uImage (.config, módulos, uImage) y los guarda en el directorio /stuff/build/tmp/deploy/glibc/images/beagleboard/.

\bigskip
Nota: Es necesario que todos estos comandos sean ejecutados en el orden adecuado para que todo funcione correctamente.

\bigskip
Primero se configura bitbake para poder utilizarlo:

\centerline{\$ export BBPATH=/stuff/build:/stuff/openembedded}

\centerline{\$ export PATH=/stuff/bitbake/bin:\$PATH}

\bigskip
Se comienza el desarrollo:

\centerline{\$ cd /stuff/build}

\centerline{\$ bitbake virtual/kernel -c clean}

\centerline{\$ bitbake virtual/kernel -c patch}

\centerline{\$ bitbake virtual/kernel -c menuconfig}

\bigskip
Luego de ejecutar este comando se abre el editor de la configuración del kernel (ver figura XXX). Es este editor es donde se indican qué módulos cargar y cuales no. En este caso un cambio de  configuración es necesario para el buen funcionamiento de la interfaz SPI y de la conexión USB-Ethernet con la Beagleboard.

\begin{figure}[H]
\centering
  \begin{center}
  \includegraphics[scale=.3]{Imagenes/kernel.png} 
  \end{center}
  \caption{Editor de configuración del kernel}\label{Fig:kernel} 
\end{figure}

Para configurar la interfaz SPI, se debe configurar como sigue:

Device Drivers – SPI Support=y y luego como en la figura XX.

\begin{figure}[H]
\centering
  \begin{center}
  \includegraphics[scale=.4]{Imagenes/spi_chica.png} 
  \end{center}
  \caption{Configuración SPI}\label{Fig:spi} 
\end{figure}


Para poder establecer la conexión por USB con la Beagleboard, se debe configurar como sigue: 

Device Drivers – USB Support=y – USB Gadget Support=y y luego como en la figuraXX.

\begin{figure}[H]
\centering
  \begin{center}
  \includegraphics[scale=.4]{Imagenes/usb_chica.png} 
  \end{center}
  \caption{Configuración USB Gadget}\label{Fig:usb} 
\end{figure}

Luego, es necesario modificar el archivo board\_omap3beagle.c que se encuentra en /stuff/build/tmp/work/beagleboard-angstrom-linux-gnueabi/linux-omap-.../git/arch/arm/mach-omap2/. En este archivo está toda la inicialización de las interfaces. Los detalles de los cambios introducidos en este archivo se pueden observar en el apéndice XXXX

\bigskip
Ahora se compila y genera el archivo uImage:

\centerline{\$ bitbake virtual/kernel -c compile}

\centerline{\$ bitbake virtual/kernel -c deploy}

\bigskip
Dentro de /stuff/build/tmp/deploy/glibc/images/beagleboard/ se encuentra el archivo uImage generado.

\bigskip
Nota: Respecto al nombre del archivo, al igual que con el caso del u-boot.bin el nombre que aparece es un nombre más largo y necesita ser renombrado a uImage para que se pueda ejecutar correctamente.

%sección 6.4.5
\subsection{FileSystem}
\leftline{Desarrollo de fileSystem}

Como se nombró anteriormente, el filesystem se generó a partir de la herramienta web Narcissus.
En el fileSystem es donde se encuentran los paquetes y programas ya instalados. Cuanto más programas se instalen más grande será en tamaño el fileSystem.

\bigskip
Para utilizar la herramienta Narcissus debemos acceder a la siguiente dirección web: http://narcissus.angstrom-distribution.org/ 

\bigskip
A continuación se detallan las diferentes características y opciones a elegir para crear un filesystem a medida para nuestra SBC y un kit de desarrollo para el PC de desarrollo:

\bigskip
Select Machine: Beagleboard.

Image Name: el nombre que se le quiera dar.

Complexity: complejidad, se eligió advanced ya que la opción simple no brinda libertad de configuración.

Release: versión, aquí hay varias opciones disponibles, se eligió unstable ya que es la más estable de las disponibles. La primera opción disponible es la más estable.

Base System: aquí se elige el soporte de drivers y paquetes que se pretenden. bare bones es la opción con menos soporte y extended es la de mayor soporte. Cuanto más soporte, más pesado se hace el filesystem. Una opción interesante es la opción regular, y es la que se eligió.

/dev manager: esto es el manejador de /dev, se recomienda udev.

Type of Image: formato en el que se quiere descargar el filesystem. Se eligió tar.gz ya que es la opción más versátil.

Software manifest: se genera un archivo en la web con todos los paquetes que se instalaron en detalle.

SDK type: esta opción permite generar un kit de desarrollo para el PC de desarrollo compatible con el filesystem generado. Esto es sumamente útil por ejemplo para croscompilar. Aquí se eligió la opción Full SDK.

User environment section: aquí se indica que tipo de sistema operativo se quiere, básicamente se tienen dos opciones; una es un filesystem sin interfaz gráfica y las otra con entorno gráfico. Se eligió la opción console (sin entorno gráfico) ya que la aplicación no exige entorno gráfico.

\bigskip
Luego se permiten seleccionar programas que se desean instalar.
Las aplicaciones elegidas son: nano editor (editor de texto) ya que hace las cosas más fáciles que el programa vi, GDB y GDBServer necesarios para la depuración de la aplicación, toolchain para tener herramientas de compilación nativas en la Beagleboard.

\bigskip
Cuando todo fue seleccionado, se da un click en build me (demora un poco).
Cuando el proceso termina, se generan dos archivos comprimidos, un archivo con el nombre elegido para la imagen en un formato .tar.gz y el SDK para el PC de desarrollo en formato .tar.bz2. 

%sección 6.4.6
\subsection{Bibliotecas}

\leftline{\bf{Software para el manejo de GPIO}}

El módulo de software para el uso de los puertos de propósito general, GPIO, en principio puede resultar poco importante a simple vista, pero esta porción de código es usada por el resto de los módulos que conforman la aplicación completa del prototipo RF$^{2}$. 
Este módulo cuenta básicamente con una estructura que permite almacenar el estado de cada puerto, una macro y 4 funciones que se datallan a continuación.
La primera de las funciones se llama config\_gpio\_pin() y permite exportar desde el espacio kernel al espacio usuario las funcionalidades necesarias para hacer uso del puerto que se indica como argumento. Al momento en que se exporta, se indica la dirección, o sea si será un puerto de entrada o salida, a través de un parámetro que es pasado a la función.
La función que permite leer el valor actual de un puerto se llama read\_gpio\_pin(), es necesario pasarle como argumento el indicador del puerto del cual queremos conocer su valor. El valor del puerto es guardado en la estructura que almacena el estado de cada puerto para posteriores consultas, sin tener que volver a llamar a dicha función.
Las últimas dos funciones son contrapuestas, set\_gpio\_pin() y clear\_gpio\_pin(), éstas permiten poner el valor de un puerto específico en el valor lógico “1” o “0” respectivamente. Previo a establecer o borrar el valor del puerto ambas funciones verifican que la dirección del puerto sea de salida; como mecanismo de seguridad no es posible cambiar el valor de un puerto de entrada.
Por su parte la macro reset\_status\_gpio() permite borrar el estado de un puerto que ya no esté en uso.

\bigskip
\leftline{\bf{SC}}

Hoy en día la mayoría de los lectores de tarjetas de contacto tienen una interfaz USB para ser conectado en un PC en aplicaciones de escritorio. Para el uso de este tipo de lectores sobre Linux existe un controlador genérico llamado CCID.
Sin embargo las tarjetas de contacto no poseen un puerto USB sino un puerto serie para establecer la comunicación con algún dispositivo, es por esto que en la nueva generación de lectores siempre hay un  ASIC para lograr la interacción, por un lado con la tarjeta de contacto y por el otro la comunicación con el PC.
Como fue descrito en la sección de hardware, el lector de tarjetas de contacto tiene una interfaz serial pura para la transferencia de datos con las tarjetas. En base al diseño hardware elegido, las capas de software sobre las que se decidió trabajar son las que se detallan en la figura XXXX. 

\begin{figure}[H]
\centering
  \begin{center}
  \includegraphics[scale=.4]{Imagenes/SW_sc1.jpg} 
  \end{center}
  \caption{Capas de software de trabajo}\label{Fig:capas} 
\end{figure}

En una primera etapa y para simplificar el desarrollo y la depuración del software, la capas empleadas fueron las que se muestran en la figura XXXX.


\begin{figure}[H]
\centering
  \begin{center}
  \includegraphics[scale=.4]{Imagenes/SW_sc2.jpg} 
  \end{center}
  \caption{Capas de software en una primera etapa}\label{Fig:capas0} 
\end{figure}

\bigskip
\leftline{Descripción de las capas:}

\bigskip
\leftline{controlador}
El kernel es el encargado de manipular directamente los registros del puerto serial, las interrupciones que desde éste se generan y la ISR para atender las interrupciones.
La implementación del controlador del lector de tarjetas se basó en el controlador serial de Linux a través de su estructura “termios”. Esta estructura nos permite configurar todos los parámetros necesarios para la comunicación serial como ser, baud rate, cantidad de bits por byte, bit de paridad, bit de parada entre otros. Las funciones read y write permiten la lectura y escritura de los bytes de datos que son recibidos y transmitidos por el puerto serial.

\bigskip
\leftline{CT/API (Card Terminal / Application Programming Interface)}
Por encima del controlador serial se encuentra CT/API[ref], una interfaz definida por varias emprasas entre las que se incluye Telekom Alemania en la década de los noventa, que permite encapsular el controlador específico de cada lector de tarjetas, de manera que la aplicación final no se vea afectada al cambiar un lector por otro.
Esta interfaz de programación está formada tan solo por 3 funciones, CT\_init, CT\_data y CT\_close, que permiten la inicialización del lector, la transferencia de datos entre host/lector o host/tarjeta (host se refiere a la SBC o PC donde se encuentra conectado el lector de tarjetas de contacto) directamente y el cierre de la comunicación.
CT\_init se encarga del pasaje de parámetros a la capa del controlador, para la configuración del puerto de comunicación entre el host y el lector de tarjetas. Los parámetros en uso aquí son: la tasa de transferencia de datos, el número de bits por cada byte, el tipo de paridad empleado y el puerto serie a ser utilizado.
CT\_data es la función encargada de transferir comandos y datos hacia y desde la tarjeta o hacia y desde el lector (en caso que el mismo esté formado por un ASIC o microprocesador). La manera de diferenciar desde donde es enviado el dato, es a través de un parámetro pasado a esta función, y de forma análoga se determina el destino del mensaje. El protocolo usado para la transferencia de datos es T=0, orientado a bytes y del cual pueden conocerse más detalles en [handbook sc]
CT\_close es la contracara de  CT\_init, se encarga de cerrar la comunicación con el lector. Lo que hace básicamente es liberar el handle (puntero) asociado al puerto serial.
Para el caso en que los comandos y/o datos estén dirigidos hacia el lector, existe otra especificación, llamada CT/BCS (Card Terminal / Basic Command Set), donde se encuentran definidos una serie de comandos básicos para el manejo del lector. Estos comandos se numeran a continuación y se da una breve descripción, por más detalles referirse a la mencionada especificación [ref]:

RESET CT permite reiniciar el lector o tarjeta (en el caso de ser un lector mudo); de manera opcional puede devolver el ATR.

REQUEST ICC tiene como objeto devolver el ATR de la tarjeta una vez que la misma se encuentra ubicada en el zócalo del lector.

GET STATUS es empleado para conocer información sobre el lector o si la tarjeta está insertada y eléctricamente conectada en el lector. 

EJECT ICC genera la desactivación eléctrica de la tarjeta.

\bigskip
\leftline{IFDHandler}
El siguiente componente en este stack de capas es ifdhandler[ref], no es otra cosa que un conjunto de funciones formando una API, empleada por pcsclite para encapsular el manejo del hardware de  lectores cuyos fabricantes quieran cumplir con las especificaciones PC/SC[ref]. Una venaja importante de esta API es que le permite a pcsclite operar tanto con lectores de puerto serial como con lectores de puerto USB.
Esta capa de software podría usarse directamente sobre el controlador del lector, prescindiendo de CT/API, aunque se decidió mantenerla por motivos de simplicidad ya que sólo es necesario sustituir la capa de aplicación por las restantes capas superiores como se indica en las figuras anteriores.
En lo que sigue se enumeran algunas de las funciones de esta API y se describen brevemente. Por más detalles ver el manual ifdhandler [ref].
IFDHCreateChannel establece el canal de comunicación con el lector. Para conseguirlo usa un parámetro llamado Channel, que indica cual es el puerto serial a usar, por ejemplo para el caso de Linux /dev/ttySx (x es el número que corresponda).
IFDHCloseChannel implementa la acción opuesta a la función anterior, cerrando el canal de comunicación con el lector de tarjetas.
IFDHGetCapabilities permite obtener las capacidades específicas del lector o de la tarjeta insertada en el mismo.
IFDHPowerICC se encarga del control de las señales de alimentación y reset que el lector suministra a la tarjeta. Desempeña tres acciones posibles, encendido, reset y apagado de la tarjeta.
IFDHTransmitToICC se encarga de la transferencia de datos con la tarjeta a través de alguno de los protocolos disponibles, como ser T=0 o T=1.
IFDHICCPresence retorna el estado de la tarjeta insertada en el zócalo del lector.

\bigskip
\leftline{PCSCLite}
Por arriba de ifdhandler se encuentra la librería pcsclite, ésta contiente todas las funciones necesarias para establecer la comunicación con un lector y la tarjeta conectada a éste último. Para usar el controlador encapsulado por ifdhandler desde pcsclite es necesario seguir los pasos de configuración detallados en el apéndice.

\bigskip
\leftline{Aplicación final}
Por arriba de toda las capas descritas antes se encuentra la aplicación del prototipo, que hace uso de las funciones suministradas por pcsclite y donde se encuentran definidos los comnados APDU específicos con los que opera la tarjeta de contacto.


\bigskip
\leftline{\bf{RFID}}
A continuación se detallan los cambios introducidos en la librfid para el correcto funcionamiento del lector/escritor RFID utilizando librfid-tool.

\bigskip
Se analizó el código principal de la aplicación librfid-tool y se vio que dentro del directorio utils en el archivo common.c se encuentra la función reader\_init la cual busca un lector/escritor entre los soportados. Esta función no tenía implementada una búsqueda para dispositivos conectados por SPI. Por lo tanto se tuvieron que agregar algunas líneas a la función para que el funcionamiento fuera posible:

\begin{verbatim}
rh = rfid_reader_open("/dev/spidev3.0", RFID_READER_SPIDEV);
if (!rh) {
    fprintf(stderr, "No SPIDEV found\n");
    return -1;
}
\end{verbatim}

Este cambio permitió detectar la interfaz SPI (spidev3.0). Algo a tener en cuenta, es que como en Linux las interfaces están asociadas a un archivo, el hecho de abrir el archivo no implica que haya nada conectado en esa interfaz. Por esta razón, la búsqueda de un lector/escritor conectado por interfaz SPI se realiza en última instancia.

\bigskip
Otro cambio fundamental es en la frecuencia de trabajo del SPI, se modifica a 10MHz, ya que con la frecuencia establecida en la librfid (1MHz) el lector/escritor RFID no funciona correctamente. Toda la configuración de la comunicación por SPI se encuentra en rfid\_reader\_spidev.c que está en /src.
La función que se modificó es spidev\_open y el cambio se muestra a continuación:

\begin{verbatim}
tmp = 10e6; /* 10 MHz */
if (ioctl(spidev_fd, SPI_IOC_WR_MAX_SPEED_HZ, &tmp) < 0)
    goto out_rath;
\end{verbatim}

\subsection{Aplicación final}

Para el desarrollo de la aplicación RF$^{2}$, se decidió trabajar sobre los fuentes de la herramienta librfid-tool ya que maneja varias funciones de utilidad y es de ayuda a la hora de compilar para el armado de una aplicación completa. Se mantuvieron todas las opciones de la herramienta ya que son muy útiles y pueden ayudar en un futuro para establecer orígenes de fallas. No se modificó ninguna función de la aplicación original y cuando fue necesaria alguna modificación, se procedió a implementar una nueva.

\bigskip
Antes de seguir fue necesario entender el funcionamiento de las reglas de compilación creadas para la aplicación librfid-tool sobre librfid. En el directorio raíz, se encuentran los siguientes archivos importantes para el desarrollo del sistema: autogen.sh, configure.in, configure, Makefile.am, Makefile.flags.am, Makefile.in, Makefile.

\bigskip
A continuación se describe a grandes rasgos la utilidad de cada uno de estos archivos:

\bigskip
configure.in es el archivo de configuración con el cual se crea el archivo configure.
Makefile.am es el archivo que establece las reglas de compilación (orden, dependencias, etc) y es con el cual se crea el archivo Makefile.in. Este archivo se encuentra en todos los subdirectorios de la librfid, por lo que se crea un Makefile.in dentro de cada subdirectorio.
Makefile.flags.am es un archivo con banderas establecidas para el sistema. Este archivo está incluido en todos los Makefile.am de la aplicación.
autogen.sh es un script que se encarga de crear los archivos configure y Makefile.in, cada uno de ellos creados a partir de los archivos correspondientes antes mencionados.
Al ejecutar el archivo configure, este establece la configuración que necesita el Makefile.in para conocer las reglas de compilación que se van a usar y algunos parámetros extra. Luego de este proceso se generan los Makefile.
El archivo Makefile es el que conoce las reglas de compilación, las dependencias y las opciones elegidas para el desarrollo. Sabe incluso donde se encuentran los demás Makefile para ejecutarlos cuando sea necesario.

\bigskip
Cada subdirectorio tiene reglas locales de construcción de sus objetos y a su vez se ayudan mutuamente para lograr una aplicación más completa. Desde el Makefile principal se controla que todo funcione correctamente.
Para saber más sobre reglas de compilación ver JHHJHJHJH

\bigskip
La aplicación RF$^{2}$ utiliza otras bibliotecas además de librfid, además estas bibliotecas van a interactuar entre sí, lo que lleva a que aparezcan nuevas dependencias. Por lo tanto, es necesario modificar las reglas de compilación.

\bigskip
Se agregó en el raíz de la librfid el directorio rf2 en el cual se encuentran subdirectorios con los fuentes necesarios para la comunicación con el display, leds, buzzer, tarjeta de contacto y otros utilitarios, formando la estructura que se muestra en la figura \ref{est_RF2}. 


\begin{figure}[H]
\centering
  \begin{center}
  \includegraphics[scale=.5]{Imagenes/estructura_librfid.png} 
  \end{center}
  \caption{Estructura de árbol de aplicación RF$^{2}$}\label{est_RF2} 
\end{figure}

gpio incluye fuentes para el manejo de los GPIO.

lcd incluye fuentes para el manejo del display.

rf incluye fuentes con funciones extra de utilidad para el manejo del lector/escritor RFID diseñado.

sam incluye los fuentes para la comunicación con la tarjeta de contacto.

utiles incluye fuentes con funciones útiles de la aplicación.

\bigskip
Dentro de cada uno de estos directorios se encuentra un archivo Makefile con las reglas de compilación correspondiente a cada uno.

\bigskip
Las modificaciones a los archivos de construcción de la aplicación para que contemplen el agregado del directorio rf2 como la creación de nuevas reglas para la construcción dentro de éste, se detallan a continuación.

\bigskip
Antes que nada se decidió modificar los archivos Makefile.am para que sepan de la existencia del nuevo directorio. Se modifican estos archivos debido a que nunca son borrados. Por ejemplo, si los cambios se hacen sobre los Makefile.in, puede pasar que en elgún momento se regeneren borrando los cambios que se hicieron. Además, luego de entender la estructura y funcionamiento de estos archivos, es más fácil incluir una modificación en Makefile.am que en un Makefile.in o Makefile directamente.
Dentro del directorio utiles, se creó el archivo Variables\_Make que establece el valor de algunas variables específicas de la aplicación RF$^{2}$ como ser CC\_arm que hace referencia al gcc asociado con la herramienta de croscompilación para ARM. 

\bigskip
Makefile.flags.am:
Aquí se indicó que se agregue rf2 a la ruta de búsqueda de archivos encabezados.

\begin{verbatim}
INCLUDES = \$(all_includes) -I$(top_srcdir)/include 
-I\$(top_srcdir)/rf2
\end{verbatim}

Makefile.am en el raíz:

\begin{verbatim}
include rf2/utiles/Variables_Make
\end{verbatim}
Con esto se aseguró que el resto de los subdirectorios pertenecientes a la librfid, sepan los valores de las variables específicas de la aplicación RF$^{2}$.

Luego se incluye el directorio rf2 a la aplicación.
\begin{verbatim}
if ENABLE_SPIDEV 
SUBDIRS += rf2 
endif
\end{verbatim}

\bigskip
Makefile.am en utils:
Este es el cambio más dificil de entender y surge de la observación del Makefile.in generado cada vez (prueba y error). Como la aplicación RF$^{2}$ se basa en la modificación de los fuentes de la herramienta librfid-tool, se deben agregar todas las dependencias con archivos del nuevo subdirectorio rf2.

\begin{verbatim}
librfid_tool_SOURCES = librfid-tool.c librfid-tool.h 
common.c common.h ../rf2/gpio/gpio.c ../rf2/gpio/gpio.h 
../rf2/rf/rc632_utils.c ../rf2/rf/rc632_utils.h 
../rf2/lcd/lcd16x2.c ../rf2/lcd/lcd16x2.h ../rf2/sam/sam.c 
../rf2/sam/sam.h ../rf2/sam/sam_util.c ../rf2/sam/sam_util.h 
../rf2/utiles/utiles.c ../rf2/utiles/utiles.h
\end{verbatim}

Cada .h y .c se traduce en un .o al crear el Makefile.in.

\bigskip
Otro cambio necesario, ya que si no se incluye provoca errores de compilación es el siguiente.

\begin{verbatim}
mifare_tool_SOURCES = mifare-tool.c common.c 
../rf2/gpio/gpio.c ../rf2/rf/rc632_utils.c 
../rf2/lcd/lcd16x2.c ../rf2/sam/sam.c 
../rf2/sam/sam_util.c ../rf2/utiles/utiles.c
\end{verbatim}

Mifare-tool es otra herramienta de la librfid.

\bigskip
Luego se creó dentro del directorio rf2 un archivo Makefile que es el que ordena la construcción de todos los objetivos dentro de cada subdirectorio.

\begin{verbatim}
include utiles/Variables_Make 

all: gpio/gpio.o rf/rc632_utils.o lcd/lcd16x2.o 
sam/sam.o utiles/utiles.o 

gpio/gpio.o: 
	$(MAKE) -C gpio 

rf/rc632_utils.o: 
	$(MAKE) -C rf 

lcd/lcd16x2.o: 
	$(MAKE) -C lcd 

sam/sam.o: 
	$(MAKE) -C sam 
	 
utiles/utiles.o: 
	$(MAKE) -C utiles 

install: 

clean: 
	rm -f *.o 
	$(MAKE) -C gpio clean 
	$(MAKE) -C rf clean 
	$(MAKE) -C lcd clean 
	$(MAKE) -C sam clean 
	$(MAKE) -C utiles clean 

distclean: clean
\end{verbatim}

\bigskip
En este caso \$(MAKE) -C “directorio” indica que la regla de construcción del objetivo se encuentra en “directorio”.

\bigskip
Se agregó una nueva opción (n) en el main de la aplicación librfid-tool que llama a la función principal(). Ésta es la función principal de la aplicación RF$^{2}$. De este modo no se modifica el main original de librfid-tool y se dejan las opciones por defecto.
Se hizo uso de algunas funciones ya escritas y se crearon otras. En este punto no se modificó ninguna función de la aplicación original y cuando fue necesaria alguna modificación, se procedió a implementar una nueva función con las modificaciones previstas.
Luego viene la etapa de croscompilación de la aplicación para ser probada en la SBC.

\bigskip
Croscompilación de la aplicación final:
Para guardar el resultado de la croscompilación se creó un directorio work en el home del usuario.

\bigskip
\centerline{\$ ./autogen.sh}
Este paso es necesario siempre que se modifiquen los Makefile.am, en otro caso no. Se recuerda que este script genera los archivos Makefile.in y configure.

\bigskip
\centerline{\$ ./configure --enable-spidev --host=arm-angstrom-linux-gnueabi --prefix=/home/proyecto/work}
Se configura el sistema para lectores/escritores RFID con interfaz SPI, se indica la arquitectura de la SBC para la cual compilamos y se indica el directorio donde se instala la aplicación. Luego de hecho esto ya no es necesario volver a ejecutarlo ya que los Makefile ya quedan creados con esta configuración.

\centerline{\$ make clean \&\& make -j5 \&\& make install}
Se construye la aplicación.

Luego, en el directorio work se encuentran cuatro directorios creados:

\bigskip
bin: incluye los binarios construidos.

include: directorio con los .h necesarios para correr la aplicación.

lib: la biblioteca en sí.

share: nada de utilidad.

\bigskip
El paso siguiente es el de copiar estos directorios en el sistema de archivos de la SBC.

\bigskip
Copia de archivos a la SBC:

\bigskip
Para realizar la copia conviene comprimir el resultado de la croscompilación y luego enviarlo a la SBC.

\centerline{\$ tar -czf rf2.tar.gz bin include lib share}

\bigskip
Luego de enviado el archivo comprimido, es necesaria la copia de estos directorios en el sistema de archivos de la SBC.

\bigskip
Instalación en la SBC:

\centerline{\$ tar -xf rf2.tar.gz -C /usr}
Esto descomprime rf2.tar.gz bajo /usr.

\bigskip
Se ejecuta la aplicación RF$^{2}$:

\centerline{\$ librfid-tool -n}


%sección 6.5
\section{Ejecución de programa principal}

%sección 6.5.1
\subsection{Script para ejecución autónoma}

%Hojas de datos
\chapter{Hojas de datos}\label{HD}

A continuación se adjuntan algunas de las hojas de datos a las que se hace referencia en el texto.
Se encuentran ordenadas de acuerdo al siguiente listado, respetando el orden en el cual se mencionan:

\begin{itemize}
\item MIFARE® and I Code CL RC632 Multiple protocol contactless reader IC.
\item 8-BIT BIDIRECTIONAL VOLTAGE-LEVEL TRANSLATOR.
\item LM1117/LM1117I 800mA Low-Dropout Linear Regulator.
\item Transistor 2N3904.
\item Transistor 2N3906.
\item ESDA6V1W5.
\item HD44780U (LCD-II), Dot Matrix Liquid Crystal Display Controller/Driver.
\item LCM-S01602DSF-A - hoja de datos del LCD16x2.
\item Micore Reader IC Family; Directly Matched Antenna Design.
\item Mifare®(14443A) 13,56 MHz RFID Proximity Antennas.
\item BeagleBoard System Reference Manual Rev C4, Revision 0.0.
\item OMAP 35x Applications Processor Technical Reference Manual.
\item Hawkboard Press Release.
\item Hawkboard Press Release Solution.
\item Build a cheap 13.56MHz MIFARE antenna for the Proxmark.
\item Angström Manual - Embedded Power -.
\end{itemize}

%%RC632 - cap 4.1
%\includepdf[pages=1-163]{PDF/RC632.pdf}
%
%%conversor de niveles de tensión - cap 4.4.2
%\includepdf[pages=1-24]{PDF/txb0108.pdf}
%
%%regulador de tensión LDO - cap 4.4.2
%\includepdf[pages=1-21]{PDF/LM1117.pdf}
%
%%2N3904 - cap 4.4.3
%\includepdf[pages=1-17]{PDF/2N3904.pdf}
%
%%2N3906 - cap 4.4.3
%\includepdf[pages=1-16]{PDF/2N3906.pdf}
%
%%ESDA6V1W5 - cap 4.4.3
%\includepdf[pages=1-11]{PDF/esdaxxxwx.pdf}
%
%%HD44780 - cap 4.4.3
%\includepdf[pages=1-60]{PDF/hd44780u.pdf}
%
%%lcd 16x2 - no se menciona
%\includepdf[pages=1-2, angle=90]{PDF/LCM-S01602DSF-A.pdf}
%
%%Micore Reader IC Family... - cap 4.4.4
%\includepdf[pages=1-42]{PDF/AN077925.pdf}
%
%%Mifare®(14443A) 13,56 MHz... - cap 4.4.4
%\includepdf[pages=1-26]{PDF/78010.pdf}
%
%%Beagleboard reference manual - cap 6.4.3
%\includepdf[pages=1-180]{PDF/BBSRM_latest.pdf}
%
%%OMAP 35x Applications Processor Technical Reference Manual - cap 6.4.3
%\includepdf[pages=768-785]{PDF/OMAP35x_TRM.pdf}
%
%%Nota - cap 7.1
%\includepdf{PDF/Hawkboard_Press_Release.pdf}
%\includepdf[pages=1-2]{PDF/Hawkboard_Press_Release_Solution.pdf}
%
%%Proxmark - cap 7.4
%\includepdf{PDF/Proxmark.pdf}
%
%%Manual Angström - Apéndice sw
%\includepdf[pages=1-48]{PDF/angstrom-manual.pdf}

% Bibliografía:
\part{Bibliografía}
\bibliographystyle{plain}
\begin{thebibliography}{200}

%Application Notes
\bibitem{ACD} Antenna Circuit Design for RFID Applications, AN710.
\bibitem{ADN} HF Antenna Design Notes, Literature Number: 11-08-26-003.
\bibitem{HFACB} HF Antenna Cook Book, Lit. Number 11-08-26-001.
\bibitem{MRICF} Micore Reader IC Family; Directly Matched Antenna Design, Rev. 2.05, 10 May 2006.
\bibitem{RFIDPA} Mifare®(14443A) 13,56 MHz RFID Proximity Antennas, Application Note.
\bibitem{} MIFARE Type Identification Procedure; AN10833 Rev. 3.1 — 07 July 2009.
\bibitem{} MIFARE ISO/IEC 14443 PICC Selection; AN10834 Rev. 3.0 — 26 June 2009.
\bibitem{} MIFARE Classic Commands; AN10856 Rev. 1.0 — 31 July 2009.
\bibitem{} MIFARE and handling of UIDs; AN10927 Rev. 2.0 — 01 September 2010.
\bibitem{} Mainstream contactless smart card IC for fast and easy; MF1S5009 Rev. 3 — 27 July 2010.
\bibitem{} MIFARE Standard Card IC MF1 IC S50 Functional Specification; Revision 5.1 May 2001.

%Hojas de datos
\bibitem{RC632} MIFARE® and I Code CL RC632 Multiple protocol contactless reader IC.
\bibitem{HD_VLT} 8-BIT BIDIRECTIONAL VOLTAGE-LEVEL TRANSLATOR.
\bibitem{dpy} HD44780U (LCD-II), Dot Matrix Liquid Crystal Display Controller/Driver, Rev. 0.0 1998. 
\bibitem{LDO} LM1117/LM1117I 800mA Low-Dropout Linear Regulator.
\bibitem{LCM} LCM-S01602DSF/A - hoja de datos del LCD16x2.
\bibitem{manualBb} BeagleBoard System Reference Manual Rev C4, Revision 0.0.

%HandBooks
\bibitem{SCHb} Smart Card Handbook, Third Edition, Wolfgang Rankl and Wolfgang Effing, 2003, ISBN: 0-470-85668-8. 
\bibitem{RFIDHb} RFID Handbook: Fundamentals and Applications in Contactless Smart Cards and Identification,
Second Edition, Klaus Finkenzeller, 2003, ISBN: 0-470-84402-7. 

%CT/API cap-6
\bibitem{ctapi} Application Independent CardTerminal Application Programming Interface for ICC Applications; 
Deutsche Telekom AG / PZ Telesec, GMD - Forschungszentrum Informationstechnik GmbH, TÜV Informationstechnik GmbH, TELETRUST Deutschland e.V., 30.10.1996.

%CT/BCS cap-6
\bibitem{ctbcs} Application independent CardTerminal Basic Command Set for ICC applications; 
TeleTrusT Deutschland e.V., GMD Forschungszentrum Informationstechnik GmbH, Version 0.9, 28.07.1995.


%IFDhandler
\bibitem{ifdhandler} IFDHandler 3.0 MUSCLE PC/SC IFD Driver API; David Corcoran \& Ludovic Rousseau, July 28, 2004.


%Libros
\bibitem{KiCad} Diseño y desarrollo de circuitos impresos con KiCad, Miguel Pareja Aparicio, 2010, ISBN: 978-84-937769-1-6.


%Normas ISO
\bibitem{7816} Norma ISO/IEC 7816.
\bibitem{14443} Norma ISO/IEC 14443.


%LINKS

\bibitem{OpenPCD} OpenPCD, http://www.openpcd.org/, agosto 2011.
\bibitem{usb4all} USB4ALL, http://www.fing.edu.uy/inco/grupos/mina/pGrado/pgusb/main.php, agosto 2011.
\bibitem{librfid} librfid, http://openmrtd.org/projects/librfid/, agosto 2011.
\bibitem{9G20} GES9G20, http://www.glomationinc.com/product\_9G20.html, agosto 2011.
\bibitem{Hawk} Hawkboard, http://www.hawkboard.org/, agosto 2011.
\bibitem{Beagle} Beagleboard, http://beagleboard.org/, agosto 2011.
\bibitem{foroBb} Foro de Beagleboard, https://groups.google.com/forum/\#!forum/beagleboard, agosto 2011.
\bibitem{OE-Bb} Openembedded-Bitbake, http://wiki.openembedded.net/index.php/BitBake\_\%28dev\%29, agosto 2011.
\bibitem{KiCad} KiCad, http://www.lis.inpg.fr/realise\_au\_lis/kicad/, agosto 2011.
\bibitem{gEDA} gEDA, http://www.gpleda.org/index.html, agosto 2011.
\bibitem{Proxmark} Proxmark, http://www.proxmark.org/proxmark, agosto 2011.
\bibitem{Angs} Angström, http://www.angstrom-distribution.org/, agosto 2011.
\bibitem{Angs_MLO} MLO, http://www.angstrom-distribution.org/demo/beagleboard/MLO/, agosto 2011.
\bibitem{Narc} Narcissus, http://narcissus.angstrom-distribution.org/, agosto 2011.
\bibitem{termios} termios, http://pubs.opengroup.org/onlinepubs/007908799/xsh/termios.h.html, agosto 2011.
\bibitem{linuxBoot} Proceso de booteo de linux, http://www.ibm.com/developerworks/linux/library/l-linuxboot/index.html, agosto 2011.
\bibitem{Make} Make, http://es.wikipedia.org/wiki/Make, agosto 2011.
\bibitem{DDD} DDD, http://www.gnu.org/s/ddd/, agosto 2011.
\bibitem{gpio} GPIO, http://www.avrfreaks.net/wiki/index.php/Documentation:Linux/GPIO\#Interfaces\_explained, agosto 2011.
\bibitem{gpioK} Kernel GPIO, http://kernel.org/doc/Documentation/gpio.txt, agosto 2011.

\bibitem{conf_SDK} Configuración SDK, http://www.electronsonradio.com/2011/04/intro-to-basic-cross-compiling-for-the-beagleboard/, agosto 2011.


%links sobre smart cards y comandos linux necesarios para entender código
\bibitem{} http://ladyada.net/make/simreader/
\bibitem{} http://www.freeinfosociety.com/electronics/schemview.php?id=1995
\bibitem{} http://www.gooze.eu/howto/smartcard-quickstarter-guide
\bibitem{} http://linux.die.net/man/3/dlsym
\bibitem{} http://www.mkssoftware.com/docs/man3/select.3.asp
\bibitem{} http://tldp.org/HOWTO/Program-Library-HOWTO/dl-libraries.html

%links sobre puerto serial y comandos linux necesarios para entender código
\bibitem{} http://www.easysw.com/~mike/serial/serial.html
\bibitem{} http://linux.die.net/man/3/fd\_set
\bibitem{} http://linux.die.net/man/2/select\_tut
\bibitem{} http://www.iearobotics.com/wiki/index.php?title=Tutorial:Puerto\_serie\_en\_Linux
\bibitem{} http://es.tldp.org/COMO-INSFLUG/COMOs/Programacion-Serie-Como/Programacion-Serie-Como-3.html
\bibitem{} http://www.rastersoft.com/articulo/pserie.html
\bibitem{} http://www.faqs.org/docs/Linux-mini/IO-Port-Programming.html
\bibitem{} http://tldp.org/HOWTO/IO-Port-Programming-2.html

%links sobre pcsc
\bibitem{pcsclite} PCSC-Lite, http://pcsclite.alioth.debian.org/
\bibitem{pcsclite_esp} PCSC Workgroup, http://www.pcscworkgroup.com/
\bibitem{pcsc/ccid_down} Descargas PCSC-Lite, https://alioth.debian.org/frs/?group\_id=30105
\bibitem{pcsctools_down} http://ludovic.rousseau.free.fr/softwares/pcsc-tools/
\bibitem{} http://ludovicrousseau.blogspot.com/2011/04/libccid-and-usb-selective-suspend.html
\bibitem{} http://readthedocs.org/docs/pvdevtools\_doc/en/latest/index.html
\bibitem{} http://www.springcard.com/support/pcsclite.html

\end{thebibliography}


\end{document}
