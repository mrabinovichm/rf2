\begin{glosario}

{\bf{AFE}} – Artefacto Feo de Exhibir, dispositivo para recargar tarjetas RFID.

{\bf{APDU}} – Application Protocol Data Unit, comando enviado desde la capa de aplicación a través de un lector a una tarjeta inteligente.

{\bf{ASCII}} – American Standard Code for Information Interchange, Código Estadounidense Estándar para el Intercambio de Información.

{\bf{ASIC}} – Application-Specific Integrated Circuit, circuito integrado para aplicaciones específicas

{\bf{ASK}} – Amplitude Shift Key, modulación por amplitud de pulsos.

{\bf{ATR}} – Answer To Reset, respuesta de una tarjeta inteligente luego de un reset.

{\bf{baud rate}} – Usado de manera análoga a la tasa de bits por segundo.

{\bf{CCID}} – Chip/Smart Card Interface Devices controlador genérico de dispositivos lectores de smart card con interfaz USB.

{\bf{CL RC632}} – Circuito integrado para lectores de tarjetas RFID de protocolo múltiple.

{\bf{EMC}} – Electromagnetic Compatibility, conjunto de reglas dadas por la FCC para la compatibilidad electrónica de sistemas electrónicos.

{\bf{ext3}} – Sistema de archivos extendido, usado en sistemas operativos Linux.

{\bf{FAT32}} – Sistema de archivos FAT (File Allocation Table) de 32 bits, desarrollado para MS-DOS.

{\bf{FCC}} – Federal Communications Commission, Comisión Federal de Comunicaciones de EEUU.

{\bf{FIFO}}  – First In First Out, consepto usado para indicar que el primer dato en ser leído es el primero en ser procesado y liberado.

{\bf{fileSystem}} – Sistema de archivos manejado por el sistema operativo.

{\bf{GDB}} – GNU Debugger,  herramienta para depurar código fuente.

{\bf{GPIO}} – General Purpose Input/Output  puerto de entrada/salida de propósito general.

{\bf{I/O}} – Input/Output puerto de entrada/salida.

{\bf{I2C}} – Inter-Integrated Circuit, bus de comunicaciones serie para interconectar microcontroladores y/o circuitos integrados entre sí.

{\bf{IDE}} – Integrated Development Environment, herramienta para el desarrollo de software.

{\bf{IIE}} – Instituto de Ingeniería Eléctrica.

{\bf{ISO}} – International Organization for Standardization, organización internacional para la estandarización.

{\bf{ISR}} – Interrupt Service Routine, rutina de atención a una interrupción.

{\bf{JTAG}} – Joint Test Action Group, puerto que permite entre otras cosas depurar la aplicación que se ejecuta en el sistema embebido.

{\bf{LCD16x2}} – Liquid Crystal Display, pantalla de dos líneas de 16 caracteres cada una.

{\bf{LDO}} – Low Drop Output, salida de baja caída en reguladores de tensión.

{\bf{lector mudo}} – Lector de tarjetas, que no cuenta con un ASIC entre la tarjeta y el microcontrolador.

{\bf{Mifare}} – Tecnología de tarjetas inteligentes sin contacto (TISC) que cumplen con el estándar ISO14443.

{\bf{Mifare Classic}} – Tarjetas inteligentes sin contacto, con memoria no volátil de 1K o 4K.

{\bf{open-firmware}} – Software que permite el manejo directo del hardware donde se encuentra almacenado, cuyo código es abierto, permitiendo ser modificado y distribuído en forma libre.

{\bf{open-hardware}} – Dispositivos de hardware cuyas especificaciones y diagramas esquemáticos son de acceso público.

{\bf{open source}} – Software cuyo código es abierto, permitiendo ser modificado y distribuído en forma libre.

{\bf{OpenPCD}} – Dispositivo lector/escritor de tarjetas RFID.

{\bf{OTG}} – On The Go, clase de puerto USB que puede ser usado como Host o como Device.

{\bf{PC}} – Personal Computer, computador personal.

{\bf{PCB}} – Printed Circuited Board, placa de circuito impresa.

{\bf{pcsclite}} – Biblioteca para el uso con smart cards.

{\bf{PIC}} – Familia de microcontroladores tipo RISC (reduced instruction set computer) fabricados por Microchip Technology Inc.

{\bf{Protocolo T=0}} – Protocolo de datos orientado a byte.

{\bf{Protocolo T=1}} – Protocolo de datos orientado a bloques.

{\bf{RF}} – Radio Frecuencia.

{\bf{RF$^{2}$}} – Recarga Fácil por Radio Frecuencia.

{\bf{RS232}} – Interfaz que designa una norma para el intercambio serie de datos binarios entre un DTE (Equipo terminal de datos) y un DCE (Data Communication Equipment, Equipo de Comunicación de datos).

{\bf{RSTPD}} – Reset and Power Down, pin de reset y apagado del integrado CL RC632.

{\bf{SBC}} – Single Board Computer.

{\bf{SC}} – Smart Card, tarjeta inteligente.

{\bf{SCUI}} – PCB que contiene un lector/escritor de tarjetas de contacto e interfaz de usuario.

{\bf{SDK}} – Software Development Kit, herramienta para desarrollo de software.

{\bf{SIMO}} – Slave In Master Out, entrada esclavo salida maestro en un puerto SPI.

{\bf{SOC}} – System On a Chip.

{\bf{SOMI}} – Slave Out Master In, salida esclavo entrada maestro en un puerto SPI.

{\bf{SPI}} – Serial Peripheral Interface, bus de comunicación serie.

{\bf{UART}} – Universal Asynchronous Receiver-Transmitter, puerto serie con transferencia de datos de forma asíncrona.

{\bf{UID}} – Código único de identificación de tarjeta RFID.

{\bf{USB}} – Universal Serial Bus.

{\bf{USB4ALL}} – Interfaz USB genérica para comunicación con dispositivos electrónicos.

{\bf{VLT}} – Voltage Level Translator, nombre del PCB que contiene circuitos integrados para el traslado de niveles de tensión.

\end{glosario}