\chapter{Funcionamiento del prototipo}

\section{Requerimientos}
Los requerimientos que deben cumplirse son los mismos que fueron definidos para el dispositivo AFE.
Es necesario poder leer y escribir tarjetas RFID, comunicarse con servidores STM, leer tarjetas de contacto (módulo de seguridad), informar al usuario de lo que ocurre a través de una interfaz simple.

\section{Funcionamiento general del prototipo}
%Esto hay que corregirlo, los mensajes son otros :)
Normalmente el prototipo está en espera de la llegada de un usuario, desplegando los
mensajes “Sistema de recarga STM”, “Aproxime su tarjeta”, en forma alternada con un
pequeño intervalo de tiempo de espera entre cada mensaje.
Cuando llega un usuario, éste acerca su tarjeta RFID y el dispositivo desplegará el mensaje
“No retire su tarjeta”. Si el usuario retirara la tarjeta, de modo de que el dispositivo no pueda
leerla, el mismo desplegará el mensaje “Mantenga cerca su tarjeta”. Si no es posible leer la
tarjeta el prototipo lo indicará desplegando el mensaje "Lectura inválida. Vuelva a
intentarlo.".
Si el servidor STM indica que hubo algún error con la tarjeta, el prototipo desplegará en el
display los mensajes “Operación incompleta. Vuelva a intentarlo.”, “Disculpe las
molestias.”.
Si la tarjeta cuenta con saldo para cargar, lo indicará con un mensaje del tipo “Transfiriendo
50 pesos” o “Transfiriendo 10 viajes” en caso de ser una tarjeta de estudiante.
Si al intentar escribir la tarjeta, el usuario la retira, se le indicará “No mueva su tarjeta.”.

Si no se pudiera escribir en la tarjeta, “No se pudo grabar su tarjeta. Vuelva a intentarlo.”.
Al finalizar la transacción, indica al usuario el nuevo saldo con un mensaje del tipo “Su
saldo es de 50 pesos” o “Su saldo es de 10 viajes”.
Si algún imprevisto ocurriera, el prototipo desplegará los mensajes “Operación incompleta.
Vuelva a intentarlo.”, “Disculpe las molestias.”.
Si el dispositivo pierde conexión con el servidor STM intercala los mensajes
“Temporalmente sin servicio”, “Disculpe las molestias.”.

