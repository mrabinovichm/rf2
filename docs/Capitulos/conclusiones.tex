\chapter{Conclusiones}

\section{Conclusión final}
El mundo de la tecnología RFID está poco explorado en nuestro país, éste 
tal vez sea el primer proyecto que incluye el diseño y fabricación de una 
antena RFID capaz de operar con tarjetas sin contacto en la banda de
frecuencia de 13,56 Mhz.

Las posibilidades de aplicación son muy bastas, entre las que se pueden 
nombrar, sistemas de transporte, pasaporte electrónico, identificación 
civil, tarjetas de crédito y/o débito, salud, telefonía, control de acceso,
entre otras.

Nuestro aporte en este campo es tan solo una primera aproximación y aún 
queda mucho por hacer al respecto. Sobre este proyecto en particular es
necesario mejorar varios aspectos antes de pasar de la fase de prototipo
a la de producción.

Repasando en particular los criterios de éxito, el proyecto ha resultado muy bueno porque se logró construir un dispositivo capaz de consultar y recargar tarjetas RFID tal como se quería. Si bien no se logró llevar a cabo de la forma específica en que se planteó solucionarlo, se logró el objetivo planteado sorteando todos los problemas que se presentaron. De modo que se hizo lo necesario para suplir la solución planteada con soluciones alternativas que hicieran funcionar el prototipo RF$^{2}$. En particular, el módulo de seguridad SAM no realizó su papel por dos simples razones. La primera que realmente no se necesitó conexión a un servidor porque se descartó en la última replanificación, por lo tanto el SAM no tuvo que encargarse de hacer segura la transacción, pero la función del servidor debió ser suplida. Lo que se hizo a través de un simple archivo que contiene los montos que se desea acreditar a cada una de las tarjetas RFID y luego de acreditarlo se escribe un cero para no incurrir en el error de acreditarlo varias veces. La segunda que al no lograr comunicación entre la SBC y el lector de tarjetas de contacto, la SAM no necesitó devolver las claves de las tarjetas RFID sino que se insertaron directamente en el código para su uso dependiendo de la tarjeta RFID utilizada.

Además, si bien se diseñó enfocado a lo que es el actual sistema de transporte, con unos pocos cambios en el sotfware se puede emplear para variadas funcionalidades, como puede ser el acceso de personal a una empresa, el control de socios en un club deportivo, etc..

En suma, el equipo está conforme con el trabajo realizado.


\section{Ventajas y desventajas}
El prototipo diseñado tiene un costo de mantenimiento bajo. No utiliza papel, lo que lo hace más ecológico.

Entre las ventajas que se pueden hallar en este dispositivo es que el lector/escritor de tarjetas RFID es un diseño realizado en PCB de dos capas que lo hace más sencillo y económico que el diseño del lector/escritor OpenPCD, y al igual que este último es compatible con la biblioteca open source conocida como librfid.
Por su lado, en el lector de tarjetas de contacto se debe destacar su simplicidad, ya que no cuenta con ningún tipo de hardware específico (ASIC) que cumpla con el estandar ISO7816, sólo es necesario tener disponible un puerto serial (UART) y un par de puertos de entrada/salida de propósito general para lo que tiene que ver con el manejo del oscilador y el reset para la tarjeta de contacto. Esto lo hace portable a cualquier SBC que cuente con los puertos detallados anteriormente.


\section{Enseñanzas y aprendizajes}
Los conceptos aprendidos sobre reglas de diseño de lectores/escritores de 
tarjetas sin contacto compatibles con la norma ISO14443, son muy importantes
para llevar a cabo mejoras y modificaciones futuras. Las dificultades afrontadas
durante la fase inicial, generadas por errores en las ecuaciones suministradas
por una de las notas de aplicación usada y la falta de soporte técnico por
parte del fabricante (por no ser quien consultó una empresa), permitió ver la 
necesidad del uso de instrumentos de medición adecuados para validar el 
diseño y comparar con los resultados teóricos. También dejó como enseñanza el hecho de ver con ojo más crítico y no creer tanto en las publicaciones de otros a la hora de diseñar.

\bigskip
El manejo de herramientas de diseño CAD, fue de mucha utilidad para luego
llevar adelante las fabricación de las placas de circuito impreso, se tuvo que aprender a utilizar el programa gEDA ya que KiCad no permite hacer pistas curvas y eran necesarias para el diseño del lector/escritor RFID.

\bigskip
A nivel de software se dejó de ser un simple usuario del sistema operativo
Linux, adquiriendo conocimientos para cross compilar su Kernel e incorporando
modificaciones y parches que permitieran poner en funcionamiento el hardware
que conforma el prototipo. Se debió entender el funcionamiento de buena parte
de las bibliotecas usadas, para modificarlas y que fuera posible su uso en la 
aplicación final. 

\bigskip
En cuanto al cliente, se entendió que tiene tiempos distintos y finalmente no se logró tenerlo como tal. Entre otras cosas, el contacto encargado de ser el cliente por parte de la IM no estaba dispuesto a oficiar como tal por entender que en un proyecto de fin de carrera no podía exigir nada porque es algo didáctico. Se tuvo más bien el respaldo en cuanto a los préstamos de tarjetas RFID y SAM. Se aprendió que los mecanismos de compra en la IM son muy complejos, no se puede hacer un proyecto apoyado por la IM durante un cambio de dirección porque las prioridades cambian y puede no ser bueno. Además se aprendió que las compras se deben realizar en el momento necesario, y no cuando un tercero encuentre los medios disponibles, puesto que pueden no encontrarse nunca.

\bigskip
Por último, pero no menos importante, lo que parece salir en dos días puede demorar dos meses por lo que es recomendable al planificar sobredimensionar tiempos previendolo, además de calcular un período con el cual poder moverse con holgura si existen inconvenientes extra.



\section{Mejoras y trabajos a futuro}
Entre las cosas que se deben investigar es que el blindaje incluido en la 
primer antena fabricada cumpla con las regulaciones de compatibilidad 
electromagnética, EMC, definidas por la EN de Europa o por la FCC de Estados 
Unidos, a tales efectos sería necesario contar con el equipamiento adecuado
para efectuar las mediciones necesarias.
Tal vez la disminución del tamaño del PCB de la antena o la incorporación
de ésta al resto del hardware puedan ser vistas como mejoras desde el 
punto de vista de disminución de costos.

\bigskip
En cuanto al lector de tarjetas de contacto su incorporación a la lista
de lectores soportados por la biblioteca pcsclite quedó inconclusa; si 
bien el hardware está operativo y es posible enviar comandos APDU al 
módulo SAM de seguridad a través de una pequeña aplicación, sería 
importante alcanzar el objetivo planteado al comenzar el proyecto.
Un aspecto importante a destacar es que no se detallan los comandos
APDU empleados, ya que por motivos de seguridad los mismos no pueden 
ser revelados.

\bigskip
La comunicación por red a través de un modem 3G con un servidor central
tampoco fue alcanzada. La alternativa encontrada fue tener en forma local
un archivo que contenga los saldos pendientes para acreditar en las tarjetas
de los usuarios. Una vez asignado el saldo a la tarjeta correspondiente 
se elimina de la lista para que no se re asigne saldo ya acreditado.

\bigskip
Algo que no se tiene en cuenta muchas veces es qué tan simple puede resultar
el armado de las partes del dispositivo, en este caso se pensó en un 
método simple y rápido donde no se emplearan cables para los conexionados.
El ensamblado entre sí de las placas de hardware que conforman el prototipo es 
una tarea simple que puede ser hecha por un niño, como si armara su mecano.
Las distintas partes encajan una encima de otra y son aseguradas a través
de un separador con tuerca, no existiendo posibilidad que sean conectadas al 
revés. En cuanto a diseño industrial faltaría diseñar y fabricar una 
carcasa acorde, que permita el anclaje rápido y seguro de las distintas
partes de hardware, sin que el material del cual esté hecha interfiera
con la propagación de radio frecuencia, esto descarta la posibilidad de
usar metal como elemento a emplear.

\bigskip
Algo no previsto en el prototipo es el hecho de que ocurra un corte de luz. Puede solucionarse incluyendo en el diseño una batería, que se active al detectar un corte de luz, capaz de alimentar el dispositivo hasta terminar una posible transacción en curso y luego simplemente se ponga fuera de línea apagándose.

\bigskip
El sistema se ejecuta desde la memoria SD, existe la posibilidad que lo haga desde la NAND incluída en la SBC. Por tanto se podría estudiar la posibilidad de copiarlo allí para no depender de la memoria SD.

\bigskip
Linux no controla el hecho de que exista un módulo SPI conectado, se podría entonces idear un mecanismo de control de inserción de módulos en el puerto SPI.

\bigskip
En principio se pensó en crear el prototipo utilizando otra SBC, Hawkboard, por problemas en el hardware se descartó y se terminó utilizando la Beagleboard. Sería bueno intentar migrar el sistema a la Hawkboard para diseñar la solución propuesta originalmente.

\bigskip
Actualmente existe la versión 3.0 de kernel, se podría probar migrar de la versión que se usó (2.6.32) a la más actual.


