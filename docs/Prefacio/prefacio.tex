\begin{prefacio}

\begin{itshape}
“El ciudadano Línea saca su billetera, extrae su tarjeta y la introduce en la máquina registradora; una serie de gestos automáticos. Unas mandíbulas de aluminio se cierran sobre ella, unos dientes de cobre buscan la clave magnética, y una lengua electrónica saborea la vida del ciudadano Línea.
Lugar y fecha de nacimiento. Padres. Raza. Religión. Historial educativo, militar y de servicios civiles. Estado. Hijos. Ocupaciones, desde el comienzo hasta el presente. Asociaciones. Medidas físicas, huellas digitales, retínales, grupo sanguíneo. Grupo psíquico básico. Porcentaje de lealtad, índice de lealtad en función del tiempo hasta el momento del último análisis...
... El ciudadano Línea se encuentra en la ciudad donde, la noche anterior, dijo que estaría, así que no ha tenido que hacer una corrección.
Los nuevos informes se añaden al historial del ciudadano Línea. Toda su vida regresa al banco de datos. Desaparece de la unidad exploradora y la unidad comparativa, para que éstas atiendan la próxima llegada.
La máquina ha tragado y digerido otro día. Está satisfecha.”

\rightline{Sam Hall (1953), Poul Anderson}
\end{itshape}
\bigskip

La narración anterior es parte de un cuento de ciencia ficción llamado “Sam Hall”, escrita por Poul Anderson en 1953. En esta historia el autor describe un mundo donde cada persona tiene asignada una tarjeta conteniendo datos que la caracterizan, y puede ser controlado su accionar a través de una super computadora que almacena y procesa los datos de toda la humanidad. 
En nuestros días este cuento de ciencia ficción no está tan alejado de la realidad, las tarjetas “inteligentes” (smart cards) son cada vez más usadas en múltiples aplicaciones como ser, pasaporte electrónico, pago electrónico, sistemas de transporte, controles de acceso y sistemas de seguridad, entre otros.
El siguiente proyecto se desarrolla con la intención de aprender las bases del mundo de las tarjetas “inteligentes” y que sirva como punto de partida para que otros entiendan su funcionamiento. 
No es intención de los autores que se use el contenido de este documento con fines como los que se indicaban en la narrativa de ciencia ficción, muy por el contrario, el empleo de esta tecnología debe estar en favor de las personas y no en su contra.

\end{prefacio}